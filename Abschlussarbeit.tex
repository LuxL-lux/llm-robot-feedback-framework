\documentclass[
  12pt,
  titlepage,
  a4paper,
  abstract=true,
  oneside,
  openright,
  headsepline,
  cleardoublepage=plain,
  bibliography=totoc
]{scrreprt}

%% ++++++++++++++++++++++++++++++++
%% Includes
%% ++++++++++++++++++++++++++++++++

% Pakete
% === Document Layout & Geometry ===
\usepackage{geometry}
\usepackage{microtype} % Verbesserter Randausgleich bei Silbentrennung

% === Fonts (LuaLaTeX) ===
\usepackage{fontspec}

% Set Times New Roman as main font (12pt is set in document class)
\setmainfont{Times New Roman}

% Define RUB Flama font family
\newfontfamily\rubflama[
  Path = Includes/fonts/,
  Extension = .ttf,
  UprightFont = RUB-Flama-Regular,
  BoldFont = RUB-Flama-Bold,
  ItalicFont = RUB-Flama-Italic,
  FontFace = {l}{n}{RUB-Flama-Light-Regular},
  FontFace = {l}{it}{RUB-Flama-Light-Italic}
]{RUB-Flama}

% === Language & Typography ===
\usepackage[german,ngerman]{babel}
\usepackage{csquotes}

% === Mathematics ===
\usepackage{amsmath}
\usepackage{amsthm}
\usepackage{amscd}
\usepackage{amsfonts}
\usepackage{amssymb}
\usepackage{bbm}

% === Graphics & Colors ===
\usepackage{xcolor}
\usepackage{graphicx}
\usepackage{svg}
\usepackage{rotating}
\usepackage[percent]{overpic}
\usepackage{pgfplots}
\pgfplotsset{compat=1.18}

% === Tables ===
\usepackage{array}
\usepackage{longtable}
\usepackage{booktabs}
\usepackage{makecell}
\usepackage{tabularx}
\usepackage{threeparttable}
\usepackage{multirow}

% === Floats & Captions ===
\usepackage{float}
\usepackage{placeins}
\usepackage[]{caption}
\usepackage[normal]{subfigure}

% === Code Listings ===
\usepackage{listings}
\usepackage{minted}
\setminted[csharp]{style=friendly,breaklines,breakanywhere,fontfamily=tt}
\setminted[json]{style=friendly,breaklines,breakanywhere,fontfamily=tt}

% === Algorithms ===
\usepackage{algorithm}
\usepackage{algorithmic}

% === Bibliography ===
\usepackage[
  style=authoryear,       % Autor-Jahr Stil für kompakte Zitate
  backend=biber,
  autocite=footnote,
  maxcitenames=2,
  mincitenames=1,
  maxbibnames=99,
  uniquelist=false,
  uniquename=false,
  giveninits=true,        % Nur Initialen der Vornamen
  terseinits=true,        % Keine Punkte nach Initialen
  dashed=false
]{biblatex}

% === KOMA-Script & Page Layout ===
\usepackage{tocbasic}
\usepackage[automark]{scrlayer-scrpage}

% === Miscellaneous ===
\usepackage{pifont}
\usepackage{dirtree}
\usepackage[bottom,hang]{footmisc}

% === Hyperlinks (load late) ===
\usepackage{url}
\usepackage[
  pdfauthor={Vor- und Nachname},
  pdftitle={Titel der Arbeit},
  pdfsubject={Bachelor-Thesis, RUB},
  hypertexnames=false,
  pdfdisplaydoctitle
]{hyperref}

\addbibresource{literatur.bib}
% Befehle für f. und ff.
\newcommand{\psq}{\,f.}
\newcommand{\psqq}{\,ff.}

% Befehle für direkte und indirekte Zitate
\newcommand{\vglcite}[2][]{\footnote{Vgl. \citeauthor{#2}
\citeyear{#2}\ifx&#1&\else, S.~#1\fi}}
\newcommand{\directcite}[2][]{\footnote{\citeauthor{#2}
\citeyear{#2}\ifx&#1&\else, S.~#1\fi}}

% Hilfsbefehle für einzelne Zitate ohne Fußnote (für Mehrfachzitate)
\newcommand{\citesingle}[2][]{\citeauthor{#2}
\citeyear{#2}\ifx&#1&\else, S.~#1\fi}

% Befehle für mehrere Zitate in einer Fußnote
\newcommand{\vglcites}[1]{\footnote{Vgl. #1}}
\newcommand{\directcites}[1]{\footnote{#1}}

% "et al." statt "u.a."
\DefineBibliographyStrings{ngerman}{
  andothers = {et al\adddot},
}

\definecolor{rubgreen}{HTML}{8DAE10}


% Neue Befehle
%% +++++++++++++++++++++++++++++++++
%% Neue Befehle
%% +++++++++++++++++++++++++++++++++

\newcommand{\changefont}[3]{\fontfamily{#1} \fontseries{#2} \fontshape{#3} \selectfont} % Schriftart ändern

\newcommand{\code}[1]{\texttt{#1}} % Code-Umgebung im Fließtext
\newcommand{\bs}[1]{\boldsymbol{#1}} % Fett geschrieben
\newcommand{\sm}[1]{{\textit{\tiny #1}}} % Fett geschrieben
\newcommand{\veci}[5]{{_{#2}^\textit{\tiny #4} \bs{#1} ^{\textit{\tiny #5}}_{#3}}} % #1 Vektorname, #2 IndexLinksUnten, #3 IndexRechtsUnten, #4 IndexLinksOben, #5 IndexRechtsOben
\newcommand{\elei}[5]{{_{#2}^\textit{\tiny #4} {#1} ^{\textit{\tiny #5}}_{#3}}} % #1 Elementname, #2 IndexLinksUnten, #3 IndexRechtsUnten, #4 IndexLinksOben, #5 IndexRechtsOben
\newcommand{\mati}[5]{{_\textit{\tiny #5}^\textit{\tiny #3} \bs{#1} ^{\textit{\tiny #4}}_{#2}}} % #1 Matrixname, #2 IndexRechtsUnten, #3 IndexLinksOben, #4 IndexRechtsOben, #5 IndexLinksUnten
\newcommand{\vecdoti}[5]{{_{#2}^\textit{\tiny #4} \dot{\bs{#1}} ^{\textit{\tiny #5}}_{#3}}} % #1 Vektorname, #2 IndexLinksUnten, #3 IndexRechtsUnten, #4 IndexLinksOben, #5 IndexRechtsOben
\newcommand{\eledoti}[5]{{_{#2}^\textit{\tiny #4} \dot{#1} ^{\textit{\tiny #5}}_{#3}}} % #1 Elementname, #2 IndexLinksUnten, #3 IndexRechtsUnten, #4 IndexLinksOben, #5 IndexRechtsOben
\newcommand{\matdoti}[5]{{_\textit{\tiny #5}^\textit{\tiny #3} \dot{\bs{#1}} ^{\textit{\tiny #4}}_{#2}}} % #1 Matrixname, #2 IndexRechtsUnten, #3 IndexLinksOben, #4 IndexRechtsOben, #5 IndexLinksUnten
\newcommand{\vecddoti}[5]{{_{#2}^\textit{\tiny #4} \ddot{\bs{#1}} ^{\textit{\tiny #5}}_{#3}}} % #1 Vektorname, #2 IndexLinksUnten, #3 IndexRechtsUnten, #4 IndexLinksOben, #5 IndexRechtsOben
\newcommand{\eleddoti}[5]{{_{#2}^\textit{\tiny #4} \ddot{#1} ^{\textit{\tiny #5}}_{#3}}} % #1 Elementname, #2 IndexLinksUnten, #3 IndexRechtsUnten, #4 IndexLinksOben, #5 IndexRechtsOben
\newcommand{\matddoti}[5]{{_\textit{\tiny #5}^\textit{\tiny #3} \ddot{\bs{#1}} ^{\textit{\tiny #4}}_{#2}}} % #1 Matrixname, #2 IndexRechtsUnten, #3 IndexLinksOben, #4 IndexRechtsOben, #5 IndexLinksUnten
\newcommand{\vecr}[3]{\left(\begin{array}{rrr}#1\\ #2 \\ #3 \end{array}\right)} % Vektor mit 3 Elementen, rechtsbündig
\newcommand{\vecc}[3]{\left(\begin{array}{ccc}#1\\ #2 \\ #3 \end{array}\right)} % Vektor mit 3 Elementen, zentriert
\newcommand{\matr}[9]{\left(\begin{array}{rrr}#1 & #2 & #3 \\ #4 & #5 & #6 \\ #7 & #8 & #9 \end{array}\right)} % Matriz mit 3 Zeilen und 3 Spalten, rechtsbündig
\newcommand{\matc}[9]{\left(\begin{array}{ccc}#1 & #2 & #3 \\ #4 & #5 & #6 \\ #7 & #8 & #9 \end{array}\right)} % Matrix mit 3 Zeilen und 3 Spalten, zentriert
\newcommand{\vecri}[6]{\left(\begin{array}{rrr}#1\\ #2 \\ #3 \\ #4\\ #5 \\ #6 \end{array}\right)} % Vektor mit 6 Elementen, rechtsbündig
\newcommand{\vecci}[6]{\left(\begin{array}{ccc}#1\\ #2 \\ #3 \\ #4\\ #5 \\ #6 \end{array}\right)} % Vektor mit 6 Elementen, zentriert
\newcommand{\vecrii}[2]{\left(\begin{array}{rr}#1\\ #2 \end{array}\right)} % Vektor mit 2 Elementen, rechtsbündig
\newcommand{\veccii}[2]{\left(\begin{array}{cc}#1\\ #2 \end{array}\right)} % Vektor mit 2 Elementen, zentriert


% Header
%% +++++++++++++++++++++++++++++++++
%% Header
%% +++++++++++++++++++++++++++++++++

% Seitengeometrie
\geometry{a4paper,inner=32mm,outer=24mm,top=30mm,bottom=40mm}

% Farben definieren
\definecolor{lightgrey}{rgb}{0.99,0.99,0.99}
\definecolor{colKeys}{rgb}{0,0,1}
\definecolor{colIdentifier}{rgb}{0,0,0}
\definecolor{colString}{rgb}{0,0.5,0}
\definecolor{darkred}{rgb}{0.5,0,0}
\definecolor{darkgreen}{rgb}{0,0.5,0}
\definecolor{darkblue}{rgb}{0,0,0.5}
\definecolor{green}{HTML}{009846}
\definecolor{blue}{rgb}{0,0,0.7}
\definecolor{red}{rgb}{0.7,0,0}
\definecolor{black}{rgb}{0,0,0}

\definecolor{gray1}{Gray}{1} %dunkelgrau
\definecolor{gray2}{Gray}{7} %grau
\definecolor{gray3}{Gray}{11} %hellgrau

% Quellcode
\lstloadlanguages{XML} 
\lstset{
		language=C++,
    float=hbp,
    keywordstyle=\color{colKeys},
    stringstyle=\color{colString},
    commentstyle=\color{gray3},
     basicstyle=\ttfamily,
		%basicstyle=\footnotesize\ttfamily,
		%basicstyle=\texttt\small,
    identifierstyle=\color{colIdentifier},
    columns=flexible,
    tabsize=2,
    frame=bt,
    extendedchars=true,
    showspaces=false,
    showstringspaces=false,    
		numbers=left, %numbers=none,
    numberstyle=\tiny,
    breaklines=true,
    breakautoindent=true,
		captionpos=t,
		xleftmargin=\fboxsep,
		xrightmargin=\fboxsep,
		frameround=tttt,
		mathescape=true,
		numberblanklines=false,
}

\lstdefinestyle{bash}{
	language=bash,	
	xleftmargin={1.0\parindent},
	keywordstyle=\bfseries\color{black},
	stringstyle=\color{black},
	numbers=none,
	frame=none,
}


\newcommand{\dollar}{\mbox{\textdollar}}

% Kopf- und Fußzeilen
\pagestyle{scrheadings}

\clearscrheadfoot

\lehead[]{\small{\textnormal{\headmark}}}
%\rehead[]{\small{\textnormal{\pagemark}}}
%\lohead[]{\small{\textnormal{\pagemark}}}
\rohead[]{\small{\textnormal{\headmark}}}


\ifoot[]{}
\cfoot[\pagemark]{\pagemark}% Seitenzahl (c = centered) 
\ofoot[]{}

% Absätze
\setlength{\parindent}{3ex}
\setlength{\parskip}{0pt}

% Zeilenabstand 
\usepackage{setspace}            % Zeilenabstand einstellbar
%\onehalfspacing                  % eineinhalbzeilig einstellen

% Fußnoten
\setlength{\footnotemargin}{0pt}

% Einstellungen für die PDF-Erzeugung
\hypersetup{
colorlinks, % Auskommentieren zum Ausdrucken
linkcolor=black,
filecolor=black,
urlcolor=black,
citecolor=black
}

\def\topfraction{1.0} 
\def\bottomfraction{1.0} 
\def\textfraction{0.0}

\renewcommand*{\partpagestyle}{empty}

% Hurenkinder und Schusterjungen verhindern
\clubpenalty10000
\widowpenalty10000
\displaywidowpenalty=10000

\interfootnotelinepenalty=10000

% Silbentrennung
\hyphenation{
  %
	Lagrange
	Schwenk-arm
	Hufnagel
	Reichert
	Jacobi
	Re-kon-fi-gu-ra-ti-ons-stra-te-gie
	Surdilovic
	Radojicic
	LogistikRuhr
	Bruckmann
	Segesta
	Um-lenk-rolle
	pa-ral-lel-ki-ne-ma-tische
	Ge-lenk-raum
	an-ta-go-nis-tisch
	an-ta-go-nis-tisch-en
	Aug-ment-ed
	Seil-aus-tritts-punk-te
	Seil-füh-run-gen
	Li-near-ein-heit	
}

% Römische Zahlen
\newcommand{\rom}[1]{\MakeUppercase{\romannumeral #1}}

% grid style
\pgfplotsset{grid style={dotted,gray}}

%pgf plots
\usepgfplotslibrary{patchplots}

%% +++++++++++++++++++++++++++++++++
%% Start des Dokuments
%% +++++++++++++++++++++++++++++++++

\raggedbottom

\begin{document}
% Apply special margins for title and front matter
\newgeometry{left=30mm,right=30mm,top=25mm,bottom=25mm}

% Titelseite
\begin{titlepage}
  \rubflama % Apply RUB Flama font for the entire title page
  \thispagestyle{empty}
  \noindent
  \begin{minipage}{0.5\textwidth}% adapt widths of minipages to your needs
    \includegraphics[width=56.9mm]{Figures/RUB-Logo-blau.png}
  \end{minipage}%
  \hfill%
  \begin{minipage}{0.5\textwidth}\raggedleft
    {\fontsize{9}{12}\selectfont \textcolor{rubgreen}{\bfseries FAKULTÄT FÜR
    MASCHINENBAU}}\par
    {\fontsize{8}{12}\selectfont \bfseries
      Institute Product and Service Engineering\\
      Lehrstuhl für Produktionssysteme\\
    PROF.\,DR.-ING.\,BERND KUHLENKÖTTER}%
  \end{minipage}
  \begin{center}
    {\bfseries \fontsize{16}{12}\selectfont Bachelor-Thesis}\par
    {Lukas Lux}
  \end{center}
  \fontsize{10}{12}\selectfont Matrikelnummer: \quad 1080 18 240012\par
  Prüfungsordnung: \quad BPO Sales Engineering and Product
  Management 2013
  \bigbreak
  \fontsize{12}{12}\selectfont{\bfseries Thema: \quad Entwicklung
    eines Frameworks zur simulationsbasierten Validierung
  LLM-generierten Robotercodes in Unity}
  \bigbreak
  \fontsize{10}{12}\selectfont Die Programmierung von Robotern ist derzeit ein
  komplexer und  zeitintensiver Prozess, der in der Regel spezielles Fachwissen
  erfordert. Um diese Hürde zu verringern, erforscht die Wissenschaft
  neue Ansätze zur Beschleunigung der Arbeitsprozesse. Generative
  KI-Systeme, insbesondere Large-Language-Modelle (LLMs), bieten
  dabei ein vielversprechendes Potenzial.
  Eine zentrale Herausforderung besteht jedoch im sicheren Einsatz
  von KI-generiertem Quellcode. Am LPS wird deshalb untersucht, wie
  Simulationstools genutzt werden können, um generierte Lösungen
  zunächst virtuell zu erproben. Die dabei gewonnenen Erkenntnisse
  sollen wiederum in die KI-Systeme zurückgeführt werden. Damit dies
  gelingt, ist eine Überführung der Simulationsergebnisse in eine für
  LLMs verständliche, textuelle bzw. natürlichsprachliche Form erforderlich.
  \bigbreak
  Ziel dieser Arbeit ist es, eine solche Überführung anhand eines
  Beispielprozesses und ausgewählter Simulationsparameter zu
  entwickeln, zu implementieren und methodisch zu evaluieren.
  \bigbreak
  Im Einzelnen sollen folgende Punkte bearbeitet werden:
  \begin{itemize}
    \item Analyse des aktuellen Stands der Technik in den Bereichen
      Roboterprogrammierung sowie Bewertung von Robotersimulationen.
    \item Konzeption einer Basis-Architektur zur Übertragung von
      Erkenntnissen aus Simulationen in eine geeignet
      weiterzuverarbeitende Form (z. B. JSON).
    \item Programmiertechnische Umsetzung der Architektur in Unity
      für ein definiertes Beispielszenario.
    \item Methodische Erprobung der entwickelten Lösung und Bewertung
      der erzielten Ergebnisse.
  \end{itemize}
  \bigbreak
  Die Arbeit leistet einen wichtigen Beitrag zum Gelingen des
  Projektes XYZ am Lehrstuhl
  für Produktionssysteme.
  \bigbreak
  Ausgabedatum: \quad 16.06.2025\par
  Betreuer: M. Sc. Daniel Syniawa

\end{titlepage}

\pagestyle{plain}

% Schriftart der Überschriften
\addtokomafont{disposition}{\rmfamily}
% Set font sizes for headings with 1.5 line spacing
\addtokomafont{chapter}{\fontsize{16}{24}\selectfont}% 16pt × 1.5 = 24pt line spacing
\addtokomafont{section}{\fontsize{14}{21}\selectfont}% 14pt × 1.5 = 21pt line spacing

% Seitennummerierung
\pagenumbering{roman}
\setcounter{tocdepth}{2}
\pagestyle{scrheadings}

% Inhaltsverzeichnis
\tableofcontents

% Abbildungsverzeichnis
\listoffigures

% Tabellenverzeichnis
\listoftables

% Restore normal page margins for main content
\restoregeometry

% Hauptkapitel
\cleardoublepage
\pagenumbering{arabic}
\sloppy

% Hier die neuen Kapitel-Dateien einbinden
% \chapter{Einleitung}
% \label{cap:Einleitung}

% \section{Motivation und Relevanz}
% \label{sec:Motivation}
% Hier fügen Sie den Text für die Motivation ein.
% Warum lohnt es sich mit dem Thema zu beschäftigen
% Wie relevant ist es in Forschung sowie Praxis

% \section{Zielsetzung und Aufbau der Arbeit}
% \label{sec:Zielsetzung}
% Was will ich mit der Arbeit erreichen
% Versuchsaufbau
% In kapitel 2 wird, Kapitel 3 das, 4 das, 5 das
% Hier fügen Sie den Text für die Zielsetzung ein.
% Entwicklungsziel: Konzeption und prototypische Implementierung...
% Untersuchungsziel: Methodische Untersuchung der Formalisierung...
% Überblick über den Aufbau der Arbeit...
%

\chapter{Einleitung}
\label{cap:Einleitung}

\section{Motivation und Relevanz}
\label{sec:Motivation}

Die technologische Entwicklung und die zunehmende Vielfalt
an Produkt- und Variantenvielfalt in der Fertigungsindustrie führt dazu, dass
industrielle Produktionsanlagen immer häufiger neu eingerichtet oder
umgerüstet werden müssen. In der Robotik erfordert dies die
Erstellung und Anpassung von Programmen, die Bewegungsabläufe,
Bearbeitungsschritte und Sicherheitsfunktionen eines Roboters
definieren.\vglcites{\citesingle{pine1993};
  \citesingle{elmaraghy2005}; \citesingle{wiendahl2007};
\citesingle{koren1999}; \citesingle{biggs2003}}
Traditionelle Programmiermethoden sind komplex,
herstellerspezifisch und setzen ein profundes Verständnis der
jeweiligen Kinematik und proprietären Sprachen
voraus.\vglcite[116]{lambrecht2011} Gleichzeitig halten generative
Sprachmodelle Einzug in die
Softwareentwicklung. Sie können aus natürlichsprachlichen
Beschreibungen ausführbaren Code erzeugen und versprechen so Potenzial, die
Hürden der Roboterprogrammierung zu
senken.\vglcites{\citesingle{salimpour2025}; \citesingle{brohan2023};
\citesingle{liang2023}} In der Praxis ist der
direkte Einsatz von Large Language Models (LLMs) zur Generierung von
Roboterprogrammcode im industriellen Kontext jedoch riskant:
Fehlerhafte Bewegungssequenzen
oder fehlerhafte Prozesslogiken können zu Kollisionen, Schäden und
Produktionsausfällen führen.\vglcite[4]{bilancia2023} Daher ist eine
sorgfältige Validierung des
generierten Codes unerlässlich. Darüber hinaus stellt die
Generierung von Roboterprogrammcode durch LLMs eine signifikante Hürde durch das
Fehlen des Verständnisses der physikalischen Welt
dar.\vglcite[1/psqq]{cohen2024}

\section{Zielsetzung und Aufbau der Arbeit}
\label{sec:Zielsetzung}
Ziel dieser Arbeit ist die Konzeption, Umsetzung und Evaluation eines
Frameworks zur simulationsbasierten Validierung von mit LLMs
generiertem Robotercode. Kernidee ist, durch die simulierte Ausführung von
Roboterprogrammcode in einer modellierten, der avisierten
Arbeitsumgebung des Roboters
entsprechenden Simulation auftretende unerwünschte Ereignisse
aufzuzeichnen und formalisiert zu dokumentieren. So kann Roboterprogrammcode
getestet werden und fehlerhaftes Verhalten wie falsche Prozessabfolgen,
Kollisionen, Singularitäten sowie Gelenkgeschwindigkeits- und
Beschleunigungsüberschreitungen frühzeitig erkannt und berichtigt werden.
Erkannte Fehlerereignisse sollen mit weiteren, der Analyse und Fehlersuche
behilflichen Daten anreichert werden.

Dazu wird in Kapitel~\ref{cap:Grundlagen}, wird zunächst der Stand
der Technik zu
Robotersimulation, Offline‑Programmierung und LLMs zusammengefasst und
herausgestellt, inwiefern die Notwendigkeit der Aufbau eines solches Frameworks
besteht und welche Rahmenbedingungen dazu zu beachten sind.
Darauf aufbauend wird in Kapitel~\ref{sec:framework} anschließend die
Architektur des zu
entwickelnden Frameworks beschrieben, welches die Einbindung einer
virtuellen Robotersteuerung in der Entwicklungs- und
Modellierungsumgebung Unity3D vorsieht.

Weiterführend werden vier verschiedene Module zur Analyse des Roboterverhaltens
implementiert werden. Dabei beschränkt sich diese Arbeit auf die Erkennung von
falschen Prozessfolgen innerhalb eines Roboterprogramms, der Kollisionserkennung
des Roboters mit seiner Umgebung, der Singularitätserkennung sowie der
Geschwindigkeits- und Beschleunigungsüberschreitung von Robotergelenken.

In einem realitätsnahen Beispielszenario einer Roboterzelle werden die
Funktionalitäten des Frameworks getestet und durch spezifische
Szenarien fehlerhaftes Roboterverhalten zu provozieren und die Detektion dessen
durch das Framework zu
verifizieren. Ein Experteninterview ergänzt die Evaluation und
reflektiert die Eignung des Ansatzes aus praktischer Sicht.

Abschließend werden die gewonnenen Erkenntnisse diskutiert, Limitationen
des Prototyps aufgezeigt und in Zusammenhang mit einer Nutzung von LLMs zur
Generierung und Verbesserung von Roboterprogrammcode gebracht.

\chapter{Stand der Technik} \label{cap:Grundlagen}

\section{Roboterprogrammierung}
Die Roboterprogrammierung kann als die
Programmierung von industriellen Manipulatoren verstanden werden, die sich durch
ihre programmierbaren und anpassungsfähigen Eigenschaften von anderen Maschinen
abheben. Roboterprogramme enthalten präzise Anweisungen und Spezifikationen für
die Bewegung des Roboters. Der entscheidende
Vorteil der Roboterprogrammierung ist somit Flexibilität: Roboter
können durch einfache Software-Umprogrammierung für völlig unterschiedliche
Aufgaben eingesetzt werden, während herkömmliche Steuerungstechnik meist auf
vordefinierte starre Abläufe beschränkt ist.\vglcite[1\psqq]{nilsson1996}

\subsection{Verfahren der Roboterprogrammierung}
Grundsätzlich wird bei der Roboterprogrammierung zwischen manuellen und
automatischen Verfahren unterschieden.\vglcite[1]{biggs2003}
Manuelle Systeme erfordern die explizite Erstellung des Programms
durch den Anwender, wobei textbasierte Programmiersprachen (z.\,B.
  herstellerspezifische Sprachen wie KUKA Robot Language (KRL) oder ABB
RAPID) sowie grafische Oberflächen genutzt werden. Textbasierte und
im eingeschränkten Umfang auch grafische Verfahren bieten
die Möglichkeit, roboterspezifische
Datentypen zu deklarieren, einfache Bewegungen zu spezifizieren und
mit Werkzeugen und Sensoren zu interagieren. Zur Ausführung von
erstelltem Roboterprogrammcode wird dieser an die Robotersteuerung übertragen
und unter Einhaltung von Echtzeitbeschränkungen
ausgeführt.\vglcite[1]{muehe2010}

Automatische Systeme der Programmierung werden hingegen in
lernende Systeme, instruktive Systeme sowie demonstrative Systeme
unterteilt. Ein typisches Verfahren demonstrativer Programmierung ist das
Führen des Roboters über ein Teach Pendant oder durch die direkte
Führung der Roboterhand durch den Arbeitsbereich, wobei die dabei
entstehenden Gelenkwinkel oder
kartesische Tool Center Point Positionen (TCP-Positionen)
aufgezeichnet und im Programm hinterlegt werden. Lernende und instruktive System
nutzen Methoden des maschinellen Lernens (z.\,B. Reinforcement Learning) sowie
computerbasiertem Sehen (engl. Computer-Vision) zum Verständnis der Intention
und der Umwelt, oft in Rahmen der Mensch-Maschine Kollaboration
(MRK).\vglcite[4\psqq]{biggs2003}

Ergänzend zur methodischen Einordnung
wird bei der Roboterprogrammierung zwischen zwei
Durchführungsmodi unterschieden:
Online-Programmierung am realen System (z.\,B. Teachen per Teach
Pendant oder Handführung), bei der Wegpunkte und Ablauflogik
unmittelbar auf der Steuerung aufgezeichnet werden, und
Offline-Programmierung (OLP), bei der Programme in
virtuellen Zellen entworfen, geprüft und erst anschließend auf die
reale Steuerung übertragen
werden.\vglcites{\citesingle[1]{biggs2003}; \citesingle[62\psqq]{holubke2014}}

Online-Verfahren sind im industriellen Alltag verbreitet, verursachen
jedoch Stillstandszeiten und bergen
Test-Risiken.\vglcite[4]{bilancia2023} Hierbei können Beschädigungen am
Roboter, Maschinen, Werkstücken und Umwelt durch Kollisionen oder fehlerhafter
Konfiguration des Roboters auftreten. OLP verlagert
Entwurf, Kollisionsprüfung und Taktzeitabschätzung in die Simulation
und reduziert so Risiken und Anlagenstillstand.\vglcite[62\psqq]{holubke2014}
Somit wird die Entwicklung und Evaluierung
von Roboterprogrammen ohne physischen Roboter
ermöglicht: Programme werden in einer virtuellen Umgebung
erstellt, getestet und
iterativ verfeinert. Hierfür werden originalgetreue dreidimensionale
Computer Aided Design Modelle (3D-CAD-Modelle) des Roboters
und seiner Umgebung benötigt, welche teilweise vom Hersteller
bereitgestellt werden oder nachmodelliert werden müssen, um den
Zielkontext möglichst realitätsnah abzubilden. Ziel ist es durch die OLP
problematische Stellen des Roboterprogramms, Ablaufprobleme sowie
Nebenwirkungen des direkt und indirekt beeinflussten
Prozesses früh zu erkennen.
\vglcite[62\psqq]{holubke2014} Externe
Faktoren wie verformbare Objekte, Fluide oder Personen bei der MRK erhöhen im
Rahmen solcher Simulationen maßgeblich die
Komplexität. In realitätsnahen Szenarien (z.\,B. automatisierte
Steckverbindungen in Schaltschränken oder Gießen von Schmelzen) treten
materialbedingte Nichtidealitäten auf, die zu Wechselwirkungen mit dem Roboter
führen. Eine hinreichend genaue physikalische Modellierung des Zielsystems ist
daher Voraussetzung.

\subsection{Landschaft gängiger Programmierumgebungen}
Im Gegensatz zur
Werkzeugmaschinenprogrammierung, die auf standardisierten Sprachen wie G-Code
basiert, erschwert das Fehlen einer universellen, herstellerunabhängigen
Programmiersprache die Integration verschiedener Robotertechnologien in einer
Produktionsanlage.\vglcite[4]{bilancia2023} Industrielle
Roboterhersteller, beispielsweise KUKA, ABB,
Fanuc oder Stäubli, bieten und unterstützen lediglich eigene,
proprietäre Programmiersprachen
und Programmierschnittstellen, wobei diese sich in
Komplexität, Syntax und Semantik
unterscheiden.\vglcites{\citesingle[116]{lambrecht2011};
\citesingle[4]{bilancia2023}} Zudem müssen Roboterprogrammierer mit
eingeschränkten Basisbefehlen
und Bibliotheken arbeiten. Diese decken zwar die meisten Standardanforderungen
ab, ermöglichen jedoch keine fortgeschrittenen Berechnungen oder komplexen
Steuerungsstrategien. Offline-Programmierwerkzeuge wie RoboDK oder Siemens
Process Simulate übersetzen 3D-Modellierungsbefehle mithilfe spezifischer
Postprozessoren in herstellerspezifische Robotercodes. Allerdings unterstützen
diese Werkzeuge nicht die vollständigen Funktionsbibliotheken der kommerziellen
Robotersprachen und können erfahrene Programmierer bei komplexen
Programmierroutinen nicht ersetzen.\vglcite[4]{bilancia2023} Der
Versuch, eine einheitliche, standardisierte Programmiersprache für
alle Industrieroboter zu definieren und zu
verbreiten, scheitert dabei an der mangelnden Kooperation der
Hersteller industrieller Roboter.\vglcite[116]{lambrecht2011}

Einen gegensätzlichen Ansatz verfolgt das open-source Projekt
Robot Operating System (ROS). ROS ist ein quelloffenes
Middleware-Framework, das Bibliotheken und Werkzeuge für
Nachrichtenübertragung, Paketverwaltung und Hardwareabstraktion
bereitstellt und damit eine hersteller- und
plattformunabhängige Integrationsschicht
anstrebt.\vglcite[1]{quigley2009ros} In der industriellen Praxis wird
ROS/ROS~2 primär als Integrations-/Orchestrierungsebene
eingesetzt (z.\,B. Wahrnehmung, Planung, Zellenkoordination).
Echtzeitkritische Bewegungs- und Sicherheitsfunktionen verbleiben
üblicherweise auf herstellerseitig eingesetzten Robotersteuerungen, und der
dokumentierte industrielle Einsatz bleibt anwendungs- und
treiberabhängig.\vglcite[3-6]{bonci2023ros2} Da ROS~1
keine harten Echtzeitanforderungen der Robotersteuerung erfüllen
kann, soll ROS~2 dieses Defizit erfüllen.
Das erfordert in der Praxis jedoch ein echtzeitfähiges
Betriebssystem sowie sorgfältiges Systemtuning,
sodass ein reproduzierbares, deterministisches Ausführungsverhalten
weiterhin von Konfiguration und
Implementierung
abhängt.\vglcites{\citesingle[1]{maruyama2016ros2};
\citesingle[3-5]{bonci2023ros2}}

\subsection{Physik-Engines zur Simulation von Robotern}
Aufgrund der herstellerabhängig begrenzten Möglichkeiten der
realistischen Abbildung eines Szenarios bei der OLP, bietet es sich
an, auf Programme zur physikalischen Simulation von Körpern und Prozessen
zurückzugreifen. Mit dem Einsatz von
Physik-Engines soll eine virtuelle Umgebung geschaffen werden, in
dem die Realität nahezu realgetreu digital abgebildet werden kann
und so eine Simulation des Roboters und seiner Umgebung
geschaffen werden. Sie modellieren dynamische Interaktionen wie Kollisionen,
Schwerkraft und Reibung, was für die präzise Nachbildung des Roboterverhaltens
entscheidend ist. Obwohl die Genauigkeit dieser Engines als nicht perfekt
angesehen wird, da sie die reale Welt nicht exakt
abbilden\vglcite[1\psqq]{audonnet2022}, sind sie für
Forschung und Entwicklung essenziell, um physikalisch anspruchsvolle Prozesse
zuverlässig simulieren zu können. Die Wahl der Physik-Engine
beeinflusst die Stabilität und Wiederholbarkeit der
Simulation. Häufig genutzte Physik-Engines sind PhysX, Bullet und
ODE. In Tabelle \ref{table:simuplattform} werden gängige
Simulationsprogramme im Hinblick auf die verwendeten Physik-Engines,
ROS-Kompatibilität und deren Möglichkeit der Integration verglichen.
In der Praxis besteht ein Zielkonflikt zwischen numerischer Stabilität,
Reproduzierbarkeit (Determinismus) und physikalischer Genauigkeit. Die Wahl der
Engine beeinflusst daher nicht nur die Plausibilität der Dynamik, sondern auch
die Vergleichbarkeit von Simulationsergebnissen.\vglcite[1\psqq]{audonnet2022}

\begin{table}
  \begin{tabularx}{\columnwidth}{X|X|X|X|X} \toprule
    \thead{\textbf{Name}}        & \thead{\textbf{Physik- \newline Engine}} &
    \thead{\textbf{Open Source}} & \thead{\textbf{ROS-Integration}}         &
    \thead{\textbf{ML-Support}}
    \\ \midrule Gazebo                & Bullet, DART,
    ODE, Simbody                 & Ja                                       & Ja
    & Extern                                                \\ \hline Ignition
    & DART                                     & Ja
    & Ja                                       & Extern
    \\ \hline Webots                & ODE                                      &
    Ja                           & Ja
    & Extern
    \\ \hline Isaac Sim             & PhysX                                    &
    Nein                         & Ja
    & Integriert
    \\ \hline Unity                 & Havok, PhysX, RaiSim                     &
    Nein                         & Nein
    & Extern
    \\ \hline PyBullet              & Bullet                                   &
    Ja                           & Nein
    & Extern
    \\ \hline CoppeliaSim (V-rep)   & Newton, Bullet, ODE,Vortex Dynamics      &
    Nein                         & Ja
    & Extern
    \\ \hline Mujoco                & Mujoco                                   &
    Ja                           & Nein
    & Extern
    \\ \bottomrule
  \end{tabularx} \caption{Vergleich verschiedener
  Robotik-Simulationsplattformen, nach Bilancia}
  \label{table:simuplattform}
\end{table}

\section{Large Language Models} \label{sec:Grundlagen_LLMs}
Aufgrund kürzlicher Entwicklungen im Bereich der LLMs bieten diese durch ihre
Fähigkeit der Generierung textueller Inhalte eine vielversprechende Möglichkeit,
bei der Programmierung von Robotern zu unterstützen.

\subsection{Funktionsweise und Architektur}
Bei großen Sprachmodellen (LLMs) handelt es sich um statistische Modelle,
welche in der Lage sind textuelle Inhalte zu übersetzen, zusammenzufassen,
Informationen abzurufen und Konversation zu betreiben. Historisch sind LLMs aus
der Möglichkeit, neuronale Netze im Modus des self-supervised learning (deutsch:
selbst überwachtes Lernen) und anhand großer Mengen textueller
Trainingsdaten zu trainieren, entstanden.
Dabei haben sich LLMs innerhalb der letzten 10 Jahre aufgrund der
hohen Verfügbarkeit digitaler
textueller Daten sowie Innovationen im Bereich der Hardware-Technologie zur
wichtigsten Technologie im Bereich Künstliche Intelligenz (KI)
entwickelt. \vglcite[1\psq]{naveed2024}\\

  Den entscheidenden Durchbruch für moderne LLMs lieferte die
Transformer-Architektur, die sequenzielle Abhängigkeiten durch parallele
Attention-Mechanismen ersetzt.\vglcite[1\psqq]{vaswani2023attentionneed}
Mithilfe des
Self-Attention-Mechanismus wird für jedes Token die Relevanz zu allen
anderen Tokens einer Sequenz berechnet, wodurch ein Modell
kontextuelle Beziehungen direkt
erfasst, ohne Informationen sequenziell durch versteckte Schichten propagieren
zu müssen. Diese Parallelisierung ermöglicht nicht nur schnelleres Training auf
den erwähnten großen Datensätzen, sondern schafft auch die Grundlage für das
Skalierungsverhalten moderner Sprachmodelle. Folglich basieren aktuelle LLMs wie
GPT-4 oder Claude auf dieser Architektur.

\subsection{Programmcodegenerierung durch Large Language Models}%
LLMs haben bemerkenswerte Fortschritte in der automatischen
Code-Generierung erzielt und revolutionieren damit die Softwareentwicklung.
Code-LLMs sind in der Lage erfolgreich Quellcode aus natürlichsprachlichen
Beschreibungen zu generieren. Aktuelle LLMs demonstrieren anhand von
empirischen Benchmarks zur Evaluation der Qualität von
Programmiercode (z.\,B. HumanEval, MBPP und BigCodeBench), dass
sie in der Lage sind progressiv bessere Leistungen bei verschiedenen
Schwierigkeitsgraden und Programmieraufgaben zu erzielen. \vglcite[1]{jiang2024}
Voraussetzungen für die erfolgreiche Programmierung mithilfe von LLMs sind die
Miteinbeziehung von Programmiersprachen und dessen technische und syntaktische
Dokumentation in die Trainingsdaten des
jeweiligen Modells.

Domänenspezifische Code-Generierung stellt LLMs vor besondere
Herausforderungen, die deren praktische Anwendbarkeit erheblich einschränken.
Neue und weniger populäre Programmiersprachen (engl.: low-ressource
languages) sowie
domänenspezifische Programmiersprachen sind in
offen abrufbaren Datensätzen oft unterrepräsentiert, was zu einem Mangel
verfügbarer Trainingsdaten und erhöhten
Hürden durch spezialisierte Syntax führt.\vglcite[1]{joel2024} LLMs zeigen
zudem schwächere Leistungen beim Umgang mit domänenspezifischen Bibliotheken
\vglcite[1]{gu2025} – zusätzlicher Kontext (z.\,B. Repository-Code,
Schnittstellenbeschreibungen) ist hier erforderlich. Diese Problematik betrifft
Millionen von Entwicklern: Beispielsweise 3,5 Millionen Rust-Nutzer, einer
performanten und vergleichsweise neuartigen Programmiersprache,
können LLM-Funktionen nicht vollständig ausschöpfen.

Agentische KI-Systeme entwickeln sich zu einer neuen Generation
autonomer Softwareagenten, die komplexe Aufgaben selbstständig ohne
kontinuierliche
menschliche Anleitung ausführen können. Diese Systeme adressieren teilweise die
zuvor beschriebenen Herausforderungen bei domänenspezifischer Code-Generierung,
indem sie über Command-Line-Interfaces und externe Tools Zugang zu
spezialisierten Bibliotheken und Applicatio Programming Interfaces
(APIs) erhalten. Das Model Context Protocol (MCP)
etabliert sich als offener Standard zur Verbindung von AI-Assistenten mit
Datensystemen, Business-Tools und Entwicklungsumgebungen.\vglcite{anthropic2024}
Viele Softwareentwicklungstools bieten bereits MCP-Unterstützung, um KI-Agenten
besseren Zugang zu domänenspezifischen Kontextinformationen und
Code-Repositories zu ermöglichen. Die Standardisierung solcher Protokolle kann
eine Lösung für die Datenmangel-Problematik bei low-ressource languages
und domänenspezifischen Programmiersprachen darstellen, wobei die praktische
Wirksamkeit noch evaluiert werden muss. Für Robotik-Anwendungen ist dabei
entscheidend, dass agentische Systeme nicht nur natürlichsprachliche
Spezifikationen interpretieren, sondern deterministische, zeitkritische
Ausführungspfade erzeugen und an bestehende Steuerungsstacks andocken können.

\section{LLMs in der Robotik}

Die Integration von LLMs in die Robotik verfolgt drei
komplementäre Paradigmen: Vision-Language-Action Modelle, Code-Generation und
Embodied Reasoning.\vglcite[2\psqq]{salimpour2025}

Die zentralen Linien lassen sich knapp gliedern:
\begin{itemize}
  \item \textbf{Vision-Language-Action (VLA)}: Gemeinsame
    Repräsentation von Wahrnehmung, Sprache und Aktion.
  \item \textbf{Code-Generierung}: Synthese ausführbarer
    Kontrollprogramme aus natürlichsprachlichen Spezifikationen.
  \item \textbf{Embodied Reasoning}: Multimodale Modelle mit
    kontinuierlichen Sensordaten und Weltmodellen.
\end{itemize}

\subsection{Aktuelle Forschungsansätze}
Google DeepMinds RT-2 repräsentiert den Vision-Language-Action Ansatz, bei dem
Roboter-Aktionen als Text-Token behandelt und gemeinsam mit visuellen und
sprachlichen Daten trainiert werden. Dieser Ansatz ermöglicht emergente
Fähigkeiten wie das Verstehen von Zahlen oder Icons ohne explizites
Training.\vglcite[1\psqq]{brohan2023} Code as Policies verfolgt hingegen die
direkte Generierung von ausführbarem Python-Code aus natürlichsprachlichen
Befehlen, wobei durch hierarchische Code-Generation komplexe
Kontrollstrukturen wie
Schleifen und Bedingungen erzeugt werden.\vglcite[1\psqq]{liang2023}
Das Modell PaLM-E demonstriert einen dritten Weg durch multimodale
Embodied Language Models, die
Sprache, Vision und kontinuierliche Sensordaten in einem gemeinsamen
Embedding-Raum verarbeiten\vglcite[2\psq]{driess2023}. Parallel entwickeln
Forscher Brain-Body-Problem Ansätze, die kognitive Architekturen mit physischen
Roboterkörpern verbinden sowie Prompt-basierte Methoden, bei denen LLMs direkt
Low-Level-Kontrollaktionen vorhersagen.\vglcites{\cite[1]{wang2024};
\cite[2]{bhat2024}} Diese Vielfalt der Ansätze zeigt, dass die Forschung noch
keine dominante Architektur etabliert hat und
verschiedene Wege zur Integration von Sprache und Robotik
exploriert.\vglcite[3\psqq]{salimpour2025}

\subsection{Herausforderungen in der Integration}

Die Verankerung abstrakter Sprachkonzepte in physischen Robotersystemen
scheitert an drei fundamentalen Problemen. Erstens müssen Roboter symbolische
Repräsentationen mit sensomotorischen Erfahrungen verknüpfen - das von Harnad
definierte Symbol-Grounding-Problem bleibt trotz jahrzehntelanger Forschung
ungelöst \vglcite[1\psqq]{cohen2024}. Zweitens fehlen standardisierte
Schnittstellen zwischen hochabstrakten Sprachbefehlen und niedere, hardwarenahe
Motorkommandos, wodurch jede Roboterplattform individuelle
Übersetzungsmechanismen benötigt. Drittens limitieren Echtzeitanforderungen die
Komplexität der Verarbeitung, da Roboter innerhalb von Millisekunden auf
Umweltveränderungen reagieren müssen. Moderne Systeme versuchen diese
Herausforderungen durch die Integration dreier Wahrnehmungsebenen zu lösen:
Interozeption erfasst interne Zustände, Propriozeption überwacht
Gelenkstellungen und Bewegungen, während Exterozeption die Umgebung durch
Kameras und Sensoren interpretiert \vglcite[3,13]{valenzo2022}. Jedoch führt
diese Komplexität zu erhöhtem Rechenaufwand und erschwert die Fehlerdiagnose bei
unerwarteten Verhaltensweisen. Folglich benötigen robotische Systeme neue
Architekturen, die effizient zwischen abstrakten Sprachmodellen und konkreten
Aktionsräumen vermitteln. In der Literatur werden deshalb hybride Architekturen
diskutiert, die symbolische Planung, differenzierbare Wahrnehmung und reaktive,
zeitkritische Kontrolle kombinieren. Für industrielle Szenarien bleibt die Frage
zentral, wie viel Autonomie ein LLM-gestütztes System erhalten kann, ohne
Validierbarkeit und Betriebssicherheit zu kompromittieren.

\subsection{Bestehende Frameworks und Tools}

Aktuelle Frameworks standardisieren die Integration von LLMs in robotische
Systeme durch modulare Architekturen. ROS-LLM verbindet das Robot Operating
System mit verschiedenen Sprachmodellen und transformiert natürlichsprachliche
Befehle automatisch in ausführbare Aktionssequenzen \vglcite[1\psq]{mower2024}.
Das Framework implementiert drei Ausführungsmodi: sequenzielle Abarbeitung für
einfache Aufgaben, Verhaltensbäume für reaktive Systeme und Zustandsautomaten
für komplexe Ablaufsteuerungen. Entwickler konfigurieren atomare Aktionen wie
das Greifen eines Objektes oder bestimmte Pfadnavigationen, die das LLM dann zu
komplexen Verhaltensketten kombiniert. Simulationsumgebungen beschleunigen
parallel die Entwicklung durch massiv-parallele GPU-Berechnungen – mit Isaac Sim
von NVIDIA lassen sich beispielsweise Tausende Roboterinstanzen gleichzeitig
simulieren.\vglcite{NVIDIATechBlog2018} Solche Plattformen
ermöglichen einen theoretischen
Zero-Shot-Transfer von der Simulation zur Realität durch systematische
simulierte Domänenrandomisierung von Physikparametern, Sensorrauschen und
Umgebungsvariationen.\vglcite[1]{fickinger2025} Die Standardisierung solcher
Werkzeugketten soll Entwicklungszeiten in Zukunft reduzieren und die
Programmierung von
Robotern niederschwelliger machen.

\section{Zwischenfazit und Forschungsfrage}

Die bisherigen Grundlagen zeigen: Simulation und
Offline-Programmierung sind zentrale Bausteine, um Robotikprozesse
vorab sicher und nachvollziehbar zu prüfen. LLMs eröffnen dabei die
Möglichkeit, Roboterprogrammcode aus natürlicher Sprache zu generieren,
stoßen in industriellen Kontexten jedoch auf die Anforderung,
Ergebnisse systematisch und reproduzierbar zu validieren.

Daraus leitet sich die zentrale Frage, wie gut sich
von LLMs erzeugter Robotercode in einer simulationsbasierten
Umgebung hinsichtlich prozessrelevanter Aspekte
prüfen und bewerten lässt.

Zur Beantwortung wird in den folgenden Kapiteln eine geeignete
Umgebung zur Simulation eines Roboters aufgebaut und an einem
kompakten, realitätsnahen
Szenario erprobt. Ziel ist das Ausführen von Robotercode innerhalb dieser
Umgebung und belastbare Prüfung nach definierten möglichen Fehler im
Programmablauf mit nachvollziehbarer Dokumentation der Ergebnisse.
Dazu beschreibt Kapitel~\ref{sec:framework} dazu die Architektur der Frameworks,
die verschiedenen zu prüfenden Fehlerarten sowie das
Testszenario. Weiterführend wird durch Anwendung des Frameworks in
Kapitel~\ref{cap:Ergebnisse} ausgewertet,
inwiefern Fehler im Roboterprogrammcode
erkannt werden sowie welche
Struktur und Inhalt dessen Dokumentation folgen.

\chapter{Methodik und Implementierung}
\label{cap:Framework}

% Was sind die Anforderungen an meine Anwendung?
% Wie wird diese Implementiert?
% Welche Muss und Soll Anforderungen lassen sich definieren?

\section{Systemarchitektur und Design}

\subsection{Architekturübersicht}
\label{sec:architektur_uebersicht}

Die Systemarchitektur des entwickelten Roboter-Kommunikationsframeworks folgt
einer mehrschichtigen Architektur (Layered Architecture), die eine klare
Trennung von Verantwortlichkeiten und eine hohe Modularität gewährleistet
\cite{Bass2012, Fowler2002}. Diese architektonische Entscheidung basiert auf
etablierten Softwarearchitekturprinzipien und ermöglicht die Realisierung eines
erweiterbaren, wartbaren und testbaren Systems für die Integration heterogener
Robotersysteme.

\subsection{Unity3D als Simulationsplattform}
Die Wahl von Unity3D als zugrundeliegende Simulationsplattform basiert auf
mehreren technischen und praktischen Erwägungen. Während es bereits mehrere
kommerzielle Programme für die Gestaltung und Simulation von Robotern in
virtuellen Umgebungen gibt - wie Robcad, Robotstudio, Igrip, Workcell und
Gazebo - sind nicht alle Programme mit anderen CAD-Systemen kompatibel,
unterstützen nicht alle Roboterbibliotheken oder andere Elemente, und einige
werden nicht unter Windows vertrieben (vgl.\cite{andaluz:16}). Unity3D hingegen
ist mit den meisten CAD-Systemen kompatibel und bietet eine
plattformübergreifende Lösung. Unity3D bietet eine ausgereifte
3D-Rendering-Pipeline mit integrierter Physik-Engine (PhysX), welche zur
Simulation von Gegenständen mit realitätsnahem Verhalten sowie komplexen
Arbeitsräumen geeignet ist. Die Engine wurde bereits erfolgreich in der
wissenschaftlichen Forschung eingesetzt und bietet Module und Plugins für
spezifische Anwendungsfälle im Simulationsbereich. Unity3D ermöglicht es auch
Nicht-Programmierern, leistungsstarke Animations- und Interaktionsdesign-Tools
zu nutzen, um Roboter visuell zu programmieren und zu animieren (vgl.
\cite{bartneck:15}).\newline Technisch ermöglicht Unity3D durch seine
Scripting-Runtime (basierend auf Mono/.NET) die Verwendung moderner
C\#-Sprachfeatures für nebenläufige Prozesse und asynchroner Programmierung,
was es ermöglicht, Prozesse in Nahe-Echtzeit darzustellen und zu überwachen.
Ein entscheidender Vorteil für die Robotik-Simulation liegt in der
Verfügbarkeit visueller Programmiertools. Die Plattform bietet umfangreiche
Debugging- und Profiling-Werkzeuge, die während der Entwicklung und zur
Laufzeit genutzt werden können. Darüber hinaus lassen sich während der Laufzeit
die Szene (hier: der Arbeitsraum) bearbeiten und aktuelle Parameter einsehen,
was besonders für die iterative Entwicklung und das Testen ist.

\subsubsection{Abhängigkeitsverwaltung und Modularität}

\subsection{Design Patterns und Prinzipien}

\subsection{Modularitäts- und Erweiterbarkeitskonzept}

Die Architektur des entwickelten Systems folgt einem konsequenten
Modularitätskonzept, das auf drei zentralen Säulen basiert: einer
Plugin-Architektur für die Integration neuer Robotertypen, einer durchgängigen
Interface-basierten Abstraktion sowie einer hierarchischen Namespace-Struktur.
Diese Designentscheidungen ermöglichen die Erweiterung des Frameworks ohne
Modifikation bestehender Komponenten und erfüllen damit das Open-Closed-Prinzip
\cite{martin2003agile}.

\subsubsection{Plugin-Architektur für Robotertypen}

Das Framework implementiert eine Plugin-basierte Architektur, die es
ermöglicht, neue Robotertypen ohne Änderungen am Core-System zu integrieren.
Jeder Robotertyp wird in einem eigenen Namespace-Verzeichnis gekapselt (z.B.
\texttt{RobotSystem/ABB/} für ABB-Roboter), wobei die Integration
ausschließlich über standardisierte Interfaces erfolgt. Diese Architektur folgt
dem Konzept der \emph{Dependency Inversion} \cite{martin2000design}, bei der
High-Level-Module (wie der \texttt{RobotManager}) nicht von Low-Level-Modulen
(spezifische Roboter-Implementierungen) abhängen, sondern beide von
Abstraktionen.

Die Vorteile dieser Architektur zeigen sich besonders bei der Integration neuer
Roboterhersteller. So könnte beispielsweise ein KUKA-Roboter durch einfaches
Hinzufügen eines \texttt{RobotSystem/KUKA/} Verzeichnisses mit entsprechenden
Interface-Implementierungen integriert werden, ohne dass bestehende
ABB-Implementierungen oder Core-Komponenten modifiziert werden müssten. Dies
reduziert das Risiko von Regressionsfehlern und ermöglicht parallele
Entwicklung verschiedener Roboter-Integrationen \cite{gamma1995design}.

\subsubsection{Interface-basierte Abstraktion}

Die Abstraktion erfolgt über vier zentrale Interface-Definitionen, die als
Kontrakte zwischen den Systemkomponenten fungieren:

\begin{itemize}
    \item \textbf{IRobotConnector}: Definiert die Schnittstelle für Roboterverbindungen, unabhängig vom Kommunikationsprotokoll
    \item \textbf{IRobotSafetyMonitor}: Standardisiert die Integration von Sicherheitsmonitoren
    \item \textbf{IRobotDataParser}: Ermöglicht austauschbare Parser für verschiedene Datenformate
    \item \textbf{IRobotVisualization}: Abstrahiert Visualisierungssysteme
\end{itemize}

Diese Interface-Segregation \cite{martin2003agile} stellt sicher, dass
Komponenten nur von den tatsächlich benötigten Abstraktionen abhängen. Der
\texttt{RobotManager} beispielsweise arbeitet ausschließlich mit der
\texttt{IRobotConnector}-Schnittstelle und ist damit vollständig entkoppelt von
spezifischen Implementierungsdetails wie WebSocket-Protokollen oder
HTTP-Polling-Mechanismen.

Die Verwendung des Strategy-Patterns \cite{gamma1995design} für die
Safety-Monitore ermöglicht es, verschiedene Überwachungsalgorithmen zur
Laufzeit auszutauschen. So implementieren sowohl
\texttt{CollisionDetectionMonitor} als auch
\texttt{SingularityDetectionMonitor} das
\texttt{IRobotSafetyMonitor}-Interface, können aber völlig unterschiedliche
Detektionsstrategien verwenden. Diese Flexibilität ist essentiell für die
Anpassung an verschiedene Sicherheitsanforderungen und Industrienormen.

\subsubsection{Namespace-Struktur und Paketierung}

Die hierarchische Namespace-Struktur folgt dem Prinzip der \emph{Package by
    Feature} \cite{uncle2012clean}, wobei funktional zusammengehörige Komponenten
in gemeinsamen Namespaces organisiert sind:

\dirtree{%
    .1 RobotSystem/.
    .2 Core/.
    .2 Interfaces/.
    .2 ABB/.
    .3 RWS/.
    .2 Safety/.
}

Diese Struktur bietet mehrere Vorteile für die Wartbarkeit und Erweiterbarkeit:

\begin{enumerate}
    \item \textbf{Klare Verantwortlichkeiten}: Jeder Namespace hat eine eindeutig definierte Zuständigkeit
    \item \textbf{Minimale Kopplung}: Abhängigkeiten verlaufen nur von spezifischen zu allgemeinen Namespaces
    \item \textbf{Einfache Navigation}: Die Struktur spiegelt die konzeptuelle Architektur wider
    \item \textbf{Versionierbarkeit}: Herstellerspezifische Implementierungen können unabhängig versioniert werden
\end{enumerate}

Die Verwendung von Unity-spezifischen Meta-Dateien (\texttt{.meta}) in
Kombination mit der Namespace-Struktur ermöglicht zudem die nahtlose
Integration in die Unity-Engine, wobei die logische Strukturierung auch auf
Dateisystemebene erhalten bleibt. Dies erleichtert die Zusammenarbeit in Teams
und die Versionskontrolle mit Git \cite{chacon2014pro}.

Das Modularitätskonzept zeigt sich auch in der Möglichkeit, einzelne Module als
Unity-Packages zu exportieren und in anderen Projekten wiederzuverwenden. Die
strikte Einhaltung der Interface-Kontrakte garantiert dabei die Kompatibilität
zwischen verschiedenen Versionen und Konfigurationen des Systems.

Ein zentrales Architekturprinzip des Frameworks ist die konsequente Anwendung
der Dependency Inversion \cite{Martin1996}. Konkrete Implementierungen hängen
von abstrakten Interfaces ab, nicht von anderen konkreten Klassen. Dies wird
durch die Definition der Kerninterfaces \texttt{IRobotConnector},
\texttt{IRobotSafetyMonitor} und \texttt{IRobotVisualization} erreicht, die als
Kontraktdefinitionen zwischen den Schichten fungieren.

Die Interfaces definieren dabei nicht nur die Methodensignaturen, sondern
etablieren auch semantische Kontrakte im Sinne des Design by Contract
\cite{Meyer1992}. Beispielsweise garantiert das
\texttt{IRobotConnector}-Interface, dass Zustandsänderungen über Events
propagiert werden und dass die Verbindung idempotent hergestellt und getrennt
werden kann.

Die lose Kopplung ermöglicht es, verschiedene Robotertypen durch Implementation
des \texttt{IRobotConnector}-Interfaces zu unterstützen, ohne Änderungen am
Kern des Frameworks vornehmen zu müssen. Dies demonstriert die Erweiterbarkeit
der Architektur und validiert die Designentscheidung für eine
interface-basierte Abstraktion \cite{Liskov1987}.

% Definition der Schnittstellen (ABB Rapid, virtuelle Steuerung)...
% Modulare Struktur (Datenakquise, Analyse-Engine, Output-Generator)...

\section{Entwicklungsmethodik und Implementierung}
\label{sec:Implementierung_Framework}
% Beschreibung des Vorgehens...
% Implementierung der Analysefunktionen in priorisierter Reihenfolge...
\subsection{Prozessfolgen}
\label{ssec:Prozessfolgen}
% Überprüfung der korrekten Abfolge von Aktionen...
User Input: Simulationsumgebung und Verbindung zu Robot Studio

Wie kann ich Prozessfolgen überprüfen? Prozessfolge: Folge and Arbeitsschritten
eines Arbeitsprozesses, hier Roboter -> Was wird in welcher Reihenfolge wohin
bewegt? Wie kann ich das Messen? - Bewegt sich das Werkstück von Position Start
zu Position Ziel? - Bewegen sich die Werkstücke in der richtigen Reihenfolge
von Start zu Ziel? Benötigt: Definition von Start \& Zielpositionen einzelner
Werkstücke

\subsection{Prozesszeiten}
\label{ssec:Prozesszeiten}
% Messung und Bewertung von Zyklus- und Wartezeiten...
\subsection{Kollisionserkennung}
\label{ssec:Kollisionserkennung}
% Detektion von Kollisionen...
\subsection{Singularitäten und Gelenkgrenzen}
\label{ssec:Singularitaeten}

Die Erkennung und Vermeidung von Singularitäten stellt einen kritischen Aspekt
bei der Robotersteuerung dar, da diese zu einem Verlust der Kontrollierbarkeit
und potentiell gefährlichen Situationen führen können. Das entwickelte
Framework implementiert eine geometrisch fundierte Methode zur
Echtzeitdetektion von Singularitäten basierend auf der Kollinearitätsanalyse
der Gelenkachsen.

\subsubsection{Theoretische Grundlagen der Singularitätsdetektion}
\label{sssec:Theorie_Singularitaeten}
Was sind Singularitäten, wo treten sie auf?
Ab wann treten Singularitaeten auf, gibt es Schwellwerte, was ist die manipulability?
Was sind die Folgen von Singularitaeten?

Eine kinematische Singularität tritt auf, wenn die Jacobi-Matrix des Roboters
ihren vollen Rang verliert, mathematisch ausgedrückt durch:
\begin{equation}
    \text{rank}(\mathbf{J}(\boldsymbol{\theta})) < \min(m, n)
    \label{eq:singularity_condition}
\end{equation}
wobei $\mathbf{J}(\boldsymbol{\theta}) \in \mathbb{R}^{m \times n}$ die Jacobi-Matrix, $\boldsymbol{\theta}$ der Gelenkwinkelvektor, $m$ die Anzahl der Freiheitsgrade im kartesischen Raum und $n$ die Anzahl der Robotergelenke darstellt.

Für serielle Robotermanipulatoren mit sechs Freiheitsgraden (wie ABB
IRB-Roboter) können drei primäre Singularitätstypen unterschieden werden:

\paragraph{Boundary Singularities (Randsingularitäten):}
Treten auf, wenn der Roboter die Grenzen seines Arbeitsraums erreicht,
typischerweise bei vollständig ausgestreckter Konfiguration.

\paragraph{Wrist Singularities (Handgelenksingularitäten):}
Entstehen, wenn die Rotationsachsen der letzten drei Gelenke (Gelenke 4, 5, 6)
kollinear werden. Mathematisch beschrieben durch:
\begin{equation}
    \mathbf{z}_4 \parallel \mathbf{z}_6 \text{ oder } |\mathbf{z}_4 \cdot \mathbf{z}_6| \approx 1
    \label{eq:wrist_singularity}
\end{equation}
wobei $\mathbf{z}_i$ die Rotationsachse (Z-Achse) des $i$-ten Gelenks im Weltkoordinatensystem darstellt.

\paragraph{Elbow Singularities (Ellbogensingularitäten):}
\subsubsection{Implementierung der achsenbasierten Singularitätserkennung}
\label{sssec:Implementierung_Singularitaeten}

Das Framework nutzt die räumlichen Transformationen des Flange-Systems, um die
aktuellen Gelenkachsenorientierungen zu bestimmen. Der implementierte
Algorithmus basiert auf der geometrischen Analyse der Achsenkollinearität:

\subsubsection{Achsenbasierte Singularitätsdetektion}
\label{alg:singularity_detection}
Die Schwellwerte sind konfigurierbar definiert als:
\begin{align}
    \tau_{\text{wrist}}    & = 0.98 \quad (\cos(11.5^\circ)) \\
    \tau_{\text{shoulder}} & = 0.93 \quad (\cos(22^\circ))   \\
    \tau_{\text{general}}  & = 0.95 \quad (\cos(18^\circ))
\end{align}

\subsubsection{Praktische Implementierung im Unity-Framework}
\label{sssec:Unity_Implementierung}

Die Singularitätsdetektion ist in der Klasse \texttt{RobotSafetyMonitor}
implementiert und nutzt die Flange-Bibliothek zur Transformation der
Gelenkachsen. Der zentrale Algorithmus wird in der Methode
\texttt{DetectSingularityByAxes()} realisiert:

\subsubsection{Anwendungsbeispiel: ABB IRB120 Handgelenksingularität}
\label{sssec:Beispiel_IRB120}

Zur Veranschaulichung der Funktionsweise wird eine typische
Handgelenksingularität des ABB IRB120 betrachtet:

\textbf{Ausgangssituation:} Der Roboter befindet sich in einer Konfiguration, bei der Gelenk 5 ($\theta_5$) nahe null Grad steht. In dieser Position werden die Rotationsachsen der Gelenke 4 und 6 nahezu kollinear.

\textbf{Geometrische Analyse:}
\begin{itemize}
    \item Gelenk 4 Achse: $\mathbf{z}_4 = [0.866, 0.5, 0]^T$
    \item Gelenk 6 Achse: $\mathbf{z}_6 = [0.848, 0.530, 0]^T$
    \item Skalarprodukt: $|\mathbf{z}_4 \cdot \mathbf{z}_6| = 0.999$
\end{itemize}

\textbf{Detektionsergebnis:}
Da $0.999 > \tau_{\text{wrist}} = 0.98$ erfüllt ist, wird eine Handgelenksingularität detektiert. Die berechnete Manipulierbarkeit ergibt sich zu:
\begin{equation}
    \mu = 1 - \frac{0.999 - 0.98}{1 - 0.98} \cdot 0.95 = 0.095
\end{equation}

Das System protokolliert: \textit{"Wrist Singularity (J4-J6 aligned)"} mit
einer Manipulierbarkeit von 0.095, was unterhalb des kritischen Schwellwerts
von 0.1 liegt.

\textbf{Vorteil gegenüber heuristischen Methoden:}
Im Gegensatz zu vereinfachten trigonometrischen Approximationen nutzt die achsenbasierte Methode die tatsächlichen räumlichen Transformationen der Robotergelenke. Dies ermöglicht eine präzise, konfigurationsunabhängige Detektion, die auch bei komplexeren Robotergeometrien und verschiedenen Herstellern funktioniert.

Die implementierte Lösung bietet folgende Eigenschaften:
\begin{itemize}
    \item \textbf{Echtzeitfähigkeit:} Berechnung erfolgt mit 5 Hz Aktualisierungsrate
    \item \textbf{Typenspezifisch:} Unterscheidung verschiedener Singularitätsarten
    \item \textbf{Konfigurierbar:} Anpassbare Schwellwerte je Singularitätstyp
    \item \textbf{Herstellerunabhängig:} Funktioniert mit allen Flange-unterstützten Robotern
\end{itemize}

\section{Testszenario}
\subsection{Versuchsaufbau}

\chapter{Validierungsergebnisse der Testszenarien}
\label{cap:Ergebnisse}
Die Architektur des Frameworks besteht somit aus den genannten Modulen, die in
Kapitel~\ref{sec:framework} im Detail beschrieben werden. Ihre
Funktionsweise wird in
Kapitel~\ref{cap:Ergebnisse} anhand von Testszenarien überprüft und ausgewertet.
\section{Überblick und Zielsetzung}

In diesem Kapitel werden die Ergebnisse der im vorherigen Kapitel beschriebenen
Implementierung vorgestellt. Im Fokus steht die Überprüfung der vier
entwickelten Safetymodule – Prozessfolgenüberwachung, Kollisionserkennung,
Achsgeschwindigkeits- und Beschleunigungsüberwachung sowie
Singularitätserkennung – innerhalb der aufgebauten Simulationsumgebung. Ziel ist
es zu überprüfen, inwiefern im gewählten Testsetup eine Detektion von Fehlern in
der Roboterbewegung und Interaktion, welche mit den vier untersuchten Parametern
zusammenhängen, auftreten. Ziel ist es zu überprüfen, inwiefern im gewählten
Testsetup eine Erkennung von Fehlern in der Roboterbewegung und Interaktion,
welche mit den vier untersuchten Parametern zusammenhängen, auftreten.

Für jedes Modul wurden gezielte Testfälle definiert, die sowohl korrekte als
auch fehlerhafte Szenarien abbilden, um die Funktionsweise und Zuverlässigkeit
der Module zu überprüfen. Die Ergebnisse werden anhand von Beobachtungen aus der
Simulation, gespeicherten Zustands- und Ereignisdaten sowie grafischen
Darstellungen aufgezeigt. Die Testfälle wurden dabei als Pfade in RobotStudio
definiert. Diese Pfade werden von RobotStudio in RAPID-Code umgewandelt und
gespeichert. Durch die Synchronisation mit dem Controller und dem Setzen des
aktuellen Programms als Standardprogramm lässt sich das Programm in RobotStudio
simulieren. So wurde für jedes Szenario vorgegangen.

Zur zusätzlichen Validierung wurde ein Experteninterview durchgeführt, in dem
die Testcases vorgestellt und die Funktionsweise des Frameworks diskutiert
wurden. Die Aussagen des Experten werden an geeigneter Stelle in diesem Kapitel
dargestellt und in Kapitel \ref{cap:diskussion} kritisch eingeordnet.

\section{Auswertung der Prozessflussüberwachung}
\label{sec:processauswertung}

Die Validierung der Prozessflussüberwachung erfolgt über die Ausführung eines
korrekten Szenarios als auch eines fehlerhaften Szenarios. Der Fehler soll
provoziert werden, in dem das Werkstück abweichend zum in Abbildung
\ref{figure:Prozessfluss} gezeigten gewünschten Ablauf direkt in das zweite
Regal bewegt wird. Semantisch bedeutet dies ein Überspringen der Station
\textit{Machine}. Das soll ein SafetyEvent triggern, sobald das Werkstück im
falschen Regal abgelegt wird.

\textbf{Korrekter Prozessfluss}
\begin{enumerate}
  \item Bewegung von Home-Position zu linkem Regal
  \item Greifen des Werkstücks
  \item Bewegung zu Maschine
  \item Platzieren des Werkstücks in Maschine
  \item Warten auf Beendigung des Bearbeitungsprozesses in
    Warteposition (Bearbeitung hier nur simuliert)
  \item Bewegen zum Werkstück und Greifen aus Maschine
  \item Bewegung zu rechtem Regal
  \item Platzieren des Objekts in rechtem Regal
  \item Rückkehr zur Home-Position
\end{enumerate}

\begin{figure}[H]
  \centering
  \includegraphics[width=0.7\linewidth]{Figures/Prozessfolge.png}
  \caption{Visuelle Darstellung des Prozessflusses durch Konfiguration der Parts
  und Stations}
  \label{figure:Prozessfluss}
\end{figure}

Im veränderten Prozessfluss wird nun der Schritt der Bearbeitung übersprungen.
Dies kann in der Praxis durch eine inkorrekte Abfolge im Roboterprogrammcode
passieren. Somit ist der veränderte Prozessfluss wie folgt definiert.

 
\textbf{Abgewandelter Prozessfluss}
\begin{enumerate}
  \item Bewegung von Home-Position zu linkem Regal
  \item Greifen des Werkstücks
  \item Bewegung zu Maschine
  \item Bewegung zu rechtem Regal
  \item Platzieren des Objekts in rechtem Regal
  \item Rückkehr zur Home-Position
\end{enumerate}

\subsection{Simulationsergebnis}
Wird der korrekte Prozessfluss durchlaufen, werden in Bezug auf den
ProcessFlowMonitor keine Ereignisse protokolliert.
Bei der Ausführung des Robotercodes mit verändertem Prozessfluss findet sich in
der nach Beendigung des Programms ein JSON-Log, benannt nach
Zeitstempel und Name des Moduls,
welcher einen durch das aufgetretene SafetyEvent des Process Flow Monitors
erstellten Eintrag enthält (vgl. Abbildung~\ref{listing:processflowerror}.)

\begin{figure}[H]
  \inputminted[fontsize=\footnotesize]{json}{code-snippets/processflowerror.json}
  \caption{JSON-Log zum Prozessfolgenfehler, Achswinkel wurden im Nachhinein
  entfernt.}
  \label{listing:processflowerror}
\end{figure}

In Abbildung~\ref{listing:processflowerror} ist abzulesen, dass der
Fehler hier korrekt erkannt und klassifiziert wurde. Dabei zeigt das Feld
\texttt{description} den genauen Hergang des Events an. Hier wurde eine invalide
Transition von StorageIn zu StoraeOut versucht. Das Framework erkennt
ebenfalls, welcher Prozessschritt der Richtige gewesen wäre. Außerdem wird hier
als Violation Type der Type \texttt{SkippedStation} angegeben. Dieser ist als
übersprungene Station zu klassifizieren, was in diesem Fall korrekt ist.
Zusätzlich werden weitere Parameter der Simulation und des Controllers
weitergegeben, unter anderem welches Modul und welche Routine des Moduls zum
Zeitpunkt des Auftretens ausgeführt wurde sowie welcher Programmzeile der
ProgramPointer sich zum Zeitpunkt der Event-Auslösung befand. Durch den Wert des
Keys \texttt{totalSafetyEvents} ist zu erkennen, wie viele Ereignisse bei der
Ausführung vom Programm getriggert wurden. Hier is es lediglich der oben
Genannte.

\section{Auswertung der Kollisionserkennung}
\label{sec:collisionauswertung}

Zur Auswertung der Kollisionserkennung
wurde der oben bereits genannte Prozess genutzt. Anschließend wurde
der Pfad, auf dem sich der Roboter zwischen seiner Home-Position und dem linken
Regal, in dem er ein Werkstück greifen soll, verändert. Rechts dargestellt in
Abbildung \ref{figure:kollision}, führt der Pfad im Gegensatz zum
kollisionsfreien, näher am Roboter verlaufenden Pfad aufgrund des ausladenden
Umschwungs durch die Säulengeometrie. Durch das Abfahren des Pfades gegen den
Uhrzeigersinn wird provoziert, dass der Roboter sich durch die in Abbildung
\ref{figure:kollision} links blau eingefärbte Säule hindurch bewegen
muss. Die Säule ist der Tag \texttt{Obstacle} zugewiesen – sie wird
durch einen vorhandenen Collider,
welcher den Dimensionen der Säule entspricht, zum Kollisionshindernis.

\begin{figure}[!htb]
  \centering
  \begin{minipage}{.535\textwidth}
    \centering
    \includegraphics[width=0.9\linewidth]{Figures/CollisionUnity.png}
  \end{minipage}%
  \begin{minipage}{0.465\textwidth}
    \centering
    \includegraphics[width=0.9\linewidth]{Figures/CollisionPathRobotStudio.png}
  \end{minipage}
  \caption{Kollision in Unity3D (links) und zugehörige Position auf Pfad in
    RobotStudio (rechts). Gelenk 4, 5 und 6 sowie Greifer befinden
    sich innerhalb der
  Säulengeometrie.}
  \label{figure:kollision}
\end{figure}

\subsection{Simulationsergebnis}

In Abbildung \ref{listing:collisiondetectionerror} ist der Output nach Kollision
mit der Säule dargestellt. Dem Event wurden zusätzlich Eventdaten angefügt,
welche hier den Punkt der Kollision im Unity Koordinatensystem als auch die
Entfernung zum Zentrum des kollidierenden Objekts darstellen. Die Kollision wird
somit zuverlässig erkannt. Wichtig ist dabei zu erwähnen, dass die
Kollisionserkennung stark von den verwendeten Collidern abhängt. Hier verwendet
das Framework Mesh-Collider zur genauen Abbildung der Robotergeometrie, das
Kollisionobjekt wird durch einen primitiven zylindrischen Collider definiert.
Beim Outputformat der eventDataJson handelt es sich um string-escaped JSON. Die
Daten sind also als Event-Daten in einem String komprimiert.\\

\begin{figure}[H]
  \inputminted[fontsize=\footnotesize]{json}{code-snippets/collisiondetection.json}
  \caption{JSON-Log zur Kollisionserkennung. Sich wiederholende
    Key-Value Paare wurden
  verkürzt}
  \label{listing:collisiondetectionerror}
\end{figure}

 
Weiterführend ist zu erkennen, dass die Kollison mit verschiedenen Gliedern des
Roboters sequentiell erkannt wird. Sobald der Roboter sich visuell weiter in die
Säule bewegt, wird jeweils bei der Kollison mit einem weiteren Glied eine
weitere Kollision erkannt und ein eigenes Event getriggert. Die Reihenfolge der
Joints in diesem Szenario ist nach eingehender Überprüfung als korrekt zu
bewerten, da die Robotergeometrie dafür sorgt, dass Gelenk 4 deutlich breiter
ist als 5 und 6. Gelenk 5 und 6 sind in die Geometrie von Gelenk 4 eingefasst,
daher kollidert der Roboter inital mit Gelenk 4, bevor eine Kollision an den
kinematisch dahinterliegenden Gelenken erkannt wird.\\

  Weiterführend lässt sich beim Greifen des Werkstücks feststellen, dass
hier ebenfalls eine Kollison erkannt wird: Durch das Greifen des Werkstücks wird
eine falsch-positive Kollison getriggert, da bevor das Werkstück gegriffen wird
und semantisch in die kinematische Kette des Roboters verschoben wird, für einen
kurzen Zeitpunk eine Kollision stattfindet. Ein Beispiel dazu findet sich im
letzten Block von Abbildung \ref{listing:collisiondetectionerror}. Gleiches
lässt sich beim Ablegen desn Werkstücks beobachten. Wichtig zu erwähnen ist der
unterschiedliche EventType, da das Werkstück aufgrund von fehlendem Tag nicht
als kritisch eingestuft wird.

\newpage
\section{Auswertung der Singularitätserkennung}
\label{sec:singularityauswertung}

Zur Untersuchung der Singularitätserkennung wurde in RobotStudio ein
Szenario erstellt,
das gezielt eine Handgelenks-Singularität provoziert. Dabei wurde der
Roboter in eine Pose
geführt, in der die Achsen~4 und~6 nahezu kollinear verlaufen und
somit die Bedingung
$\theta_{5} \approx 0^\circ$ erfüllt ist. Die Pose ist in Abbildung
\ref{fig:wristSingularity} zu erkennen, hier befindet sich der Roboter nach dem
Greifen des Werkstücks auf dem Weg zum rechten Regal, muss sich für das
Platzieren des Werkstücks im Regal jedoch umorientieren. So kommt die
Singularität
zustande.

\begin{figure}[H]
  \centering
  \includegraphics[width=0.7\textwidth]{figures/wristSingularity.png}
  \caption{Pose in Unity, bei der eine Handgelenks-Singularität mit
  ($\theta_{5} \approx 0^\circ$) entsteht}
  \label{fig:wristSingularity}
\end{figure}

\subsection{Simulationsergebnis}

Die vom Monitor in Unity aufgezeichneten Safety Events sind in
Abbildung~\ref{lst:singularity_json} als gekürzter Auszug
dargestellt. Es werden sowohl
das \enquote{Entering}- als auch das \enquote{Exiting}-Ereignis
erfasst, jeweils mit den
zugehörigen Gelenkwinkeln und einem berechneten
Manipulierbarkeitswert. Nicht relevante
Felder des Snapshots wurden entfernt, da die Gelenkwinkel bereits im
\texttt{eventDataJson} enthalten sind.

\begin{figure}[H]
  \inputminted[fontsize=\footnotesize,breaklines]{json}{code-snippets/singularityerror.json}
  \caption{Gekürzter Auszug der in Unity aufgezeichneten Safety
    Events zur Wrist-Singularität. Zusätzliche Informationen wurden mit "[...]"
  abgekürzt.}
  \label{lst:singularity_json}
\end{figure}

Die Analyse der Ereignisse zeigt, dass der Monitor den Eintritt in
die Wrist-Singularität
bei einer Gelenkkonfiguration von etwa
$[-82.7, -6.9, 36.6, -112.8, -4.9, -153.5]^\circ$ registrierte. Der berechnete
Manipulierbarkeitswert lag hier bei $w \approx 0.19$. Beim Verlassen der Pose
($[-91.8, -8.3, 37.8, -91.3, 5.2, -182.6]^\circ$) wurde das
\enquote{Exiting}-Ereignis
ausgegeben, wobei der Manipulierbarkeitswert mit $w \approx 0.20$
ebenfalls sehr niedrig blieb.
Die Ereignisse decken sich mit dem in RobotStudio provozierten
Szenario und markieren den
Übergang in und aus einer singulären Konfiguration.
Zur Detektion von Singularitäten anderen Typs (hier Schulter- und
Ellbogensingularitäten) wird das gleiche Verfahren angewendet. Hier decken sich
die Ergebnisse mit den oben Beschriebenen.

\section{Auswertung der Gelenkdynamiküberwachung}
\label{sec:Analyse_Sicherheit}
Der \textit{Joint Dynamics Monitor} erfasst kontinuierlich die Dynamik der sechs
Roboterachsen. Die Implementierung in Unity3D basiert auf den vom digitalen
Abbild gestreamten Gelenkwinkeln, welche aus denen Geschwindigkeiten und
Beschleunigungen differenziell berechnet werden. Zur Signalanalyse werden
mehrere Mechanismen kombiniert, darunter exponentielle Glättung sowie
Fenster-basiertes Mittel zur Dämpfung von Ausreißern. Zusätzlich wird ein
Sicherheitsfaktor von 0{,}8 auf die in RobotStudio spezifizierten Maximalwerte
angewendet, sodass die Schwellwerte im Monitor niedriger liegen als die realen
physikalischen Limits (vgl. Implementierung in
\texttt{JointDynamicsMonitor.cs}). Hier wird mittels des Befehl \texttt{v7000}
im RAPID-Code in RobotStudio die Geschwindigkeit auf den programmierbaren
Maximalwert gesetzt. Dies wurde an zwei Stellen der Roboterbewegung umgesetzt:
Bei der Bewegung von der Home-Position zum linken Regal und bei der Bewegung
nach dem Greifen des Werkstücks zur Maschine. Als primäre Drehachse ist hier
eine Überhöhung der Geschwindigkeit von Gelenk 1 (zentrale horizontale Drehachse
des Roboters) zu erwarten.

\subsection{Simulationsergebnis}
Während der Testsimulation wurden durch den Monitor zwei Überschreitungen der
Geschwindigkeitsgrenzen auf \textbf{Joint 1} registriert. Diese Ereignisse sind
in der aus Unity aufgezeichneten Logdatei dokumentiert. Ein Auszug ist in
Abbildung~\ref{lst:jointdynamics_json} dargestellt. Dort wird jeweils die
Überschreitung der Geschwindigkeit und das nachfolgende
\enquote{resolved}-Ereignis vermerkt, zusammen mit dem aktuellen
Gelenkwinkelzustand des Roboters.

\begin{figure}[H]
  \inputminted[fontsize=\footnotesize,breaklines]{json}{code-snippets/jointdynamicserror.json}
  \caption{Gekürzter Auszug der in Unity aufgezeichneten Safety
  Events des Joint Dynamics Monitor, Ereignis widerholt sich}
  \label{lst:jointdynamics_json}
\end{figure}

Zum Vergleich wurden die Gelenkwinkel aus RobotStudio exportiert und
anhand der berechneten
Achsgeschwindigkeiten ausgewertet. Abbildung~\ref{fig:jointdynamics}
zeigt die Ergebnisse:
Im oberen Diagramm sind die Achswinkel aller sechs Gelenke
dargestellt (J1 hervorgehoben in rot),
darunter die berechneten Geschwindigkeiten mit markierten Schwellwertbereichen.
Die rot eingefärbten Abschnitte kennzeichnen Intervalle, in denen die
aus RobotStudio
exportierten Daten die definierte Grenze von $\pm 50\,^\circ$/s überschreiten.
Die blau markierten Bereiche repräsentieren die durch den Monitor in
Unity ausgegebenen
Safety Events.
Zu erkennen ist eine starke Überschneidung der Bereiche.
Weiterführend zeichnet sich vor allem in der zweiten
Geschwindigkeitsüberschreitung eine Verzögerung beim Ein- und Austritt in den
kritischen Zustand ab. Die Events in Unity werden hier zeitverzögert
weitergegeben.

\begin{figure}[H]
  \centering
  \includegraphics[width=\textwidth]{Figures/achsgeschwindigkeitPlot.png}
  \caption{Achswinkel (oben) und Achsgeschwindigkeiten (unten) mit markierten
    Bereichen (rot: Schwellenübertritte in den Rohdaten, blau: Safety
  Events aus Unity).}
  \label{fig:jointdynamics}
\end{figure}

Eine Gegenüberstellung der Zeitintervalle ist in
Tabelle~\ref{tab:jointdynamics} enthalten.
Dort werden die Start- und Endzeitpunkte der Abschnitte, die Dauer
sowie die Gelenkwinkel
und -geschwindigkeiten an diesen Punkten angegeben. Anhand dieser
Darstellung wird sichtbar,
dass die blau markierten Safety Events zeitlich nach den rot
markierten Schwellenübertritten
liegen. Die in Tabelle~\ref{tab:jointdynamics} und
Abbildung~\ref{fig:jointdynamics} dargestellten Intervalle wurden ermittelt,
indem die aus RobotStudio exportierten Gelenkwinkel mit den in den Safety Events
gespeicherten Zuständen vergleichbar sind. Dazu wurde der euklidische
Abstand zwischen
den Vektoren der Achswinkel berechnet, um den jeweils nächstliegenden
Zeitpunkt in
den Referenzdaten zu bestimmen. Auf diese Weise lassen sich die vom
Monitor in Unity3D
gemeldeten Ereignisse mit den in den Rohdaten beobachteten Schwellenübertritten
korrelieren.

\begin{table}[H]
  \centering
  \small
  \begin{tabularx}{\textwidth}{lrrrrrrX}
    \toprule
    Typ               & Start [ms] & Ende [ms] & Dauer [ms] & Start
    J1 [\textdegree] & Ende J1 [\textdegree] \\
    \midrule
    Dynamik (rot)     & 216        & 1800      & 1584       & -40.35
    & 107.62                \\
    Zielwinkel (blau) & 624        & 1848      & 1224       & -1.25
    & 109.89                \\
    Dynamik (rot)     & 7824       & 9408      & 1584       & 107.08
    & -41.88                \\
    Zielwinkel (blau) & 8208       & 9552      & 1344       & 74.15
    & -45.24                \\
    \bottomrule
  \end{tabularx}
  \caption{Zeitintervalle und Zustände der Joint-Dynamics-Auswertung}
  \label{tab:jointdynamics}
\end{table}

\section{Validierung der Ergebnisse durch Experteninterview}

Im Rahmen der Ergebnisdarstellung wurde ein Experteninterview mit Daniel Syniawa
(M.Sc.) durchgeführt. Ziel war es, die
Funktionsfähigkeit des Frameworks praxisnah zu validieren und Einschätzungen zur
industriellen Einordnung zu gewinnen. Das Interview diente ausschließlich der
\emph{deskriptiven} Ergänzung der Ergebnisse; eine vertiefte Interpretation
erfolgt in Kapitel~\ref{cap:diskussion}.

\subsection{Vorgehen und Gegenstand}

Im Interview wurden die Software und ihre Architektur vorgestellt. Anschließend
wurden die in Kapitel~4 beschriebenen Testfälle gemeinsam in RobotStudio
nachvollzogen: Der Roboter führte die Szenarien aus, während parallel beobachtet
wurde, ob und wie die Module Ereignisse auslösen und ob diese im Logging erfasst
werden. Damit wurde die grundsätzliche Funktionsweise des Frameworks
demonstriert (Auslösen von Safety Events, Erzeugung von JSON-Logs mit
Programmkontext).

\subsection{Beobachtungen}

Prinzipiell konnte ein schnelles Verständnis für die Anwendung des Frameworks
und dessen Funktionsweise gewonnen werden. Bei der Testung der Fälle wurde der
praktische Nutzen der abgespeicherten JSON-Logs hervorgehoben: Für jedes
Ereignis liegen der aktuelle Programmzeiger, das aktuell ausgeführte Programm
sowie die relevante Roboterpose bzw.\ Kontextinformationen vor. Dies erleichtert
die Fehlersuche und macht die Analyse auch für weniger erfahrene Anwender
nachvollziehbar.

\subsection{Hinweis zur Evaluationsmethodik}

Für eine belastbare Evaluation schlug Syniawa vor, die Leistung des Frameworks
\emph{quantitativ} gegen manuelle Verfahren zu vergleichen. Konkret: Wie schnell
findet eine Person den Fehler (z.\,B.\ Auftreten einer Singularität) im
RAPID-Code in RobotStudio im Vergleich zur Erkennung durch Ausführung und
Logging im Framework? Ein solcher Vergleich wäre mit erheblichem Aufwand
verbunden, würde aber die Einordnung der Wirksamkeit deutlich schärfen.

\subsection{Einordnung des Robotersimulations-Ökosystems}

Syniawa wies darauf hin, dass die aktuelle Werkzeuglandschaft stark proprietär
geprägt ist und es nur wenige plattformübergreifende Lösungen mit integrierter
Physik gibt. Häufig ist bereits die stabile Anbindung eines Roboters
herausfordernd; Simulation in RobotStudio erfordert viel Expertenwissen
und Zeit. Die Integration in RWS sei komplex und eher knapp
dokumentiert – insgesamt sei die Nutzerbasis in diesem Bereich klein. Diese
Beobachtungen unterstreichen die Relevanz eines modularen, erweiterbaren
Ansatzes wie in dieser Arbeit umgesetzt.

\subsection{Zusammenfassung}

Das Interview bestätigte die grundsätzliche
Funktionalität des Frameworks in den demonstrierten Testfällen und
identifizierte einen klaren Pfad für eine zukünftige, quantitative Evaluation.
Zudem wurde der praktische Mehrwert der strukturierten Logs (Programmzeiger,
laufendes Programm, Pose/Kontext) betont. Die Einordnung der industriellen
Robotersimulationslandschaft liefert den Rahmen, in dem die
vorgestellten Ergebnisse zu
sehen sind.

\section{Zusammenfassung der Ergebnisse}

Die Ergebnisse zeigen, dass die in Unity implementierten Monitore in allen
Testfällen die in RobotStudio provozierten Szenarien widerspiegeln konnten.
Dabei wurde deutlich, dass sich für jedes Modul charakteristische Muster im
Logging abzeichnen: Prozessabweichungen wurden sequenziell dokumentiert,
Kollisionen mit Schweregraden versehen, Singularitäten mit Gelenkwinkeln und
Manipulierbarkeitswerten erfasst, und Geschwindigkeitsverletzungen durch
Event-Paare (exceeded/resolved) gekennzeichnet.\\

\begin{table}[H]
  \centering
  \small
  \begin{tabularx}{\textwidth}{lXX}
    \toprule
    \textbf{Monitor}      & \textbf{Getestetes Szenario}
    & \textbf{Erkannte Ereignisse / Beobachtungen}
    \\
    \midrule
    Process Flow          & Ablauf mit bewusst fehlerhafter
    Reihenfolge der Operationen                                &
    Abweichung von der erwarteten Sequenz korrekt erkannt, Events
    dokumentieren Verletzung der Prozessfolge.
    \\
    \addlinespace
    Collision Detection   & Simulation mit Kollision zwischen Greifer
    und Werkstück bzw. Störkörper                    & Mehrere
    Kollisionen aufgezeichnet, inklusive beteiligter Objekte; Events
    mit Schweregrad (critical/warning) unterschieden.                 \\
    \addlinespace
    Singularity Detection & Pose in RobotStudio, die eine
    Wrist-Singularität ($\theta_{5} \approx 0^\circ$) provoziert &
    Entering- und Exiting-Events erfasst; Gelenkwinkel und
    Manipulierbarkeitswerte im JSON protokolliert; Szenario deckt
    sich mit RobotStudio. \\
    \addlinespace
    Joint Dynamics        & Bewegung mit Überschreitung der
    Geschwindigkeitsgrenzen auf J1 ($\pm 50^\circ$/s)          & Zwei
    Event-Paare (exceeded/resolved) aufgezeichnet; Verzögerung
    zwischen Schwellenübertritt (Rohdaten) und Event (Monitor)
    sichtbar.       \\
    \bottomrule
  \end{tabularx}
  \caption{Übersicht der getesteten Monitore, Szenarien und erkannten
  Ereignisse im Ergebnisteil}
  \label{tab:monitor_overview}
\end{table}

Die Übersicht in Tabelle~\ref{tab:monitor_overview} verdeutlicht die
Unterschiede
zwischen den Monitoren hinsichtlich Art der Szenarien und Form der
erfassten Events.
Auffällig ist, dass sich in einigen Fällen eine zeitliche Verzögerung zwischen
den in den Rohdaten beobachteten Zuständen und den vom Monitor generierten
Ereignissen zeigt. Diese Beobachtung ergibt sich aus den in Kapitel~3
beschriebenen
Mechanismen (z.\,B. Glättung, Abtastrate).

\chapter{Diskussion}
\label{cap:diskussion}

\section{Einleitung} In diesem Kapitel werden die in
Kapitel~\ref{cap:Ergebnisse} präsentierten
Ergebnisse kritisch reflektiert und in den fachlichen Kontext eingeordnet. Im
Zentrum stehen die Gesamtarchitektur des Frameworks und die vier implementierten
Safetymodule. Ergänzend wird ein Experteninterview mit Daniel Syniawa
herangezogen, in dem die Software, die Architektur sowie konkrete Testcases in
RobotStudio gemeinsam betrachtet wurden. Ziel ist es, die Stärken und Grenzen
der Arbeit transparent zu machen und konkrete Ansatzpunkte für zukünftige
Arbeiten zu identifizieren.

\section{Diskussion des Frameworks}

Die Architektur hat sich als tragfähige Grundlage erwiesen: Durch den
adapterbasierten Zugriff auf die Roboterschnittstelle (z.\,B. ABB Robot Web
Services) und eine konsequent \emph{event-getriebene} Struktur konnten
Safetymodule unabhängig voneinander entwickelt und über ein gemeinsames
Interface in das \emph{RobotSystem} integriert werden. Das Event-System
(Observer-Pattern) entkoppelt Erkennung und Verarbeitung; die JSON-basierte
Persistenz vereinfacht Nachvollziehbarkeit und Weiterverarbeitung. Positiv
wirkte sich zudem die Nutzung der Denavit–Hartenberg-Parameter aus, die eine
konsistente kinematische Modellierung verschiedener Robotermodlle erlaubt.\\

\noindent Gleichzeitig bleiben offen, inwiefern einzelne Simulationsparameter
oder Modellierungsentscheidungen in Unity bei hochkomplexen Geometrien eine
Limitation darstellen \emph{könnten}. Die Unity-Engine bietet hier zwar eine
Grundlage physischer Modellerung, abschliessende Aussagen zur Genauigkeit in
sehr dynamischen oder kleinschrittigen Prozessen können aber nicht getrofen
werden. Diese Aspekte wurden nicht systematisch untersucht und markieren Raum
für künftige Studien.\\

\noindent Das Experteninterview mit Daniel Syniawa ordnet die Arbeit in die
Praxis ein: Er bestätigte, dass es insgesamt nur wenige plattformübergreifende
Werkzeuge mit integrierter Physik gibt und dass die Tool-Landschaft stark
proprietär geprägt ist. Nach seiner Erfahrung ist bereits die stabile
Inbetriebnahme und Anbindung von Robotern für viele Firmen aufwendig;
physikalische Simulation mit einer Modellerung des Arbeitsraumes in RobotStudio
sowie die Auswertung der hier untersuchten Parameter ist nur in begrenztem
Umfang möglich und benötigt beträchtliches Expertenwissen und Zeit. Syniawa wies
außerdem darauf hin, dass Services proprietär Hersteller wie die RobotStudio-API
(Robot Web Services) nicht trivial in der Anwendung sind, oft schlecht
dokumentiert und auf eine insgesamt kleine Nutzerbasis trifft. Auch das lässt
sich im Rahmen der Entwicklung dieses Frameworks bestätigen.

Als hilfreich lässt sich ausserdem der Output des JSON-basierten Loggings
werten: Syniawa spricht hier vor allem von den mit einem SafetyEvent zusammen
herausgegebenen Metadaten zum aktuellen Stand des Programmzeigers und
der aktuellen
ausgeführten Routine. Dies stellt einen vielversprechenden Ansatz zum Debugging
im Roboterprogrammcode dar, welcher manuell mit deutlich mehr Aufwand verbunden
wäre. Hier gilt es weiter zu evaluieren, inwiefern sich dies quantitativ
beschreiben lässt, so Syniawa.

\section{Diskussion der Safetymodule}

\subsection{Prozessfolgenüberwachung}

Das Modul erkennt Abweichungen in der vorgesehenen Reihenfolge zuverlässig,
sofern Stationen korrekt modelliert und ausgelöst werden. Nicht abgedeckt sind
Konstellationen, in denen ein Werkstück \emph{zwischen} zwei Stationen
unbeabsichtigt abgelegt oder verloren wird, ohne dass eine Station detektiert
wird. Die Praxistauglichkeit hängt damit von der Qualität der
Prozessmodellierung und der Art des Prozesses ab. Im aktuellen Fokus stand hier
ein sequentieller Prozess, die Literatur beschreibt hier im Kontext
industrieller Fertigung jedoch mehrere Prozessarten und theoretische
Modellierungsansätze. Im Interview wurde die grundsätzliche Relevanz dieses
Moduls bestätigt; zugleich wird deutlich, dass die industrielle Praxis oft
komplexere Ablaufmodelle erfordert.

\subsection{Kollisionserkennung}

In der Simulation wurden die vorgesehenen Kollisionsfälle erkannt; zugleich
traten Fehlalarme auf, insbesondere beim Greifen und Loslassen von Werkstücken.
Wie stark Modellierungsdetails oder Simulationsparameter das Verhalten in
Grenzfällen beeinflussen, wurde in dieser Arbeit nicht systematisch untersucht
und ist als potenzielle Limitation zu betrachten. Weiterführend ist die
Genauigkeit der Modellierung der Meshes des Roboters und der Umgebung hier
essentiell: Unity modelliert konvexe Meshes, welche die räumlichen Grenzen
darstellen mit maximal 255 Kanten. Daher kann es passieren, dass die
tatsächliche Topologie des Robotermodells vom Kollisionskörper abweicht.
Insgesamt liefert das Modul einen belastbaren Proof-of-Concept, dessen
Generalisierung im Rahmen weiterführender Evaluierungen zu prüfen ist.

\subsection{Achsgeschwindigkeiten und -beschleunigungen}

Grenzwertverletzungen wurden zuverlässig detektiert. In den Ergebnissen zeigte
sich allerdings ein zeitlicher Versatz zwischen den in RobotStudio vorliegenden
Referenzdaten und der Detektion in Unity: Überschreitungen wurden etwas später
erkannt und endeten geringfügig später. Dieser Versatz ist plausibel auf
Aktualisierungsrate und eingesetzten Glättungsalgorithmus zurückzuführen, dessen
Parameter konfigurierbar sind. Ob dadurch ein Versatz mit den in der Praxis
vom Roboter gefahrenen Achsgeschwindigkeiten und -beschleunigungen
entsteht, lässt sich hier
nicht abschliessend bewerten.

\subsection{Singularitätserkennung}

Die gewählte, winkelbasierte Heuristik funktionierte für den betrachteten
Roboter, ist jedoch nicht universell. Alternativ bieten sich Kennzahlen an, die
näher an der Kinematik operieren, etwa das Manipulability-Maß (Yoshikawa) oder
der kleinste Singulärwert der Jacobi-Matrix als Abstandsmaß zur Singularität.
Eine generische, Jacobian-basierte Methode zur Erkennung von
Freiheitsgradverlusten wurde implementiert, im Rahmen der vorliegenden Tests
jedoch nicht eingesetzt; eine Erweiterung auf andere Roboter wäre möglich, wurde
aber nicht vorgenommen.

Im Interview formulierte Syniawa einen pragmatischen Maßstab für die Evaluation:
Für eine belastbare Beurteilung wäre ein direkter Vergleich mit manuellem
Debugging in RobotStudio sinnvoll, also ein Messaufbau, der die Zeit bis zur
Fehlerlokalisierung im RAPID-Code der Zeit gegenüberstellt, die das Framework
über Ausführung und Logging benötigt. Zugleich hob er hervor, dass die
automatische Bereitstellung von Programmzeiger, aktueller Pose und Kontext im
Event-Log eine erhebliche Arbeitserleichterung darstellt und die Analyse
prinzipiell auch weniger erfahrenen Anwendern ermöglicht.

\section{LLM-gestützte Rückkopplung auf Basis der Safety-Events}

Die Ergebnisse zeigen, dass das Framework Fehlerzustände konsistent
und kontextreich protokolliert: Für jedes Ereignis liegen
Aktuelle Bewegungs- und Programmdaten, beispielsweise das aktuelles
Modul, Routine,
Programmzeile sowie Motordaten und Achswinkel vor. Hinzu kommen
event-spezifische Felder im \texttt{eventDataJson}, etwa
Kollisionspunkt und Distanz oder Gelenkwinkel und
Manipulierbarkeitswert bei Singularitäten\footnote{Vgl. u.\,a.
  Process-Flow-Event inkl.\ Programmpointer und RobotStateSnapshot
  sowie die Beschreibung des JSON-Aufbaus; Kollisionsevents mit
  \texttt{collisionPoint} und Distanz; Singularitätsevents mit
  Gelenkwinkeln und Manipulierbarkeitswert; Joint-Dynamics-Events als
\enquote{exceeded}/\enquote{resolved}.}. Diese maschinenlesbare
Struktur eignet sich unmittelbar als gezielter Eingabekontext für
generative Modelle: Der Fehler wird präzise beschrieben, der
relevante Zustandsausschnitt ist enthalten, und die Semantik stammt
aus den domänenspezifischen Monitoren. Das bietet eine gute Basis, um Code-
oder Pfadänderungen vorzuschlagen und anschließend identisch zu
verifizieren.

Operativ lässt sich darauf ein kurzer Iterationszyklus aufbauen: Aus
einem fehlgeschlagenen Lauf durch Nutzer- oder LLM-generierten
Roboterprogrammcode wird ein kompakter Fehlerkontext aus
gebildet und an ein LLM übergeben: Das
LLM liefert einen minimalen Änderungsvorschlag am Robotercode
bzw.\ an der Pfaddefinition und durch die Re-Simulation kann eine erneute
Prüfung des Codes stattfinden. Die Akzeptanzkriterien für eine erfolgreiche
Behebung des Fehlers beispielsweise mit dem minimalen Achswinkel bereits
im Rahmen der Eventdaten und Monitorlogik im System vor wodurch die
Bewertung reproduzierbar bleibt.
Damit kann das Framework als
Schnittstelle zwischen realitätsnaher Ausführung und
LLM-gestützter Verbesserung dienen. Praktisch sind drei Punkte entscheidend
und aus den Ergebnissen ableitbar: erstens ein schlanker, stabiler
Prompt-Ausschnitt aus genau den Feldern, die im Logging ohnehin
verfügbar sind (z.\,B.\ Programmpointer, Gelenkwinkel,
Kollisionspunkt), zweitens feste Akzeptanzregeln in der Simulation,
drittens Versionierung und Wiederholbarkeit der Läufe. Auf dieser
Basis lässt sich in zukünftiger Arbeit untersuchen, in welchem Umfang
sich Fehlerquote und Nachbearbeitungszeit durch die Rückkopplung
tatsächlich reduzieren – die in den Ergebnissen sichtbaren Muster wie
liefern bereits geeignete Zielgrößen.

\section{Reflexion des eigenen Vorgehens}

Die Entwicklung folgte bewusst einem iterativen Vorgehen: von einem kleinen
Testfall hin zu einer breiteren Abdeckung, mit Zwischenstufen des Refactorings.
Dieses Vorgehen erwies sich als geeignet, um Architekturentscheidungen
(Adapter/Observer) empirisch zu validieren. Rückblickend entstanden stellenweise
Komponenten, die für den unmittelbaren Use-Case komplexer waren als nötig;
gleichwohl war die iterativ-explorative Herangehensweise im Kontext einer
Framework-Entwicklung zweckmäßig und hat zur jetzigen Struktur geführt.

\section{Grenzen und Generalisierbarkeit}

Die vorliegende Arbeit wurde in der Simulation evaluiert; Echtzeitverhalten,
Sensitivität gegenüber Simulationsparametern sowie Übertragbarkeit auf weitere
Roboter wurden nicht systematisch untersucht. Darüber hinaus beschränkt sich die
Adaptervalidierung auf ABB Robot Web Services. Das Interview verdeutlicht, dass
proprietäre Ökosysteme, komplexe Schnittstellen und eine kleine Nutzerbasis
zusätzliche Hürden für Verallgemeinerung und Transfer darstellen. Diese Punkte
markieren bewusste Grenzen des aktuellen Stands und leiten unmittelbar zu
Folgestudien über.

\chapter{Fazit und Ausblick}
\label{cap:Fazit}

\section{Zusammenfassung}
\label{sec:Zusammenfassung}
% Kurze Zusammenfassung der zentralen Erkenntnisse...

\section{Ausblick}
\label{sec:Ausblick}
% Aufzeigen von Möglichkeiten für zukünftige Forschung...


% Literaturverzeichnis
\printbibliography

% Anhang
\chapter*{Anhang}
\label{cap:Anhang}

\addcontentsline{toc}{chapter}{Anhang} 

%%%%%%%%%%%%%%%%%%%%%%%%%%%%%%%%%%%%%%%%%%%%%%%%%%%%%%%%%%%%%%%%%%%%%%%%%%%%%%%%%%%%%%%%%%%%%%%%%%%%%%%%%%%%%%%%%%%%%%%%%%%%%%%%%%%%%%%%%%%%%%%%%%%%%%%%%%%%%%%%%%%%%%%%%%%%%%%%%%%%%%%%%%%%%%%%%%%%%%%%%%%%%%%%%%%%%%%%%%%%%%%%%%%%%%%%%%%%%%%%%%%%%%%%
\section*{Anhang A}
\label{sec:Anhang A}
%%%%%%%%%%%%%%%%%%%%%%%%%%%%%%%%%%%%%%%%%%%%%%%%%%%%%%%%%%%%%%%%%%%%%%%%%%%%%%%%%%%%%%%%%%%%%%%%%%%%%%%%%%%%%%%%%%%%%%%%%%%%%%%%%%%%%%%%%%%%%%%%%%%%%%%%%%%%%%%%%%%%%%%%%%%%%%%%%%%%%%%%%%%%%%%%%%%%%%%%%%%%%%%%%%%%%%%%%%%%%%%%%%%%%%%%%%%%%%%%%%%%%%%%

...


%%%%%%%%%%%%%%%%%%%%%%%%%%%%%%%%%%%%%%%%%%%%%%%%%%%%%%%%%%%%%%%%%%%%%%%%%%%%%%%%%%%%%%%%%%%%%%%%%%%%%%%%%%%%%%%%%%%%%%%%%%%%%%%%%%%%%%%%%%%%%%%%%%%%%%%%%%%%%%%%%%%%%%%%%%%%%%%%%%%%%%%%%%%%%%%%%%%%%%%%%%%%%%%%%%%%%%%%%%%%%%%%%%%%%%%%%%%%%%%%%%%%%%%%
\section*{Anhang B}
\label{sec:Anhang B}
%%%%%%%%%%%%%%%%%%%%%%%%%%%%%%%%%%%%%%%%%%%%%%%%%%%%%%%%%%%%%%%%%%%%%%%%%%%%%%%%%%%%%%%%%%%%%%%%%%%%%%%%%%%%%%%%%%%%%%%%%%%%%%%%%%%%%%%%%%%%%%%%%%%%%%%%%%%%%%%%%%%%%%%%%%%%%%%%%%%%%%%%%%%%%%%%%%%%%%%%%%%%%%%%%%%%%%%%%%%%%%%%%%%%%%%%%%%%%%%%%%%%%%%%

...




% Eidesstaatliche Erklärung
\addchap*{Eidesstattliche Erklärung}

Hiermit versichere ich an Eides statt,
\begin{itemize}
  \item dass ich die vorliegende fachwissenschaftliche Arbeit
    selbstständig angefertigt und mich ausschließlich der im
    beigefügten Verzeichnis angegebenen Hilfsmittel bedient habe.
  \item dass diese fachwissenschaftliche Arbeit keiner anderen
    Prüfungsbehörde in gleicher oder ähnlicher Form vorgelegt wurde.
  \item dass ich jegliche wörtlich und sinngemäß aus veröffentlichten
    oder nicht veröffentlichten Quellen übernommenen Passagen
    eindeutig kenntlich gemacht habe.
\end{itemize}
\bigskip
\bigskip
\bigskip
\bigskip
\bigskip

\begin{flushright}
  \begin{tabular}{c}
    Ort, Datum
    \vspace*{1.5cm} \\

    %\hrulefill \\
    \dotfill \\
    Vor- und Nachname \\
    \hspace{5cm} \\
  \end{tabular}
\end{flushright}

\addchap*{Eidesstattliche Erklärung}

Hiermit versichere ich an Eides statt,
\begin{itemize}
  \item dass ich die vorliegende fachwissenschaftliche Arbeit
    selbstständig angefertigt und mich ausschließlich der im
    beigefügten Verzeichnis angegebenen Hilfsmittel bedient habe.
  \item dass diese fachwissenschaftliche Arbeit keiner anderen
    Prüfungsbehörde in gleicher oder ähnlicher Form vorgelegt wurde.
  \item dass ich jegliche wörtlich und sinngemäß aus veröffentlichten
    oder nicht veröffentlichten Quellen übernommenen Passagen
    eindeutig kenntlich gemacht habe.
\end{itemize}
\bigskip
\bigskip
\bigskip
\bigskip
\bigskip

\begin{flushright}
  \begin{tabular}{c}
    Ort, Datum
    \vspace*{1.5cm} \\

    %\hrulefill \\
    \dotfill \\
    Vor- und Nachname \\
    \hspace{5cm} \\
  \end{tabular}
\end{flushright}


\end{document}
