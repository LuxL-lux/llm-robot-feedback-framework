\addchap{Zusammenfassung}
Die Programmierung von Industrierobotern erfordert bislang
detaillierte Kenntnisse in herstellerspezifischen Programmiersprachen
sowie tiefgehendes kinematisches Fachwissen. Mit der Verfügbarkeit
generativer Sprachmodelle (engl. Large Language Models, LLMs) ergibt sich
die Möglichkeit, Roboterprogramme aus natürlichsprachlichen
Beschreibungen zu erzeugen. Ein direkter Einsatz solcher Ansätze im
industriellen Umfeld ist jedoch mit erheblichen Risiken verbunden, da
fehlerhafte Bewegungsabläufe oder Prozesslogiken zu Kollisionen,
Produktionsausfällen oder sicherheitskritischen Zuständen führen
können. Eine simulationsbasierte Validierung stellt daher einen
notwendigen Zwischenschritt dar, um generierten Code zunächst
virtuell zu prüfen und Risiken vor einer Übertragung auf reale
Systeme zu minimieren.

Im Rahmen dieser Arbeit wird ein Framework zur simulationsbasierten
Validierung von LLM-generiertem Robotercode entwickelt, implementiert
und evaluiert. Die Realisierung erfolgt in Unity3D unter Einbindung
eines ABB IRB 6700 als virtuellen Testroboter. Das Framework basiert
auf einer modularen, ereignisgesteuerten Architektur und integriert
Überwachungsmechanismen, die Prozessfolgen prüfen, Kollisionen
detektieren, kinematische Singularitäten identifizieren sowie
Gelenkwinkel, -geschwindigkeiten und -beschleunigungen überwachen.
Alle erfassten Abweichungen werden systematisch aufgezeichnet und mit
zeitlichen sowie kontextbezogenen Informationen ergänzt, um eine
spätere Auswertung zu ermöglichen.

Die Funktionalität wurde anhand eines realitätsnahen
Pick-and-Place-Szenarios untersucht, in dem gezielt fehlerhafte
Sequenzen und kritische Roboterzustände provoziert wurden. Ergänzend
erfolgte eine qualitative Reflexion im Rahmen eines
Experteninterviews. Die Ergebnisse zeigen, dass sich
Prozessabweichungen, Kollisionen, Singularitäten und
Grenzwertverletzungen zuverlässig erkennen und reproduzierbar
dokumentieren lassen. Damit leistet die Arbeit einen Beitrag zur
sicheren Integration von LLMs in die industrielle Robotik, indem sie
eine methodische Grundlage für die Validierung generierten Codes
bereitstellt und eine Perspektive für die Rückkopplung von
Simulationsergebnissen an Sprachmodelle eröffnet.
