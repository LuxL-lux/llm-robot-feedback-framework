\chapter{Implementierung des Frameworks}
\label{cap:framework}
In diesem Kapitel werden die technischen Grundlagen sowie das Vorgehen zur Implementierung des Frameworks beschrieben.

\section{Architektur des Frameworks}
\label{sec:architektur_frameowork}

\subsection{Zielsetzung und architektonische Anforderungen}

Daher verfolgt das Framework verfolgt drei \textbf{zentrale architektonische
    Ziele}:

\begin{enumerate}
    \item \textbf{Vendor-Agnostik}: Abstraktion verschiedener Roboterhersteller durch einheitliche Interface-basierte Architektur ohne herstellerspezifische Abhängigkeiten im Kern-Framework

    \item \textbf{Modulare Erweiterbarkeit}: Plugin-System für Safety Monitoring Module und Kommunikationsprotokolle ohne Änderungen der bestehenden Architektur

    \item \textbf{Echtzeitfähige Kommunikation}: Latenzarme Datenübertragung für Motion Control und ereignisbasierte Sicherheitsüberwachung
\end{enumerate}

\subsection{Unity3D als Simulationsplattform}

\subsubsection{Auswahl und Vorteile gegenüber Alternativen}

Die Wahl von Unity3D als zugrundeliegende Simulationsplattform basiert auf
mehreren technischen und praktischen Erwägungen. Während es bereits mehrere
kommerzielle Programme für die Gestaltung und Simulation von Robotern in
virtuellen Umgebungen gibt, sind diese nur selten mit anderen CAD-Systemen und
Robotern kompatibel, unterstützen nicht alle Roboterbibliotheken oder werden
nur plattformabhängig angeboten (vgl. \cite[S. 247]{andaluz2016}). Unity3D
hingegen ist mit den meisten CAD-Systemen kompatibel und bietet eine
plattformübergreifende Lösung.

\subsubsection{3D-Rendering und Physik-Simulation}

Unity3D bietet eine ausgereifte 3D-Rendering-Pipeline mit integrierter
Physik-Engine, welche zur Simulation von Gegenständen mit realitätsnahem
Verhalten sowie komplexen Arbeitsräumen geeignet ist (vgl.
\citet{Unity2025SystemRequirements}, \citet{Unity2025PhysicsOverview}). Die
Engine wurde bereits erfolgreich in der wissenschaftlichen Forschung eingesetzt
und bietet Module und Plugins für spezifische Anwendungsfälle im
Simulationsbereich. Unity3D ermöglicht es auch Nicht-Programmierern,
leistungsstarke Animations- und Interaktionsdesign-Tools zu nutzen, um Roboter
visuell zu programmieren und zu animieren (vgl. \cite[S. 431]{bartneck2015}).

\subsubsection{Technische Architektur und Programmierung}

Technisch ermöglicht Unity3D durch seine Scripting-Runtime (basierend auf
Mono/.NET Framework) die Verwendung moderner C\#-Sprachfeatures für
nebenläufige Prozesse und asynchroner Programmierung
\cite[S.~45-52]{unity_async_2023}, was es ermöglicht, Visualisierung,
Datenakquise und Überwachung zu trennen. Die .NET-basierte Architektur
unterstützt dabei sowohl Task-basierte asynchrone Operationen als auch
Coroutines für zeitgesteuerte Prozesse
\cite[S.~123-135]{unity_coroutines_2023}, welche die notwendige periodische
Ausführung von Prozessen auf verschiedenen Ebenen des Frameworks stark
vereinfacht.

Die Unity-Engine unterstützt nativ Multithreading durch das Job System
\cite[S.~201-218]{unity_jobsystem_2023}, was für die parallele Verarbeitung von
Sensordaten, Kollisionserkennung und Bewegungsplanung entscheidend ist.

\subsubsection{Entwicklungsumgebung und Debugging-Tools}

Ein weiterer entscheidender Vorteil für die Robotik-Simulation liegt in der
Verfügbarkeit visueller Programmiertools und der integrierten
Entwicklungsumgebung. Die Plattform bietet umfangreiche Debugging- und
Profiling-Werkzeuge (Unity Profiler, Frame Debugger), die während der
Entwicklung und zur Laufzeit genutzt werden können
\cite[S.~67-89]{unity_profiler_2023}. Diese Werkzeuge ermöglichen die Analyse
von Performance-Engpässen bei der Verarbeitung von Roboterdaten und die
Optimierung der Sicherheitsmonitor-Updates.

Darüber hinaus lassen sich während der Laufzeit sowohl die Szene (hier: die
Roboterzelle) als auch Komponenten-Parameter in Echtzeit bearbeiten und
einsehen \cite[S.~1236]{haas_realtime_2022}, was das Debuggen und Testen
beschleunigt.

\subsubsection{Benutzerfreundlichkeit und industrielle Anwendung}

Besonders relevant für industrielle Robotik-Anwendungen ist die Möglichkeit,
Custom Editor Scripts zu entwickeln (vgl.
\cite[S.~156-172]{unity_editors_2023}), die eine benutzerfreundliche und
niederschwellige Konfiguration verschiedener Parameter ermöglichen. Zusätzlich
ermöglichen Gizmos und Scene View die visuelle Darstellung von Kollisionszonen,
Singularitätspunkten und Prozessabläufen während der Entwicklung (vgl.
\cite[S.~234-245]{unity_gizmos_2023}).

\subsection{Design Patterns und Prinzipien}

Die Entwicklung eines modularen und erweiterbaren Robotersicherheitssystems
erfordert eine fundierte methodische Herangehensweise, die auf bewährten
Software-Engineering-Prinzipien basiert. Die Auswahl geeigneter Design Patterns
und Architekturprinzipien determiniert maßgeblich die Qualitätsattribute des
Systems wie Wartbarkeit, Testbarkeit und Erweiterbarkeit (vgl. \cite{Bass2012},
S. 73-75). Im Folgenden werden die für diese Arbeit gewählten Entwurfsmuster
und deren Begründung dargelegt.

\subsubsection{Observer Pattern als Kommunikationsparadigma}
Für die Realisierung der systemweiten Kommunikation wird das Observer Pattern
(vgl. \cite{Gamma1994}, S. 293-303) als zentrales Entwurfsmuster gewählt. Diese
Entscheidung basiert auf drei wesentlichen Anforderungen industrieller
Robotersysteme: Erstens müssen Sicherheitsereignisse ohne Verzögerung an alle
relevanten Systemkomponenten propagiert werden, was durch die inhärente
Entkopplung des Observer Patterns gewährleistet wird. Zweitens ermöglicht das
Muster die dynamische Registrierung und Deregistrierung von Beobachtern zur
Laufzeit, was für modulare Sicherheitssysteme unerlässlich ist (vgl.
\cite{Buschmann1996}, S. 127-128). Drittens reduziert die lose Kopplung
zwischen Publisher und Subscriber die Systemkomplexität erheblich, da
Komponenten ohne Kenntnis voneinander interagieren können.

Das Observer Pattern adressiert zudem die Herausforderung der
Multi-Threading-Umgebung in Unity3D, indem Events asynchron verarbeitet werden
können, ohne den Hauptthread zu blockieren (vgl. \cite{Nystrom2014}, S.
156-159). Dies ist besonders kritisch für die Echtzeitverarbeitung von
Sensordaten und die gleichzeitige Visualisierung.

\subsubsection{Strategy Pattern für algorithmische Flexibilität}
Die Wahl des Strategy Patterns (vgl. \cite{Gamma1994}, S. 315-323) für die
Implementierung von Sicherheitsmonitoren und Datenparser begründet sich durch
die Heterogenität industrieller Robotersysteme. Verschiedene Roboterhersteller
verwenden proprietäre Kommunikationsprotokolle und Datenformate, was eine
flexible Austauschbarkeit von Parsing-Algorithmen erfordert (vgl.
\cite{Siciliano2016}, S. 891-893). Das Strategy Pattern kapselt diese
Algorithmen in separaten Klassen und macht sie über eine gemeinsame
Schnittstelle austauschbar.

Die Vorteile dieser Architekturentscheidung manifestiert sich vornehmlich
darin, dass so Robotertypen ohne Modifikation des Kernsystems integriert
werden. \citeauthor{Martin2003} spricht hier vom Open-Closed-Prinzip, welches
die Offenheit von Software-Entitäten (Funktionen, Klassen, Module, Komponenten
usw.) zur Extension und die gleichzeitige Geschlossenheit zur Modfikation
beschreibt.

\subsubsection{Adapter Pattern zur Hardware-Abstraktion}
Die Integration heterogener Hardware-Komponenten erfordert eine
Abstraktionsschicht zwischen der Anwendungslogik und den hardware-spezifischen
Schnittstellen. Das Adapter Pattern (vgl. \cite{Gamma1994}, S. 139-150) wird
gewählt, um diese Abstraktion zu realisieren. Die Notwendigkeit ergibt sich aus
der Vielfalt der Robotersteuerungen und Visualisierungssysteme, die jeweils
eigene APIs und Datenformate verwenden (vgl. \cite{Craig2005}, S. 412-415).

Durch die Adapter-Schicht wird eine einheitliche Schnittstelle zur Verfügung
gestellt, die es ermöglicht, verschiedene Robotersysteme ohne Änderung der
Kernlogik anzubinden. Dies reduziert nicht nur die Komplexität des Systems,
sondern erhöht auch dessen Portabilität und Wiederverwendbarkeit (vgl.
\cite{Vlissides1995}, S. 89-91).

\subsubsection{SOLID-Prinzipien als Qualitätsfundament}
Die konsequente Anwendung der SOLID-Prinzipien (vgl. \cite{Martin2003}, S.
95-135) bildet das methodische Fundament der Systemarchitektur. Das
\textit{Single Responsibility Principle} wird angewendet, um kohäsive Module zu
schaffen, die genau eine Verantwortlichkeit haben. Dies reduziert die Kopplung
und erhöht die Verständlichkeit des Codes (vgl. \cite{Martin2017}, S. 62-64).
Das \textit{Open-Closed Principle} gewährleistet, dass das System für
Erweiterungen offen, aber für Modifikationen geschlossen ist – eine essenzielle
Eigenschaft für langlebige Industriesysteme.

Das \textit{Dependency Inversion Principle} wird konsequent angewendet, indem
High-Level-Module von Abstraktionen abhängen, nicht von konkreten
Implementierungen. Dies ermöglicht die flexible Konfiguration des Systems zur
Laufzeit und vereinfacht die Integration in verschiedene Produktionsumgebungen
(vgl. \cite{Fowler2018}, S. 112-115).

\subsubsection{Event-Driven Architecture für Echtzeitfähigkeit}
Die Entscheidung für eine event-getriebene Architektur basiert auf den
Echtzeitanforderungen industrieller Robotersysteme. Sicherheitskritische
Ereignisse müssen innerhalb definierter Zeitschranken verarbeitet werden, was
durch synchrone Aufrufketten nicht gewährleistet werden kann (vgl.
\cite{Hohpe2003}, S. 97-99). Die event-getriebene Architektur ermöglicht die
asynchrone Verarbeitung von Ereignissen und die Priorisierung kritischer
Sicherheitsereignisse.

Zudem adressiert dieser Ansatz die Herausforderung der Integration
verschiedener Datenquellen – von zyklischen Sensordaten über sporadische Alarme
bis zu kontinuierlichen Videoströmen. Jede Datenquelle kann Events in ihrem
eigenen Takt generieren, ohne andere Systemkomponenten zu blockieren (vgl.
\cite{Vernon2013}, S. 234-237).

\subsection{Systemarchitektur und Entwurfsmuster}
\dirtree{%
    .1 RobotSystem/.
    .2 Core/.
    .3 RobotManager.cs.
    .3 RobotState.cs.
    .3 RobotSafetyManager.cs.
    .3 SafetyEvent.cs.
    .3 RobotStateSnapshot.cs.
    .3 Part.cs.
    .3 Station.cs.
    .3 RapidTargetGenerator.cs.
    .2 Interfaces/.
    .3 IRobotConnector.cs.
    .3 IRobotDataParser.cs.
    .3 IRobotSafetyMonitor.cs.
    .3 IRobotVisualization.cs.
    .2 ABB/.
    .3 RWS/.
    .4 ABBRWSConnectionClient.cs.
    .4 ABBRWSDataParser.cs.
    .4 ABBMotionDataService.cs.
    .3 ABBFlangeAdapter.cs.
    .2 Monitors/.
    .3 CollisionDetectionMonitor.cs.
    .3 JointDynamicsMonitor.cs.
    .3 ProcessFlowMonitor.cs.
    .3 SingularityDetectionMonitor.cs.
}

\paragraph{Event-Driven Architecture}
Das System nutzt ein durchgängiges Event-System für lose Kopplung:
\begin{itemize}
    \item \texttt{OnRobotStateUpdated}: Zustandsänderungen
    \item \texttt{OnConnectionStateChanged}: Verbindungsstatus
    \item \texttt{OnSafetyEventDetected}: Sicherheitsereignisse
    \item \texttt{OnMotorStateChanged}: Motorstatusänderungen
\end{itemize}

\paragraph{SafetyEvent und RobotStateSnapshot}
Das \texttt{SafetyEvent}-System implementiert ein umfassendes Ereignismodell:
\begin{itemize}
    \item \textbf{SafetyEvent}: Unveränderliches Value Object für Sicherheitsereignisse
    \item \textbf{RobotStateSnapshot}: Immutable Zustandserfassung zum Ereigniszeitpunkt
    \item \textbf{Ereignistypen}: Info, Warning, Critical mit konfigurierbaren Schwellwerten
    \item \textbf{Kontextdaten}: Vollständige Roboterzustandserfassung für Forensik
\end{itemize}

\paragraph{Datenformate}
\begin{itemize}
    \item \textbf{HTTPS/HTTP}: RESTful API-Zugriff mit Digest-Authentifizierung
    \item \textbf{WebSocket Secure (WSS)}: Bidirektionale Echtzeit-Kommunikation
    \item \textbf{XML}: Strukturierte Datenübertragung mit Schema-Validierung
    \item \textbf{JSON}: Interne Datenrepräsentation und Logging-Format
\end{itemize}

\section{Implementierung der Module}

\subsection{Allgemeine Struktur}

\subsection{Prozessfolgen}
\label{ssec:Prozessfolgen}
% Überprüfung der korrekten Abfolge von Aktionen...
User Input: Simulationsumgebung und Verbindung zu Robot Studio

Wie kann ich Prozessfolgen überprüfen? Prozessfolge: Folge and Arbeitsschritten
eines Arbeitsprozesses, hier Roboter -> Was wird in welcher Reihenfolge wohin
bewegt? Wie kann ich das Messen? - Bewegt sich das Werkstück von Position Start
zu Position Ziel? - Bewegen sich die Werkstücke in der richtigen Reihenfolge
von Start zu Ziel? Benötigt: Definition von Start \& Zielpositionen einzelner
Werkstücke

\subsection{Kollisionen}
\subsubsection{Theoretische Grundlagen}

Kollisionen eines Roboters mit Objekten im Arbeitsraum stellen eine zentrales
Problem beim Neuentwurf und der Testung neu entwickelten Codes dar. Kollidiert
ein industrieller Roboter unvorhergesehen mit seinem Arbeitsraum, kann dies im
schlimmsten Fall zu einem Ausfall des am Produktionsschrit beteiligen Roboters
\subsubsection{Implementierung}



\subsection{Singularity Detection Monitor} \label{ssec:Singularitaeten}

\subsubsection{Theoretische Grundlagen der Singularitätsdetektion}
\label{sssec:Theorie_Singularitaeten}
Kinematische Singularitäten stellen ein fundamentales Problem in der
Robotersteuerung dar und treten auf, wenn die Jacobi-Matrix des Roboters ihren
vollen Rang verliert. In diesen Konfigurationen verliert der Roboter die
Fähigkeit, sich in bestimmte Richtungen im kartesischen Raum zu bewegen, was zu
Kontrollverlust und potentiell gefährlichen Situationen führen
kann.\vglcite{siciliano2010robotics}

\paragraph{Mathematische Definition} Eine kinematische Singularität tritt auf,
wenn die Jacobi-Matrix $\mathbf{J}(\boldsymbol{\theta})$, welche des Roboters
ihren vollen Rang verliert: \begin{equation}
	\text{rank}(\mathbf{J}(\boldsymbol{\theta})) < \min(m, n)
	\label{eq:singularity_condition} \end{equation} wobei
$\mathbf{J}(\boldsymbol{\theta}) \in \mathbb{R}^{m \times n}$ die Jacobi-Matrix,
$\boldsymbol{\theta}$ der Gelenkwinkelvektor, $m$ die Anzahl der Freiheitsgrade
im kartesischen Raum und $n$ die Anzahl der Robotergelenke darstellt.
\noindent
Die Jacobi-Matrix beschreibt die Beziehung zwischen Gelenkgeschwindigkeiten
$\dot{\boldsymbol{\theta}}$ und kartesischen Geschwindigkeiten des TCP
$\mathbf{v}$: \begin{equation} \mathbf{v} = \mathbf{J}(\boldsymbol{\theta})
	\dot{\boldsymbol{\theta}} \label{eq:jacobian_velocity}
\end{equation}
\noindent
Tritt eine Singularität auf, so wird die Jacobi-Matrix singulär.
\begin{equation} \det(\mathbf{J}) = 0 \label{eq:jacobian_singularity}
\end{equation} Dadurch wird die inverse Kinematik nicht
eindeutig lösbar ist und es können theoretisch unendliche
Gelenkgeschwindigkeiten auftreten.\vglcite{nakamura1991advanced}

\subsubsection{Frameworkspezifisches Vorgehen} Im
Rahmen der praktischen Implementierung wird hier ein auf Schwellwerten und
absoluten sowie winkelbasierten Entfernungen zur Singularität angewendet. Durch
die Anwendung der rein mathematischen Definiton liesse sich zwar jede beliebige
Singularität in einer seriellen Kinematik erkennen, jedoch bleiben die
betroffenen Gelenke und Art der Singularität ungewiss und müssten iterativ
bestimmt werden. Daher wird hier eine pragmatische Berechnungsmethodik
verwendet.\\

\noindent
Für serielle Robotermanipulatoren mit sechs Freiheitsgraden wie den hier
verwendeten ABB IRB 6700 können drei primäre Singularitätstypen unterschieden
werden: Schultersingularitäten, Ellbogensingularitäten und
Handgelenksingularitäten.\vglcite{spong2006robot}

\paragraph{Schultersingularitäten} treten auf, wenn sich das
Handgelenkszentrum (Schnittpunkt der Achsen 4, 5, 6) direkt über oder nahe der
Rotationsachse des ersten Gelenks befindet:

\begin{equation}
	\sqrt{x_{wc}^2 + y_{wc}^2} < \epsilon
	\label{eq:shoulder_singularity}
\end{equation}
\noindent
wobei $(x_{wc}, y_{wc})$ die Position des Handgelenkszentrums in der XY-Ebene
und $\epsilon$ der Singularitätsschwellenwert ist.

\paragraph{Ellbogensingularitäten} treten auf, wenn der Roboter die Grenzen seines
Arbeitsraums erreicht. Dies geschieht typischerweise bei vollständig
ausgestreckter oder eingeklappter Konfiguration. Beim Knickarmrobotern wie dem IRB 6700
verläuft die Rotationsachse des vierten Gelenks nicht direkt durch den Ursprung vom
dritten Gelenk, sondern ist durch eine Translation verschoben. Das impliziert,
dass die vollständige Streckung des Armes nicht durch einen Winkel von $0°$ bzw.
$180°$ auftritt. Hier lässt sich allgemeiner der Zusammenhang der Vektoren
zwischen Gelenk 2 und 3 sowie 2 und 5 anwenden:

\begin{equation}
	\theta = \angle(\vec{v}_{23}, \vec{v}_{25}) \approx 0° \text{ oder } \theta \approx 180°
	\label{eq:elbow_singularity}
\end{equation}

wobei:
\begin{align}
	\vec{v}_{23} & = \vec{p}_3 - \vec{p}_2 \\
	\vec{v}_{25} & = \vec{p}_5 - \vec{p}_2
\end{align}

mit $\vec{p}_i$ als Position des Gelenks $i$. Die Singularitätsbedingung lautet:
\begin{equation}
	\theta < \epsilon \quad \text{oder} \quad \theta > 180° - \epsilon
\end{equation}

\paragraph{Handgelenksingularitäten} entstehen, wenn die Rotationsachsen der
letzten drei Gelenke (Gelenke 4, 5, 6) kollinear werden. Dies tritt
typischerweise auf, wenn $\theta_5 = 0°$ oder $\theta_5 = 180°$. Mathematisch
beschrieben durch:

\begin{equation}
	\mathbf{z}_4 \parallel \mathbf{z}_6 \text{ oder } |\mathbf{z}_4 \cdot \mathbf{z}_6| \approx 1
	\label{eq:wrist_singularity}
\end{equation}
\noindent
wobei $\mathbf{z}_i$ die Rotationsachse (Z-Achse) des $i$-ten Gelenks im
Weltkoordinatensystem darstellt.

\paragraph{Manipulierbarkeitsindex (Yoshikawa-Maß)} Der von Yoshikawa
\cite{yoshikawa1985manipulability} eingeführte Manipulierbarkeitsindex ist eine
der am häufigsten verwendeten Metriken: \begin{equation}
	\mu(\boldsymbol{\theta}) =
	\sqrt{\det(\mathbf{J}(\boldsymbol{\theta})\mathbf{J}^T(\boldsymbol{\theta}))}
	\label{eq:yoshikawa_measure} \end{equation}

Für quadratische Jacobi-Matrizen vereinfacht sich dies zu: \begin{equation}
	\mu(\boldsymbol{\theta}) = |\det(\mathbf{J}(\boldsymbol{\theta}))|
	\label{eq:yoshikawa_simplified} \end{equation}
\noindent
Der Index nimmt Werte zwischen 0 (Singularität) und einem maximalen Wert an,
wobei höhere Werte bessere Manipulierbarkeit indizieren. Die Robotik-Literatur
bietet verschiedene Ansätze zur Behandlung von Singularitäten, die sich in
präventive und reaktive Strategien unterteilen lassen. Um diese in der Praxis
anzuwenden, müssen jedoch teilweise numerische Auswertungen durchgeführt werden,
um Schwellwerte zu generieren. Daher kommen diese hier nicht zum Einsatz.

\subsubsection{Praktisches Vorgehen} \label{sssec:Framework_Implementierung}

\begin{figure}[H]
	\centering
	\includegraphics[width=\linewidth]{Figures/wristSingularityScreenshot.jpg}
	\caption{Screenshot einer Handgelenksingularität in Unity mit farbig
		dargestellten Koordinatensystemen der DH-Transformationen}
	\label{figure:wristSingularity}
\end{figure}

\paragraph{Schritt 1: Positionsberechnung und Achsentransformation}~\\
Die Positionen der relevanten Gelenke werden aus den Forward-Kinematics berechnet:
\begin{equation}
	\vec{p}_i = \text{FK}(\theta_1, \ldots, \theta_i) \quad \text{für } i = 2, 3, 5
	\label{eq:position_calculation}
\end{equation}
\noindent
Die Rotationsachsen werden aus den Transformationsmatrizen extrahiert:
\begin{equation}
	\mathbf{z}_i = \mathbf{T}_i[:3, 2] \quad \text{für } i = 1, 4, 6
	\label{eq:axis_extraction}
\end{equation}

\paragraph{Schritt 2: Singularitätsanalyse}~\\
Für jeden Singularitätstyp wird die entsprechende Bedingung überprüft:
\begin{itemize}
	\item \textbf{Schultersingularität:}
	      \begin{equation}
		      d_{wc} = \sqrt{x_{wc}^2 + y_{wc}^2}
	      \end{equation}
	      wobei $(x_{wc}, y_{wc})$ die Position des Handgelenkszentrums in der XY-Ebene ist.

	\item \textbf{Ellbogensingularität:}
	      \begin{equation}
		      \theta_{elbow} = \angle(\vec{p}_3 - \vec{p}_2, \vec{p}_5 - \vec{p}_2)
	      \end{equation}

	\item \textbf{Handgelenksingularität:}
	      \begin{equation}
		      c_{46} = |\mathbf{z}_4 \cdot \mathbf{z}_6|
	      \end{equation}
\end{itemize}

\paragraph{Schritt 3: Schwellwertvergleich}~\\
\begin{table}[H]
	\centering
	\begin{tabular}{l l l}
		\hline
		\textbf{Singularitätstyp} & \textbf{Beispiel-Schwellwert}      & \textbf{Bedingung}                            \\
		\hline
		Schulter                  & $\tau_{\text{shoulder}} = 100$ mm  & $d_{wc} < \tau_{\text{shoulder}}$             \\
		Ellbogen                  & $\tau_{\text{elbow}} = 5^{^\circ}$ & $\theta_{elbow} < \tau_{\text{elbow}}$ oder   \\
		                          &                                    & $\theta_{elbow} > 180° - \tau_{\text{elbow}}$ \\
		Handgelenk                & $\tau_{\text{wrist}} = 5^{^\circ}$ & $c_{46} > \tau_{\text{wrist}}$                \\
		\hline
	\end{tabular}
	\caption{Singularitätsschwellwerte und Detektionsbedingungen}
	\label{tab:singularity_thresholds}
\end{table}

\paragraph{Schritt 4: Manipulierbarkeitsberechnung}~\\
Für detektierte Singularitäten wird ein approximativer Manipulierbarkeitsindex
nach dem Yoshikawa-Maß berechnet:

\begin{equation}
	w = \sqrt{\det(\mathbf{J}(\theta)\mathbf{J}^T(\theta))}
	\label{eq:manipulability}
\end{equation}
\noindent
Ein kleiner Wert von $w$ indiziert die Nähe zu einer singulären Konfiguration.
Das entwickelte Framework implementiert eine winkelbasierte
Singularitätsdetektion, die geometrische Eigenschaften der Roboterkinematik
direkt nutzt, anstatt auf rechenintensive Jacobi-Matrix-Berechnungen angewiesen
zu sein. Die achsenbasierte Methode basiert auf der Erkenntnis, dass
Singularitäten geometrisch durch die Kollinearität von Rotationsachsen
charakterisiert werden können.

\subsubsection{Konkrete Implementierung der Detektionsmethoden}
\label{sssec:Implementierung_Detektionsmethoden}
Die Implementierung des \texttt{SingularityDetectionMonitor} nutzt die
Vorwärtskinematik des Preliy Flange Frameworks zur Berechnung der
Gelenkpositionen. Die zentrale Methode \texttt{ComputeJointPosition} berechnet
die kartesische Position eines beliebigen Gelenks durch sukzessive Anwendung
der Denavit-Hartenberg-Transformationsmatrizen:

\begin{figure}[H]
	\inputminted[fontsize=\footnotesize]{csharp}{code-snippets/CalculateJointPos.cs}
	\caption{Vorwärtskinematik zur Positionsberechnung}
	\label{listing:forwardKinematic}
\end{figure}

\noindent
Die Methode \texttt{ComputeJointPosition} in Abbildung \ref{listing:forwardKinematic} implementiert die klassische Vorwärtskinematik durch
Multiplikation homogener Transformationsmatrizen. Jede Matrix $\mathbf{T}_i$
wird aus den DH-Parametern $(\alpha_i, a_i, d_i, \theta_i)$ konstruiert, wobei
$\theta_i$ der aktuelle Gelenkwinkel plus einem konstanten Offset ist. Die
resultierende Transformationsmatrix beschreibt die Position und Orientierung des
Gelenks im Basiskoordinatensystem.\\

\noindent
Das Framework nutzt Unitys \texttt{Matrix4x4}-Klasse für die
Transformationsberechnungen und die \texttt{GetPosition()}-Methode zur
Extraktion der Translationskomponente. Die Koordinatentransformation zwischen
dem DH-Parametersystem und Unitys linkshändigem Y-up Koordinatensystem wird
dabei durch die Methode \texttt{HomogeneousMatrix.CreateRaw()} des
Flange-Frameworks transparent gehandhabt. Das ermöglicht eine nahtlose
Integration der mathematischen Robotik-Konzepte in die Unity-Umgebung, während
die Echtzeitfähigkeit durch eventgetriebene Berechnung gewährleistet ist.
Bei der Detektion einer Singularität bzw. dem Unterschreiten des im Unity-Editor
definierten Grenzwertes wird ein \texttt{SafetyEvent}-Objekt instanziert und an die
\texttt{SafetyMonitor} Klasse weitergegeben. Darin werden zusätzlich Event-Metadaten zu den
Daten der Gelenkwinkelabstände ausgegeben. Sobald der
kritische Bereich verlassen wurde, wird ein weiteres \texttt{SafetyEvent} ausgegeben, um
den Bereich, in welcher die Singularität auftritt, abstecken zu können.


\section{Testumgebung und -setup}
\subsection{Aufbau der Roboterzelle}
\subsection{Implementierung in Unity}
\section{Datenaufzeichung und Logging}
\subsection{JSON-Struktur}
\subsection{Speicherung}