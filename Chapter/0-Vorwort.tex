\chapter*{Aufgabenstellung}
\label{cap:Aufgabenstellung}

\section*{Thema: Entwicklung eines Frameworks zur simulationsbasierten Validierung LLM-generierten Robotercodes in Unity}
\label{sec:Thema: Entwicklung eines Frameworks zur simulationsbasierten Validierung LLM-generierten Robotercodes in Unity}
\noindent
Für den Einsatz von Arbeitsplatzsystemen im Bereich der Montage, an denen Mensch und Roboter miteinander kollaborieren (MRK), gibt es bislang noch keine digitalen Planungswerkzeuge, welche ein MRK-System hinsichtlich Automatisierbarkeit, tech-nisch-wirtschaftlicher Eignung, Ergonomie und Sicherheit simulieren und bewerten können.
Um zukünftig die einfache Planung und Simulation von kollaborativen Montagesystemen zu ermöglichen, wird im Forschungsprojekt "`KoMPI"' ein digitales Planungswerkzeug entwickelt, sodass sowohl Mensch als auch Roboter gezielt gemeinsam im Montageprozess eingesetzt werden können.
Derzeit existieren bereits unterschiedliche Simulationswerkzeuge, die Menschmodelle in manuellen Montagesystemen abbilden.
Hierzu zählt bspw. die digitale Planungssoftware \textsc{Ema} der Fa. imk.
Bis dato bietet auch \textsc{Ema} keine hinreichenden Möglichkeiten zur Simulation automatisierungstechnischer Komponenten (z.\,B. Roboter), insbesondere in Kollaboration mit dem Menschen.
...
Im Rahmen dieser Arbeit gilt es ...

\section*{Im Einzelnen sollten folgende Punkte bearbeitet werden:}
\noindent
\begin{itemize}
	\item Literaturrecherche zum Thema Mensch-Roboter-Kollaboration (MRK) und Planung
	      kollaborativer Montagesysteme
	\item ...
	\item mein name ist lukas und dieser satz wird immer laenger das wuerde bedeuten dass
	      sdfnmsdfs asdasda und was bedeutet das in echt?
\end{itemize}

\noindent
\textbf{Ausgabedatum:} 16.06.2025
