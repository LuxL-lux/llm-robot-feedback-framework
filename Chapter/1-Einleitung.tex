% \chapter{Einleitung}
% \label{cap:Einleitung}

% \section{Motivation und Relevanz}
% \label{sec:Motivation}
% Hier fügen Sie den Text für die Motivation ein.
% Warum lohnt es sich mit dem Thema zu beschäftigen
% Wie relevant ist es in Forschung sowie Praxis

% \section{Zielsetzung und Aufbau der Arbeit}
% \label{sec:Zielsetzung}
% Was will ich mit der Arbeit erreichen
% Versuchsaufbau
% In kapitel 2 wird, Kapitel 3 das, 4 das, 5 das
% Hier fügen Sie den Text für die Zielsetzung ein.
% Entwicklungsziel: Konzeption und prototypische Implementierung...
% Untersuchungsziel: Methodische Untersuchung der Formalisierung...
% Überblick über den Aufbau der Arbeit...
%

\chapter{Einleitung}
\label{cap:Einleitung}

\section{Motivation und Relevanz}
\label{sec:Motivation}

Die rasante technologische Entwicklung und die zunehmende Vielfalt
kundenindividuell konfigurierbarer Produkte führen dazu, dass
industrielle Produktionsanlagen immer häufiger neu eingerichtet oder
umgerüstet werden müssen. In der Robotik erfordert dies die
Erstellung und Anpassung von Programmen, die Bewegungsabläufe,
Bearbeitungsschritte und Sicherheitsfunktionen eines Roboters
definieren. Traditionelle Programmiermethoden sind komplex,
controllerspezifisch und setzen ein profundes Verständnis der
jeweiligen Kinematik und proprietären Sprachen voraus. Für kleine und
mittlere Unternehmen sowie für Fertigungsbereiche mit hoher
Variantenvielfalt stellt dies eine Hemmschwelle dar, zumal sich in
vielen Branchen ein Mangel an qualifizierten Fachkräften bemerkbar macht.

Gleichzeitig halten generative Sprachmodelle Einzug in die
Softwareentwicklung. Sie können aus natürlichsprachlichen
Beschreibungen ausführbaren Code erzeugen und versprechen so, die
Hürden der Roboterprogrammierung zu senken. In der Praxis ist der
direkte Einsatz solcher Large Language Models (LLMs) im
industriellen Kontext jedoch riskant: Fehlerhafte Bewegungssequenzen
oder fehlerhafte Prozesslogiken können zu Kollisionen, Schäden und
Produktionsausfällen führen. Im Zusammenspiel mit den vorherrschenden
Sicherheitsanforderungen ist daher eine sorgfältige Validierung des
generierten Codes unerlässlich. Simulationstools bieten die
Möglichkeit, Roboterprogramme risikofrei zu testen und Abweichungen
vom Soll‑Verhalten systematisch zu untersuchen. Die vorliegende
Arbeit verknüpft diese beiden Entwicklungen: Sie verknüpft
LLM‑basierte Codegenerierung mit einer simulationsbasierten Prüfung,
die sicherheitskritische Aspekte des Programmablaufs überwacht und
strukturierte Ereignisprotokolle liefert. Damit wird eine Lücke
zwischen der schnellen, aber potenziell unsicheren Codeerzeugung und
der industriellen Praxis geschlossen.

\section{Zielsetzung und Aufbau der Arbeit}
\label{sec:Zielsetzung}

Ziel dieser Arbeit ist die Konzeption, Umsetzung und Evaluation eines
Frameworks zur simulationsbasierten Validierung von mit LLMs
generiertem Robotercode. Kernidee ist, dass während der Simulation
auftretende Ereignisse – insbesondere Abweichungen in der
Prozessabfolge, Kollisionen mit der Umgebung oder dem Werkstück,
kinematische Singularitäten sowie Grenzwertverletzungen von
Achsgeschwindigkeiten – automatisch detektiert und in einer
standardisierten Form gespeichert werden. Diese sogenannten Safety
Events umfassen neben einer Beschreibung des erkannten Problems stets
einen vollständigen Zustandsabzug der Steuerung (RobotStateSnapshot),
der Positionsdaten, Programmzeiger und Metadaten umfasst. Durch die
Formalisierung in JSON können die Ergebnisse sowohl für eine manuelle
Analyse als auch für eine mögliche Rückkopplung an generative
Sprachmodelle genutzt werden. Auf diese Weise soll ein iterativer
Verbesserungsprozess entstehen, der die Qualität der erzeugten
Programme steigert und die Eintrittsbarrieren für den Einsatz von
Robotik reduziert.

Um dieses Ziel zu erreichen, wird zunächst der Stand der Technik zu
Robotersimulation, Offline‑Programmierung und LLMs zusammengefasst.
Darauf aufbauend wird die Architektur des Frameworks beschrieben, die
eine herstellerunabhängige Einbindung unterschiedlicher
Robotersteuerungen (über ein Adaptermuster) sowie die modulare
Erweiterung um weitere Safety‑Monitore ermöglicht. Die
Implementierung der vier Überwachungsmodule – Process Flow Monitor,
Collision Detection Monitor, Joint Dynamics Monitor und Singularity
Detection Monitor – wird anschließend detailliert erläutert. In einem
realitätsnahen Beispielszenario mit einer Roboterzelle werden die
Module getestet; spezifische Szenarien provozieren Abweichungen im
Ablauf, Kollisionen und Grenzwertverletzungen, um die Detektion zu
verifizieren. Ein Experteninterview ergänzt die Evaluation und
reflektiert die Eignung des Ansatzes aus praktischer Sicht.
Abschließend werden die gewonnenen Erkenntnisse diskutiert, Grenzen
des Prototyps aufgezeigt und zukünftige Erweiterungsschritte skizziert.

Der Aufbau der Arbeit gliedert sich wie folgt: Kapitel~2 stellt den
Stand der Technik dar und ordnet die Problemstellung in den Kontext
bestehender Forschungsarbeiten ein. Kapitel~3 beschreibt die
Konzeption und Implementierung des Frameworks sowie der einzelnen
Safetymodule. Kapitel~4 präsentiert die Ergebnisse der Simulationen
und des Experteninterviews. In Kapitel~5 werden die Resultate
kritisch diskutiert und die Leistungsfähigkeit, Grenzen und
Generalisierbarkeit des Ansatzes bewertet. Kapitel~6 schließt die
Arbeit mit Fazit und Ausblick ab und zeigt potenzielle Wege zur
Weiterentwicklung auf.
