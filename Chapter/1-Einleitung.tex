% \chapter{Einleitung}
% \label{cap:Einleitung}

% \section{Motivation und Relevanz}
% \label{sec:Motivation}
% Hier fügen Sie den Text für die Motivation ein.
% Warum lohnt es sich mit dem Thema zu beschäftigen
% Wie relevant ist es in Forschung sowie Praxis

% \section{Zielsetzung und Aufbau der Arbeit}
% \label{sec:Zielsetzung}
% Was will ich mit der Arbeit erreichen
% Versuchsaufbau
% In kapitel 2 wird, Kapitel 3 das, 4 das, 5 das
% Hier fügen Sie den Text für die Zielsetzung ein.
% Entwicklungsziel: Konzeption und prototypische Implementierung...
% Untersuchungsziel: Methodische Untersuchung der Formalisierung...
% Überblick über den Aufbau der Arbeit...
%

\chapter{Einleitung}
\label{cap:Einleitung}

\section{Motivation und Relevanz}
\label{sec:Motivation}

Die technologische Entwicklung und die zunehmende Vielfalt
an Produkt- und Variantenvielfalt in der Fertigungsindustrie führt dazu, dass
industrielle Produktionsanlagen immer häufiger neu eingerichtet oder
umgerüstet werden müssen. In der Robotik erfordert dies die
Erstellung und Anpassung von Programmen, die Bewegungsabläufe,
Bearbeitungsschritte und Sicherheitsfunktionen eines Roboters
definieren.\vglcites{\citesingle{pine1993};
  \citesingle{elmaraghy2005}; \citesingle{wiendahl2007};
\citesingle{koren1999}; \citesingle{biggs2003}}
Traditionelle Programmiermethoden und -sprachen sind komplex,
herstellerspezifisch und setzen ein profundes Verständnis der
jeweiligen Kinematik und proprietären Sprachen
voraus.\vglcite[116]{lambrecht2011} Gleichzeitig halten generative
Sprachmodelle Einzug in die
Softwareentwicklung. Sie können aus natürlichsprachlichen
Beschreibungen ausführbaren Code erzeugen und versprechen so Potenzial, die
Hürden der Roboterprogrammierung zu
senken.\vglcites{\citesingle{salimpour2025}; \citesingle{brohan2023};
\citesingle{liang2023}} In der Praxis ist der
direkte Einsatz von Large Language Models (LLMs) zur Generierung von
Roboterprogrammcode im industriellen Kontext jedoch problematisch:
Fehlerhafte Bewegungssequenzen
oder fehlerhafte Prozesslogiken können zu Kollisionen, Schäden und
Produktionsausfällen führen.\vglcite[4]{bilancia2023} Daher ist eine
sorgfältige Validierung des
generierten Codes unerlässlich. Darüber hinaus stellt das Fehlen des
Verständnisses der physikalischen Welt eine signifikante Hürde für  die
Generierung von funktionalem Roboterprogrammcode durch LLMs
dar.\vglcite[1/psqq]{cohen2024}

\section{Zielsetzung und Aufbau der Arbeit}
\label{sec:Zielsetzung}
Ziel dieser Arbeit ist die Konzeption, Umsetzung und Evaluation eines
Frameworks zur simulationsbasierten Validierung von mit LLMs
generiertem Robotercode. Kernidee ist, durch die simulierte Ausführung von
Roboterprogrammcode in einer modellierten, der avisierten
Arbeitsumgebung des Roboters
entsprechenden Simulation auftretende unerwünschte Ereignisse
aufzuzeichnen und formalisiert zu dokumentieren. So kann Roboterprogrammcode
getestet werden und fehlerhaftes Verhalten wie falsche Prozessabfolgen,
Kollisionen, Singularitäten sowie Gelenkgeschwindigkeits- und
Beschleunigungsüberschreitungen frühzeitig erkannt und berichtigt werden.
Erkannte Fehlerereignisse sollen mit weiteren, der Analyse und Fehlersuche
behilflichen Daten angereichert werden.

Dazu wird in Kapitel~\ref{cap:Grundlagen} zunächst der Stand
der Technik zu
Robotersimulation, Offline-Programmierung und LLMs zusammengefasst und
herausgestellt, inwiefern die Notwendigkeit des Aufbaus eines solchen Frameworks
besteht und welche Rahmenbedingungen dazu zu beachten sind.

Auf dieser Basis erfolgt in Kapitel~\ref{cap:framework} die
Darstellung der Implementierung des Frameworks zur
simulationsbasierten Validierung. Dazu wird zunächst in
Abschnitt~\ref{sec:architektur_frameowork} die Architektur des zu
entwickelnden Frameworks
beschrieben, welches die Einbindung einer virtuellen Robotersteuerung
in der Entwicklungs- und Modellierungsumgebung Unity3D vorsieht.
Darauf aufbauend werden in Abschnitt~\ref{sec:implementierungMonitore}
vier verschiedene Module zur Fehlererkennung des Roboterverhaltens
implementiert. Dabei beschränkt sich diese Arbeit auf die Erkennung von falschen
Prozessfolgen innerhalb eines Roboterprogramms, der
Kollisionserkennung des Roboters mit seiner Umgebung, der
Singularitätserkennung sowie der Geschwindigkeits- und
Beschleunigungsüberschreitung von Robotergelenken.
In Abschnitt~\ref{sec:testumgebung} wird eine modellierte Testumgebung mit einem
realitätsnahen Pick-and-Place-Szenarios vorgestellt, innerhalb dessen die
Funktionalitäten des Frameworks getestet werden sollen.

Anschließend wird in Kapitel~\ref{cap:Ergebnisse} dargestellt,
inwiefern sich durch Veränderung des
Roboterprogrammcodes innerhalb der Testumgebung fehlerhaftes
Roboterverhalten provozieren lässt. Zusätzlich wird im Rahmen eines
Experteninterviews ergänzt, inwiefern der Ansatz in
der Praxis nutzbare Ergebnisse liefert.

Im Rahmen von Kapitel~\ref{cap:diskussion} werden die gewonnenen
Erkenntnisse diskutiert, Limitationen des entwickelten Frameworks
aufgezeigt und in
Zusammenhang mit einer Nutzung von LLMs zur Generierung und
Verbesserung von Roboterprogrammcode gebracht. Dazu werden die Stärken und
Schwächen des Ansatzes diskutiert sowie die Generalisierbarkeit und
Weiterentwicklungsmöglichkeiten aufgezeigt.

Die Arbeit schließt mit einer Zusammenfassung und Fazit in
Kapitel~\ref{cap:Fazit} ab.
