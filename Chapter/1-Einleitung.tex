% \chapter{Einleitung}
% \label{cap:Einleitung}

% \section{Motivation und Relevanz}
% \label{sec:Motivation}
% Hier fügen Sie den Text für die Motivation ein.
% Warum lohnt es sich mit dem Thema zu beschäftigen
% Wie relevant ist es in Forschung sowie Praxis

% \section{Zielsetzung und Aufbau der Arbeit}
% \label{sec:Zielsetzung}
% Was will ich mit der Arbeit erreichen
% Versuchsaufbau
% In kapitel 2 wird, Kapitel 3 das, 4 das, 5 das
% Hier fügen Sie den Text für die Zielsetzung ein.
% Entwicklungsziel: Konzeption und prototypische Implementierung...
% Untersuchungsziel: Methodische Untersuchung der Formalisierung...
% Überblick über den Aufbau der Arbeit...
%

\chapter{Einleitung}
\label{cap:Einleitung}

\section{Motivation und Relevanz}
\label{sec:Motivation}

Die technologische Entwicklung und die zunehmende Vielfalt
an Produkt- und Variantenvielfalt in der Fertigungsindustrie führt dazu, dass
industrielle Produktionsanlagen immer häufiger neu eingerichtet oder
umgerüstet werden müssen. In der Robotik erfordert dies die
Erstellung und Anpassung von Programmen, die Bewegungsabläufe,
Bearbeitungsschritte und Sicherheitsfunktionen eines Roboters
definieren.\vglcites{\citesingle{pine1993};
  \citesingle{elmaraghy2005}; \citesingle{wiendahl2007};
\citesingle{koren1999}; \citesingle{biggs2003}}
Traditionelle Programmiermethoden sind komplex,
herstellerspezifisch und setzen ein profundes Verständnis der
jeweiligen Kinematik und proprietären Sprachen
voraus.\vglcite[116]{lambrecht2011} Gleichzeitig halten generative
Sprachmodelle Einzug in die
Softwareentwicklung. Sie können aus natürlichsprachlichen
Beschreibungen ausführbaren Code erzeugen und versprechen so Potenzial, die
Hürden der Roboterprogrammierung zu
senken.\vglcites{\citesingle{salimpour2025}; \citesingle{brohan2023};
\citesingle{liang2023}} In der Praxis ist der
direkte Einsatz von Large Language Models (LLMs) zur Generierung von
Roboterprogrammcode im industriellen Kontext jedoch riskant:
Fehlerhafte Bewegungssequenzen
oder fehlerhafte Prozesslogiken können zu Kollisionen, Schäden und
Produktionsausfällen führen.\vglcite[4]{bilancia2023} Daher ist eine
sorgfältige Validierung des
generierten Codes unerlässlich. Darüber hinaus stellt die
Generierung von Roboterprogrammcode durch LLMs eine signifikante Hürde durch das
Fehlen des Verständnisses der physikalischen Welt
dar.\vglcite[1/psqq]{cohen2024}

\section{Zielsetzung und Aufbau der Arbeit}
\label{sec:Zielsetzung}
Ziel dieser Arbeit ist die Konzeption, Umsetzung und Evaluation eines
Frameworks zur simulationsbasierten Validierung von mit LLMs
generiertem Robotercode. Kernidee ist, durch die simulierte Ausführung von
Roboterprogrammcode in einer modellierten, der avisierten
Arbeitsumgebung des Roboters
entsprechenden Simulation auftretende unerwünschte Ereignisse
aufzuzeichnen und formalisiert zu dokumentieren. So kann Roboterprogrammcode
getestet werden und fehlerhaftes Verhalten wie falsche Prozessabfolgen,
Kollisionen, Singularitäten sowie Gelenkgeschwindigkeits- und
Beschleunigungsüberschreitungen frühzeitig erkannt und berichtigt werden.
Erkannte Fehlerereignisse sollen mit weiteren, der Analyse und Fehlersuche
behilflichen Daten anreichert werden.

Dazu wird in Kapitel~\ref{cap:Grundlagen}, wird zunächst der Stand
der Technik zu
Robotersimulation, Offline‑Programmierung und LLMs zusammengefasst und
herausgestellt, inwiefern die Notwendigkeit der Aufbau eines solches Frameworks
besteht und welche Rahmenbedingungen dazu zu beachten sind.
Darauf aufbauend wird in Kapitel~\ref{sec:framework} anschließend die
Architektur des zu
entwickelnden Frameworks beschrieben, welches die Einbindung einer
virtuellen Robotersteuerung in der Entwicklungs- und
Modellierungsumgebung Unity3D vorsieht.

Weiterführend werden vier verschiedene Module zur Analyse des Roboterverhaltens
implementiert werden. Dabei beschränkt sich diese Arbeit auf die Erkennung von
falschen Prozessfolgen innerhalb eines Roboterprogramms, der Kollisionserkennung
des Roboters mit seiner Umgebung, der Singularitätserkennung sowie der
Geschwindigkeits- und Beschleunigungsüberschreitung von Robotergelenken.

In einem realitätsnahen Beispielszenario einer Roboterzelle werden die
Funktionalitäten des Frameworks getestet und durch spezifische
Szenarien fehlerhaftes Roboterverhalten zu provozieren und die Detektion dessen
durch das Framework zu
verifizieren. Ein Experteninterview ergänzt die Evaluation und
reflektiert die Eignung des Ansatzes aus praktischer Sicht.

Abschließend werden die gewonnenen Erkenntnisse diskutiert, Limitationen
des Prototyps aufgezeigt und in Zusammenhang mit einer Nutzung von LLMs zur
Generierung und Verbesserung von Roboterprogrammcode gebracht.
