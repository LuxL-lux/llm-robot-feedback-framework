\chapter{Fazit und Ausblick}
\label{cap:Fazit}

Das entwickelte Framework demonstriert die Machbarkeit eines modularen,
eventgetriebenen Ansatzes für sicherheitsrelevante Überwachungsaufgaben in der
Roboterprogrammierung. Die Module erfüllten ihre Kernfunktionen, zugleich wurden
klare Erweiterungspfade sichtbar. Das Experteninterview mit Daniel Syniawa
bestätigte Relevanz und Nutzen des Ansatzes und unterstrich die praktischen
Hürden einer stark proprietären Werkzeuglandschaft. Insgesamt stellt die Arbeit
einen belastbaren Proof-of-Concept dar, der durch quantitative Evaluationen und
architektonische Erweiterungen in Richtung eines praxistauglichen Systems
weitergeführt werden kann.

Aus den Diskussionen ergeben sich zwei unmittelbare Blickrichtungen:
Erstens eine quantitative Evaluation, die die Leistungsfähigkeit des
Frameworks systematisch gegen manuelle Verfahren stellt. Mögliche Metriken
umfassen etwa Erkennungszeit, Zuverlässigkeit (Recall) der Detektion und
Fehlalarmrate; für die Dynamiküberwachung sind zudem akzeptable Toleranzbänder
zu definieren und transparent zu begründen. Zweitens ergibt sich eine
architektonische Weiterentwicklung: engere Kopplung an
Controllerschnittstellen, Erweiterung auf weitere Herstelleradapter sowie die
Prüfung echtzeitnaher Ausführungsumgebungen.

Langfristig eröffnet die strukturierte JSON-Ausgabe eine Brücke zu
automatisierten Analyse- und Korrekturprozessen. Eine naheliegende Linie ist die
Einbindung von LLMs, die auf Basis von Logdaten und
Layoutinformationen Roboterprogramme analysieren und Verbesserungsvorschläge
generieren. In einem nächsten Schritt ließe sich dies als MCP-Server
konzipieren, der Rückmeldungen zyklisch in das Framework einspeist und damit den
Bedarf an tiefem Expertenwissen reduziert, ohne den Sicherheitsanspruch zu
unterlaufen.
