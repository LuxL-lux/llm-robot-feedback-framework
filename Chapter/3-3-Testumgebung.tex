\subsection{Aufbau der Roboterzelle}
\label{sec:aufbauTestszenario}

\subsubsection{Arbeitsraum-Layout}

\begin{figure}[H]
	\centering
	\includegraphics[width=10cm]{Figures/Roboterzelle.png}
	\caption{Arbeitsraum-Layout des Roboters in der Draufsicht mit Roboter in
		Home-Position}
	\label{figure:arbeitsraum}
\end{figure}

Die Roboterzelle ist im Rahmen dieses Testszenarios als isolierter Arbeitsraum
mit mehreren Hindernisse, Begrenzungen und 3 semantischen Stationen aufgebaut.
Die vom Roboter zu bewältigenden Aufgaben beschränken sich dabei auf ein
Pick-and-Place-Szenario von Zylinderköpfen, dargstellent in Abbildung
\ref{figure:arbeitsraum}
\noindent
Das hierfür erstellte Layout sieht den Roboter in Zentrum dem Arbeitsraumes auf
dem Boden stehend vor. Hindernisse, Begrenzungen und Stationen sind kreisförmig
um den Roboter angeordnet. In Abbildung
\ref{figure:arbeitsraum}
blau markiert, befindet sich eine Säule als künstliches Hindernis. Links und rechts unten vom Mittelpunkt der Zelle aus befinden sich
zwei Regale als Teilelager. Oben befindet sich eine 5-Achs-Fräsmaschine mit
einer offenen Teileablage durch Entfernung der Schutztür zu
Vereinfachungszwecken dieses Szenarios. Der Arbeitsraum ist begrenzt durch
halbdurchsichtige Wände. Der Prozessablauf sieht vor, dass der Roboter
Zylinderköpfe aus dem linken Regal aufnimmt, diese in der Maschine ablegt, in
einer Warteposition auf die Beendigung der Bearbeitung wartet und anschliessend
das bearbeitete Teil aufnimmt, um es im rechten Regal abzulegen. Ein
beispielhafter Bearbeitungsprozess mit der 5-Achs-Fräsmaschine ist hier das Bohren und Schneiden von Gewinden
in die Zylinderköpfe.

Der Aufbau und die Anordnung des Arbeitsraums ergibt sich aus den funktionalen
Anforderungen, die das Framework abdecken soll. Durch den relativ begrenzten
Raum müssen Bewegungspfade des Roboter an Hindernissen wie der Säule vorbei
geplant werden und etwaige durch den Roboter gegriffene Werkstücke
mitberechnet werden. Weiterführend verringern die Regale den Bewegungsraum für
den Roboter zur Umorienterung vor und nach dem Aufnehmen und Ablegen der Teile,
was zu einer erhöhten Wahrscheinlichkeit von Singularitäten führt. Die 3
semantischen Stationen sorgen dafür, dass sich hier Fehler im Prozessablauf
abbilden lassen. Die Achsgeschwindkeiten lassen sich in diesem Szenario
ebenfalls beliebig variieren. Somit lassen sich alle funktionalen Anforderungen
des Frameworks innerhalb dieses vereinfachten Szenarios abbilden.

\subsubsection{Zylinderköpfe als Werkstücke}

\begin{figure}[H]
	\centering
	\includegraphics[width=\linewidth]{Figures/CyclinderHead-1.png}
	\caption{Technische Zeichung des verwendeten Zylinderkopfes, dargstellt
		mithilfe von Autodesk Fusion 360}
	\label{figure:cylinderhead}
\end{figure}

Als Werkstücke werden hier Zylinderköpfe für V6-Motoren gewählt. Das Werkstück
bringt ein mit den Spezifikationen des ABB IRB 6700 übereinstimmendes Gewicht
mit und ist an den Seitenflächen durch einen Parallelgreifer gut greifbar. Die
Abmasse, des Zylinderkopfes sind Abbildung \ref{figure:cylinderhead}
zu entnehmen.

\subsubsection{Greiferauslegung}

Um das Werkstück sicher handhaben zu können, wird zusätzlich zum Roboter auch
End-Effektor benötigt, welcher das Werkstück sicher durch den Arbeitsraum
bwwegen kann. Für die Handhabung des Zylinderkopfs wurde speziell ein Parallelgreifer
ausgelegt, da die Entscheidung für einen geeigneten Greifer komplex ist und das
Anforderungsprofil kaum normierbar ist \vglcite[91]{hesse2011}.\\


\begin{figure}[H]
	\centering
	\includegraphics[width=\linewidth]{Figures/SchunkGreifer-1.png}
	\caption{Technische Zeichung des ausgelegten Parallelgreifers, dargstellt
		mithilfe von Autodesk Fusion 360}
	\label{figure:schunkGripper}
\end{figure}

\noindent Mithilfe eines Online-Konfigurationstools der Firma
SCHUNK\vglcite{schunk2025} lässt sich ein parametrisierter Greifer auslegen und
3D-CAD-Modelle in verschiedenen Formaten herunterladen. Hier wurde aufgrund der
Werkstückdimensionen ein Parallelgreifer der Baureihe PGN-Plus-P gewählt und die
Form der Greifbacken dem Werkstück entsprechend angepasst (siehe Abbildung
\ref{figure:schunkGripper}).
Das genaue Datenblatt des Greifers ist dem Anhang zu entnehmen.

\subsection{Flange als Visualisierungstool}

Im Rahmen dieses Frameworks wird das Unity-Package \textit{Flange} genutzt.
Implementiert durch GitHub-User \textit{Preliy} und frei verfügbar im Rahmen
einer BSD 3-Clause Lizenz, bietet ein spezialisiertes Framework für die
Robotersteuerung in Unity mit modularer Architektur, die Gelenksteuerung,
Kinematik, Koordinatentransformationen und Echtzeitüberwachung als separate
Komponenten organisiert. Das System unterstützt sowohl direkte
Gelenkmanipulation auf niedriger Ebene als auch kartesische Steuerung auf
höherer Abstraktionsebene, wodurch es für verschiedene Roboteranwendungen
einsetzbar ist.\vglcite{preliyflange2024} Im Kontext dieser Entwicklungsarbeit
wird Flange zur Implementierung und Visualisierung der
Denavit-Hartenberg-Parameter und damit einhergehender Achstransformationen
eingesetzt, da diese Notation als fundamentales Werkzeug der Robotik eine
systematische Beschreibung der Geometrie serieller Robotermechanismen ermöglicht
und somit die Anwendung etablierter algorithmischer Verfahren für kinematische
Berechnungen, Jacobi-Matrizen sowie Bewegungsplanung
unterstützt.\vglcite[590]{corke2007}\\

\noindent
Flange ermöglicht eine direkte Konfiguration des Roboters über \textit{Frame}
und \textit{JointTransformation} Scripts. Ein Frame definiert dabei die
Denavit-Hartenberg-Parameter für einen Teil der kinematischen Kette, eine
JointTransformation die Paramter des Gelenks, also möglicher maximaler Ausschlag
in positive und negative Achsrotationsrichtung sowie maximale Geschwindigkeit
und Beschleunigung. Diese Funktion werden im Rahmen dieser Implentierung
genutzt, um den zu simulierenden Roboter theoretisch nachvollziehbar im Raum
bewegen zu können.

\subsection{Modellierung in Unity}
Der Arbeitsraum wurde der vorangegangenen Darstellung entsprechend in Unity
modelliert. Dabei wurden die von Unity standardmässig gewählten Einstellungen
verwendet, sowohl bei der Modellierung als auch im Kontext der
Physik-Engine. Jegliche Hindernisse wurden dabei mit Collidern versehen und dem
Tag texttt{Obstacle} und respektive \texttt{Machine} für die Maschine, um diese
bei der späteren Auswertung als Hindernisse kenntlich zu machen. Weiterführend
wurden den einzelnen Stationen (linkes und rechtes Regal, Maschine) das Script
\texttt{Station.cs} angehängt und dort im Sinne des bereits beschriebenen
Arbeitsablaufs ein steigender Index zugewiesen. Analog dazu wird allen
Zylinderköpf-Ojekte das Script \texttt{Part.cs} zugewiesen, welche diese als
Werkstücke identifiziert und mit dem durch Drag-and-Drop im Unity Inspector die
Reihenfolge der einzelnen Stationen definiert wird, welche das Teil durchlaufen
muss.\\

\noindent
Alle Monitore wurden in einer einheitlichen Unity-Testumgebung ausgeführt.
Die Simulationen basieren auf einem digitalen Zwilling des Roboters, der
kontinuierlich seine Gelenkwinkel und Zustände an die Monitore übermittelt.
Die erkannten Ereignisse werden unmittelbar protokolliert und in eine
JSON-Logdatei geschrieben. Auf diese Weise sind die Ergebnisse der
verschiedenen Monitore konsistent dokumentiert und im Ergebniskapitel
direkt vergleichbar.
