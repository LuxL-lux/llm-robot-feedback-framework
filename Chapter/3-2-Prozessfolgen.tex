\subsection{Process Flow Monitor}
\label{section:prozessfolgen}
\subsubsection{Theoretische Grundlagen der Prozessflussüberwachung}
\noindent
Die Überwachung von Produktionsabläufen in der Robotik basiert auf der Validierung
deterministischer Werkstückpfade durch diskrete Fertigungsstationen. In der
Literatur werden solche Systeme als Discrete Event Systems (DES) modelliert, bei
denen Zustandsübergänge durch definierte Ereignisse ausgelöst werden
\vglcite{cassandras2008introduction}. Die formale Verifikation von Prozesssequenzen
erfolgt typischerweise durch Petri-Netze oder endliche Automaten
\vglcite{murata1989petri}.

\subsubsection{Praktisches Vorgehen}
\noindent
Das System modelliert den Fertigungsprozess durch zwei zentrale Domänenobjekte.
Die \texttt{Part}-Klasse repräsentiert Werkstücke mit eindeutiger ID, Name und
Typ. Jedes Werkstück speichert seine erforderliche Bearbeitungssequenz als
Array von Station-Referenzen und verfolgt seinen aktuellen Fortschritt durch
einen Sequenzindex. Die \texttt{Station}-Klasse definiert Bearbeitungsbereiche
durch Trigger-Collider an den physischen Positionen der Arbeitsstationen. Jede
Station besitzt einen eindeutigen Namen und numerischen Index zur Sortierung.

\noindent
Die Prozesspfade werden deklarativ im Unity-Inspector durch Zuweisung der
Station-Referenzen definiert. Diese Konfiguration ermöglicht die flexible
Modellierung verschiedener Fertigungsabläufe ohne Programmänderungen. Zur
Laufzeit validiert das System automatisch die Einhaltung der definierten
Sequenzen und protokolliert jeden Stationsbesuch mit Zeitstempel für spätere
Analysen.

\noindent
Die Schwierigkeit liegt hier in der korrekten Interpretation von
Trigger-Events unter Berücksichtigung der dynamischen Objekthierarchie. Wenn ein
Werkstück gegriffen wird, wechselt es in Unity vom Weltkoordinatensystem in das
lokale Koordinatensystem des Greifers. Diese Hierarchieänderung erzeugt
scheinbare Ein- und Austritts-Events an Stationen, die nicht als
Prozessübergänge interpretiert werden dürfen.

\noindent
Der Monitor implementiert daher eine Grip-State-Awareness, die vier 
relevante Zustandsübergänge unterscheidet: Ablegen (Validierung der Zielstation),
Aufnehmen (Protokollierung der Quellstation), Transport (wird ignoriert) und
ungegriffene Bewegungen (Fehlererkennung). Diese Unterscheidung erfolgt durch
hierarchische Analyse der Transform-Komponente - ein Werkstück gilt als
gegriffen, wenn es Kind-Objekt des Roboters ist.

\noindent
Die Implementierung nutzt ein Layer-System zur Ereignisfilterung. Werkstücke
operieren auf Layer 30 (Parts), Stationen auf Layer 31 (ProcessFlow). Diese
Trennung ermöglicht selektive Trigger-Erkennung ohne Interferenz mit der
physikalischen Kollisionserkennung des Roboters.

\subsubsection{Konkrete Implementierung}
\noindent
Der \texttt{ProcessFlowMonitor} koordiniert die Überwachung durch Subscription
auf Station-Events. Bei jedem Trigger-Event korreliert er den Grip-State mit
der Werkstückbewegung und validiert Transitionen gegen die definierte Sequenz.
Der Monitor klassifiziert drei Verletzungstypen: \texttt{WrongSequence} für
nicht-sequenzielle Bewegungen, \texttt{SkippedStation} für übersprungene
Stationen und \texttt{UnknownStation} für undefinierte Ziele.

\noindent
Bei Verletzungserkennung generiert der Monitor ein \texttt{SafetyEvent} mit
vollständigen Metadaten. Die Schweregrade sind konfigurierbar - standardmäßig
werden falsche Sequenzen als kritisch und übersprungene Stationen als Warnungen
klassifiziert. Ein Cooldown-Mechanismus verhindert mehrfache Meldungen
identischer Verletzungen. Das Event wird an den \texttt{RobotSafetyManager}
weitergeleitet, der es mit dem aktuellen Roboterzustand anreichert und gemäß
der konfigurierten Logging-Strategie verarbeitet.
