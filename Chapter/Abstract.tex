\addchap{Abstract}
The programming of industrial robots traditionally requires detailed
knowledge of vendor-specific programming languages as well as a
profound understanding of kinematics. With the availability of
generative language models (LLMs), the possibility arises to generate
robot programs from natural language descriptions. However, directly
applying such approaches in industrial environments entails
considerable risks, as faulty motion sequences or process logic may
lead to collisions, production downtime, or safety-critical
situations. Simulation-based validation therefore represents a
necessary intermediate step to virtually test generated code and
reduce risks before transferring it to real systems.

Within the scope of this thesis, a framework for simulation-based
validation of LLM-generated robot code is developed, implemented, and
evaluated. The framework is realized in Unity3D using an ABB
articulated robot as test system. It follows a modular, event-driven
architecture and integrates monitoring mechanisms that validate
process flows, detect collisions, identify kinematic singularities,
and monitor joint positions, velocities, and accelerations. All
detected deviations are systematically recorded and enriched with
temporal and contextual information to enable subsequent analysis.

The functionality was examined in a realistic pick-and-place
scenario, in which faulty sequences and critical robot states were
deliberately provoked. In addition, a qualitative reflection was
carried out in the form of an expert interview. The results
demonstrate that process deviations, collisions, singularities, and
limit violations can be reliably detected and reproducibly
documented. Thus, this work contributes to the safe integration of
LLMs into industrial robotics by providing a methodological basis for
the validation of generated code and by outlining a perspective for
feeding simulation results back into language models.
