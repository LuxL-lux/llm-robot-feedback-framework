\chapter{Grundlagen und Stand der Technik}
\label{cap:Grundlagen}

\section{Grundlagen der Robotersimulation}
\label{sec:Grundlagen_Robotersimulation}
% Einführung in relevante Parameter...
% Überblick über Simulationslandschaft und Vorteile von Unity...

\section{Large Language Models (LLMs)}
\label{sec:Grundlagen_LLMs}
% Grundlegende Erklärung der Funktionsweise...
% Abgrenzung und Definition von KI..
\section{Stand der Technik: LLMs in der Robotik}
\label{sec:Stand_LLMs_Robotik}
% Aktuelle Forschungsansätze (Brain-Body, Gemini Robotics)...
% Herausforderungen (Grounding, Sicherheit)...
% Bestehende Frameworks...
\section{Stand der Technik: Robotersimulation}
\label{sec:Stand_Robotersimulation}
Bezugnahme auf aktuelle Techniken in der Robotersimulation -> Lehrbücher etc. bieten hier eine Grundlagen
Aktualität muss nicht gewähleistet sein da relativ statisch
Auf jeden Fall Abgrenzung warum ich Unity verwende: Open-Source von Unity
