\chapter{Grundlagen und Stand der Technik} \label{cap:Grundlagen}

\section{Grundlagen der Robotersimulation}
\label{sec:Grundlagen_Robotersimulation}
% Einführung in relevante Parameter... Überblick über Simulationslandschaft und
% Vorteile von Unity...

\section{Large Language Models (LLMs)} \label{sec:Grundlagen_LLMs}
% Grundlegende Erklärung der Funktionsweise... Abgrenzung und Definition von
% KI..
\section{Stand der Technik: LLMs in der Robotik} \label{sec:Stand_LLMs_Robotik}
% Aktuelle Forschungsansätze (Brain-Body, Gemini Robotics)... Herausforderungen
% (Grounding, Sicherheit)... Bestehende Frameworks...
\section{Stand der Technik: Robotersimulation}
\label{sec:Stand_Robotersimulation} Roboterprogrammierung kann als die
Programmierung von industriellen Manipulatoren verstanden werden, die sich durch
ihre programmierbaren und anpassungsfähigen Eigenschaften von anderen Maschinen
abheben. Roboterprogramme enthalten präzise Anweisungen und Spezifikationen für
die Bewegung des Roboters. Im Kern ist die Roboterprogrammierung ein
Steuerungsproblem, das neben den allgemeinen Aspekten der Computerprogrammierung
auch roboterspezifische Herausforderungen umfasst. \vglcite[1\psqq]{nilsson1996}
Ein entscheidender Vorteil der Roboterprogrammierung im Gegensatz zur
herkömmlichen Steuertechnik ist dabei die Möglichkeit, einen realen Prozess
mittels Programmcode abbilden und verändern zu können und somit die Effizienz
und Anpassungsfähigkeit eines industriellen Prozessen zu erhöhen.


\subsection{Offline-Programmierung und Robotersimulationsprogramme} Da Anlagen
oft dauerhaft in Betrieb sind un die Notwendigkeit von Prozessveränderungen bei
der Fertigung von Produkten unabdingbar ist, wird eine Möglichkeit benötigt,
industrielle Anlagen schon bevor der Inbetriebnahme, Umfunktionierung oder
Verbesserung testen zu können. Auch bei Robotern ist dies, unter anderem
aufgrund hoher Kosten und Sicherheitsrisiken bei fehlerhafter Konifguration,
unabdingbar. Die \textbf{Offline-Programmierung (OLP)} ermöglicht die
Entwicklung eines Steuerungsprogramms, ohne dass ein physischer Roboter
erforderlich ist. Ein wesentlicher Vorteil ist die Möglichkeit, Programme in
einer virtuellen Umgebung zu erstellen und zu simulieren. Die Methode nutzt
3D-CAD-Modelle, um den Körper des Roboters möglichst originalgetreu
darzustellen. Ziel der Offline-Programmierung ist es, den visuellen Prozess
schon vorher evaluieren zu können und Probleme im Ablauf früher zu erkennnen.
\vglcite[62\psqq]{holubke2014}


\subsection{Limitierende Faktoren proprietärer Systeme} Traditionelle
Robotersteuerungssysteme sind oft durch eine \textbf{proprietäre Hardware- und
	Softwarearchitektur} begrenzt. \vglcite[1]{bilancia2023} Dies führt dazu, dass
Anwender für jede Robotermarke eine spezifische Programmiersprache mit
unterschiedlichen Befehlen und Funktionen erlernen müssen. Solche Systeme
verursachen \textbf{Zeitverlust} in der Produktion, da jede Änderung eine
Unterbrechung erfordert, sowie \textbf{Codeduplikation}, da Programme nicht
zwischen Robotermarken ausgetauscht werden können \vglcite[3]{bilancia2023}.Um
diese Beschränkungen zu überwinden, ist eine akkurate Modellierung der
physischen Welt, in welcher der Roboter agieren soll, in der Simulation
unabdingbar.

\subsection{Physics-Engines}
Dies führt zum Einsatz von {Physics-Engines}. Um eine
realistische Simulation zu gewährleisten, sind diese unerlässlich. Sie
modellieren dynamische Interaktionen wie Kollisionen, Schwerkraft und Reibung,
was für die präzise Nachbildung des Roboterverhaltens entscheidend ist. Obwohl
die Genauigkeit dieser Engines als nicht perfekt angesehen wird, da sie die
reale Welt nicht exakt abbilden, sind sie für die Forschung in den Bereichen
Deep Learning und Digital Twins von entscheidender Bedeutung
\vglcite[1/psqq]{audonnet2022}. Die Wahl der Physics-Engine ist ein kritischer
Faktor und beeinflusst die Stabilität und Wiederholbarkeit der Simulation.
Häufig genutzte Engines sind PhysX, Bullet und ODE.
