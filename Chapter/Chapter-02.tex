%%%%%%%%%%%%%%%%%%%%%%%%%%%%%%%%%%%%%%%%%%%%%%%%%%%%%%%%%%%%%%%%%%%%%%%%%%%%%%%%%%%%%%%%%%%%%%%%%%%%%%%%%%%%%%%%%%%%%%%%%%%%%%%%%%%%%%%%%%%%%%%%%%%%%%%%%%%%%%%%%%%%%%%%%%%%%%%%%%%%%%%%%%%%%%%%%%%%%%%%%%%%%%%%%%%%%%%%%%%%%%%%%%%%%%%%%%%%%%%%%%%%%%%%
%%%&& Aufbau und Struktur einer fachwissenschaftlichen Arbeit %%%%%
%%%%%%%%%%%%%%%%%%%%%%%%%%%%%%%%%%%%%%%%%%%%%%%%%%%%%%%%%%%%%%%%%%%%%%%%%%%%%%%%%%%%%%%%%%%%%%%%%%%%%%%%%%%%%%%%%%%%%%%%%%%%%%%%%%%%%%%%%%%%%%%%%%%%%%%%%%%%%%%%%%%%%%%%%%%%%%%%%%%%%%%%%%%%%%%%%%%%%%%%%%%%%%%%%%%%%%%%%%%%%%%%%%%%%%%%%%%%%%%%%%%%%%%%

\chapter{Aufbau und Struktur einer fachwissenschaftlichen Arbeit}
\label{cap:Aufbau und Struktur einer fachwissenschaftlichen Arbeit}

%%%%%%%%%%%%%%%%%%%%%%%%%%%%%%%%%%%%%%%%%%%%%%%%%%%%%%%%%%%%%%%%%%%%%%%%%%%%%%%%%%%%%%%%%%%%%%%%%%%%%%%%%%%%%%%%%%%%%%%%%%%%%%%%%%%%%%%%%%%%%%%%%%%%%%%%%%%%%%%%%%%%%%%%%%%%%%%%%%%%%%%%%%%%%%%%%%%%%%%%%%%%%%%%%%%%%%%%%%%%%%%%%%%%%%%%%%%%%%%%%%%%%%%%
\section{Grundstruktur}
\label{sec:Grundstruktur}
%%%%%%%%%%%%%%%%%%%%%%%%%%%%%%%%%%%%%%%%%%%%%%%%%%%%%%%%%%%%%%%%%%%%%%%%%%%%%%%%%%%%%%%%%%%%%%%%%%%%%%%%%%%%%%%%%%%%%%%%%%%%%%%%%%%%%%%%%%%%%%%%%%%%%%%%%%%%%%%%%%%%%%%%%%%%%%%%%%%%%%%%%%%%%%%%%%%%%%%%%%%%%%%%%%%%%%%%%%%%%%%%%%%%%%%%%%%%%%%%%%%%%%%%

Die Grundstruktur einer fachwissenschaftlichen Arbeit ergibt sich aus den Teilen "`\textbf{Einleitung}"', "`\textbf{Hauptteil}"' und "`\textbf{Schluss}"'.

In der \textbf{Einleitung} erfolgt zunächst die Darlegung der Problemstellung der Arbeit, wobei der thematische Bezugsrahmen (z.\,B. Einbettung in Forschungsprojekt, Fachgebiet etc.) erläutert werden sollte. Entscheidend ist hierbei, dass für den Leser deutlich wird, welche Fragestellung bearbeitet wird. Anschließend kann kurz dargelegt werden, in welcher Richtung die Arbeit Antworten liefern wird und welche Ergebnisse erwartet werden (und welche nicht – Abgrenzung des Themas). Die Einleitung (das einleitende Kapitel) sollte knapp gehalten werden und in ihrem Umfang in Relation zur gesamten Arbeiten stehen.

Im \textbf{Hauptteil}, dem Kernstück der wissenschaftlichen Arbeit, erfolgt die thematische Behandlung der jeweiligen Aufgabenstellung (Problematik) auf Basis des Standes der Technik. Gegliedert und in systematischer Reihenfolge (\emph{"`roter Faden"'}) werden hier die theoretischen Ansätze sowie das methodische Vorgehen erarbeitet, angewendet und mit den ermittelten Ergebnissen präsentiert. Ausgehend von der in der Einleitung formulierten Frage- oder Problemstellung werden diese in argumentativ-beweisender Form dargelegt.

Der \textbf{Schlussteil} einer wissenschaftlichen Arbeit fasst das Ergebnis der Arbeit in knapper Form zusammen und benennt offene weiterführende Fragestellungen. Eine kurze persönliche Stellungnahme kann an dieser Stelle abgegeben werden.

Grundsätzlich sollte gelten: \emph{Einfach einfach schreiben!} Diesbezüglich sollten zusammenhängende Texte nicht zu lang gestaltet werden, sondern nach Möglichkeit durch Absätze voneinander getrennt und ggf. mit zusätzlichen Überschriften ergänzt werden. Dies ermöglicht dem Leser ein einfacheres Verständnis komplexer Zusammenhänge.

Um einen logischen Aufbau in der Struktur der fachwissenschaftlichen Arbeit zu erzielen und inhaltliche Vollständigkeit zu gewährleisten, sollten folgende Fragen in der fachwissenschaftlichen Arbeit beantwortet werden:
\begin{itemize}
	\item \textbf{Frage- und Problemstellung:} Worum geht es?
	\item \textbf{Relevanz:} Warum ist es wichtig/interessant dieser Frage nachzugehen?
	\item \textbf{Stand der Forschung und Technik:} Wer hat was wie darüber herausgefunden?
	\item \textbf{Problemstellung:} Was hat man bisher nicht herausgefunden/untersucht?
	\item \textbf{Experimenteller/methodischer/analytischer Ansatz:} Wie bin ich bei der Bearbeitung der Problemstellung vorgegangen?
	\item \textbf{Ergebnisse:} Was habe ich dabei herausgefunden?
	\item \textbf{Diskussion / kritische Zusammenfassung:} Wie sind meine Ergebnisse hinsichtlich der bisherigen Forschung zu bewerten?
	\item \textbf{Ausblick:} Welche neuen Forschungsfragen ergeben sich?
\end{itemize}

%%%%%%%%%%%%%%%%%%%%%%%%%%%%%%%%%%%%%%%%%%%%%%%%%%%%%%%%%%%%%%%%%%%%%%%%%%%%%%%%%%%%%%%%%%%%%%%%%%%%%%%%%%%%%%%%%%%%%%%%%%%%%%%%%%%%%%%%%%%%%%%%%%%%%%%%%%%%%%%%%%%%%%%%%%%%%%%%%%%%%%%%%%%%%%%%%%%%%%%%%%%%%%%%%%%%%%%%%%%%%%%%%%%%%%%%%%%%%%%%%%%%%%%%
\section{Spezielle Formalia für fachwissenschaftliche Arbeiten}
\label{sec:Spezielle Formalia für fachwissenschaftliche Arbeiten}
%%%%%%%%%%%%%%%%%%%%%%%%%%%%%%%%%%%%%%%%%%%%%%%%%%%%%%%%%%%%%%%%%%%%%%%%%%%%%%%%%%%%%%%%%%%%%%%%%%%%%%%%%%%%%%%%%%%%%%%%%%%%%%%%%%%%%%%%%%%%%%%%%%%%%%%%%%%%%%%%%%%%%%%%%%%%%%%%%%%%%%%%%%%%%%%%%%%%%%%%%%%%%%%%%%%%%%%%%%%%%%%%%%%%%%%%%%%%%%%%%%%%%%%%

Die fachwissenschaftliche Abschlussarbeit ist in einer \textbf{festen Bindung} (vgl. Abschnitt \ref{sec:Abgabe der Arbeit}) und in \textbf{zweifacher Ausfertigung unterschrieben beim Prüfungsamt} einzureichen bzw. abstempeln zu lassen und anschließend bei dem zuständigen Betreuer respektive im Sekretariat des Lehrstuhls abzugeben.
Wird der Abgabetermin nicht eingehalten, gilt die Arbeit als nicht bestanden.
Außerdem sind der fachwissenschaftlichen Arbeit eine \textbf{eidesstattliche Erklärung} mit Unterschrift (siehe Abschnitt \ref{sub:Eidesstattliche Erklärung}) anzufügen. Zudem ist vor Anmeldung der Arbeit im Prüfungsamt und im LPS-Sekretariat in Absprache mit dem Betreuer eine \textbf{Verpflichtungserklärung} zu unterschreiben.

%%%%%%%%%%%%%%%%%%%%%%%%%%%%%%%%%%%%%%%%%%%%%%%%%%%%%%%%%%%%%%%%%%%%%%%%%%%%%%%%%%%%%%%%%%%%%%%%%%%%%%%%%%%%%%%%%%%%%%%%%%%%%%%%%%%%%%%%%%%%%%%%%%%%%%%%%%%%%%%%%%%%%%%%%%%%%%%%%%%%%%%%%%%%%%%%%%%%%%%%%%%%%%%%%%%%%%%%%%%%%%%%%%%%%%%%%%%%%%%%%%%%%%%%
\section{Formatierung}
\label{sec:Formatierung}
%%%%%%%%%%%%%%%%%%%%%%%%%%%%%%%%%%%%%%%%%%%%%%%%%%%%%%%%%%%%%%%%%%%%%%%%%%%%%%%%%%%%%%%%%%%%%%%%%%%%%%%%%%%%%%%%%%%%%%%%%%%%%%%%%%%%%%%%%%%%%%%%%%%%%%%%%%%%%%%%%%%%%%%%%%%%%%%%%%%%%%%%%%%%%%%%%%%%%%%%%%%%%%%%%%%%%%%%%%%%%%%%%%%%%%%%%%%%%%%%%%%%%%%%

Die Seitenränder betragen oben 30\,mm und unten 40\,mm, innen 32\,mm und außen 24\,mm. Ein neues Kapitel beginnt stets auf einer neuen Seite.
Grundsätzlich ist dieser Leitfaden als Formatvorlage für studentische Arbeiten des Lehrstuhls für Produktionssysteme zu verwenden.


%%%%%%%%%%%%%%%%%%%%%%%%%%%%%%%%%%%%%%%%%%%%%%%%%%%%%%%%%%%%%%%%%%%%%%%%%%%%%%%%%%%%%%%%%%%%%%%%%%%%%%%%%%%%%%%%%%%%%%%%%%%%%%%%%%%%%%%%%%%%%%%%%%%%%%%%%%%%%%%%%%%%%%%%%%%%%%%%%%%%%%%%%%%%%%%%%%%%%%%%%%%%%%%%%%%%%%%%%%%%%%%%%%%%%%%%%%%%%%%%%%%%%%%%
\section{Gliederung}
\label{sec:Gliederung}
%%%%%%%%%%%%%%%%%%%%%%%%%%%%%%%%%%%%%%%%%%%%%%%%%%%%%%%%%%%%%%%%%%%%%%%%%%%%%%%%%%%%%%%%%%%%%%%%%%%%%%%%%%%%%%%%%%%%%%%%%%%%%%%%%%%%%%%%%%%%%%%%%%%%%%%%%%%%%%%%%%%%%%%%%%%%%%%%%%%%%%%%%%%%%%%%%%%%%%%%%%%%%%%%%%%%%%%%%%%%%%%%%%%%%%%%%%%%%%%%%%%%%%%%

Der formale Aufbau einer fachwissenschaftlichen Arbeit sollte strukturiert sein und zum Beispiel wie folgt aussehen:
\\
\\
\noindent
\textbf{1. Einleitung} \\
1.1 Motivation (Allgemeine Ausgangssituation und Einstieg ins Thema) \\
1.2 Zielsetzung und Aufbau der Arbeit \\
1.3 $\ldots$ \\
\textbf{2. Grundlagen / Stand der Technik}  \\
\textbf{3. $\ldots$ }\\
$\vdots$ \\
\textbf{x. Zusammenfassung und Ausblick} \\
\textbf{Literaturverzeichnis} \\
\textbf{Anhang} \\

%%%%%%%%%%%%%%%%%%%%%%%%%%%%%%%%%%%%%%%%%%%%%%%%%%%%%%%%%%%%%%%%%%%%%%%%%%%%%%%%%%%%%%%%%%%%%%%%%%%%%%%%%%%%%%%%%%%%%%%%%%%%%%%%%%%%%%%%%%%%%%%%%%%%%%%%%%%%%%%%%%%%%%%%%%%%%%%%%%%%%%%%%%%%%%%%%%%%%%%%%%%%%%%%%%%%%%%%%%%%%%%%%%%%%%%%%%%%%%%%%%%%%%%%
\subsection{Inhaltsverzeichnis}
\label{sub:Inhaltsverzeichnis}

Eine Gliederung der gesamten Arbeit erfolgt durch das Inhaltsverzeichnis mit der Angabe von Seitenzahlen. Im Inhaltsverzeichnis erhöht ein Einzug nach rechts für jede weitere Untergliederung die Übersichtlichkeit. Die dritte Gliederungsebene sollte nicht überschritten werden (d.\,h. max. 1.1.1).
Das Inhaltsverzeichnis enthält alle im Text aufgeführten Überschriften und alle Untergliederungsebenen. Literaturverzeichnis und Anhang mit Tabellen werden in die Gliederung einbezogen, haben aber im Inhaltsverzeichnis keine Nummerierung. Die einzelnen Seiten des Anhangs müssen mit den fortlaufenden Seitenzahlen des Textes weiter durchnummeriert werden.
Das Vorwort, die Aufgabenstellung, die Kurzfassung, das Abbildungs- und Tabellenverzeichnis sowie die eidesstattliche Erklärung tauchen im Inhaltsverzeichnis nicht auf.

%%%%%%%%%%%%%%%%%%%%%%%%%%%%%%%%%%%%%%%%%%%%%%%%%%%%%%%%%%%%%%%%%%%%%%%%%%%%%%%%%%%%%%%%%%%%%%%%%%%%%%%%%%%%%%%%%%%%%%%%%%%%%%%%%%%%%%%%%%%%%%%%%%%%%%%%%%%%%%%%%%%%%%%%%%%%%%%%%%%%%%%%%%%%%%%%%%%%%%%%%%%%%%%%%%%%%%%%%%%%%%%%%%%%%%%%%%%%%%%%%%%%%%%%
\subsection{Literaturverzeichnis und Zitierkonventionen}
\label{sub:Literaturverzeichnis und Zitierkonventionen}

Der richtige Umgang mit Literatur ist wichtig für den Fortschritt der Arbeit und deren Qualität. "`Richtiges"' und selektives Lesen ist meistens nötig, denn nicht jedes Buch muss vollständig gelesen werden.

Wörtliche Zitate sollten generell vermieden werden. Nur in Ausnahmefällen, wie z.\,B. bei Definitionen, sind wörtliche Zitate zulässig. Diese Zitate werden dann zusätzlich immer in Kursivschrift und mit Anführungszeichen angeben.

Nützlich ist das \textbf{Anlegen einer Literaturdatei} von Anfang an, um später systematisch weiterarbeiten zu können. Es ist zu diesem Zweck von großer Bedeutung, die relevanten Daten zu den jeweiligen Literaturquellen zu notieren.
Die Recherche dient dazu, sich einen Überblick über das gewählte oder vorgegebene Themengebiet zu verschaffen.

Der Bearbeiter soll sachkundig werden, erkennen welche, und auf welche Weise, Fakten des Themas bisher in der Literatur abgehandelt wurden und sich so den Stand der Technik erarbeiten. Meist stellt sich beim Einlesen eine erste Grundordnung des Themas heraus, sodass die Gliederungserstellung erleichtert wird. Eine fundierte Analyse des Standes der Kenntnisse anhand von Literatur stellt auch für die spätere Bewertung Ihrer Arbeit eine wichtige Grundlage dar. Im Startpaket für Studenten, welches Sie von ihrem LPS-Betreuer bekommen, finden Sie Unterlagen zur Durchführung einer Literaturrecherche samt möglichen Datenbanken. Sprechen Sie Ihren Betreuer an, falls er Ihnen die Unterlagen nicht ausgehändigt hat.

Achten Sie auf die Qualität Ihrer Quellen! Das Internet bietet eine hohe Informationsflut. Die Informationen sind zwar vielfältig, aber oftmals auch unbrauchbar und sind somit für das wissenschaftliche Arbeiten nur im bestimmten Rahmen geeignet. Wikipedia gilt beispielsweise nicht als wissenschaftliche Online-Quelle!

Angaben im Literaturverzeichnis sollten so knapp wie möglich sein, andererseits eine schnelle und sichere Identifizierung der angegebenen Stellen ermöglichen.
Bei mehr als zwei Autoren wird die Abkürzung \emph{et al.} nach der Nennung des Erstautors verwendet. Wichtig ist es, die gewählte Zitiertechnik konsequent beizubehalten.
Essenziell für das Literaturverzeichnis sind die Angabe aller Autoren, Publikationsjahr, Titel der Zeitschrift oder des Buches sowie, bei Artikeln, die Seitenzahlen innerhalb der Zeitschrift/des Bandes.
Das Literaturverzeichnis ist alphabetisch geordnet, wobei immer nach dem Nachnamen des erstgenannten Autors sortiert wird.

Die standardmäßige Zitation in LaTeX erfolgt über den Befehl \verb+\cite{...}+. Beispielsweise erzeugt der Aufruf \verb+\cite{hufnagel:11}+ hier die folgende Ausgabe: \cite{hufnagel:11}
Soll zusätzlich der Autorname angegeben werden, erfolgt der Aufruf über \verb+\citet{hufnagel:11}+.
Die Ausgabe sieht dann wie folgt aus: \citet{hufnagel:11}

\textbf{Wichtiger Hinweis:}
Bei Verwendung des Literaturverzeichnisses \code{biblatex} muss im TeXnicCenter \code{biber.exe} anstelle von \code{bibtex.exe} ausgewählt (siehe \autoref{fig:Bibtex}).
Hierzu wird im TeXnicCenter unter \emph{Ausgabe > Ausgabeprofile definieren > (La)TeX > Pfad des BibTeX-Compilers} beispielsweise der folgende Pfad angegeben:
\begin{lstlisting}[style=bash]
C:/Program Files/MiKTeX 2.9/miktex/bin/x64/biber.exe
\end{lstlisting}

% Abbildung: Bibtex-Einstellungen
\begin{figure}[h] % t = top
	\centering
	\begin{overpic}[width=0.7\textwidth]
		{Figures/bibtex.png}
	\end{overpic}
	\caption{TeXnicCenter-Einstellungen für die Verwendung von \code{biblatex}}
	\label{fig:Bibtex}
\end{figure}


\subsubsection*{Digitale Dokumente}
\label{sub:Digitale Dokumente}

Wird ein Dokument aus dem Internet zitiert, muss geprüft werden, ob es sich um eine wissenschaftlich verwertbare Information handelt, d.\,h., ob es zitierfähig ist.
Wenn dies der Fall ist, muss das Dokument so vollständig zitiert werden, dass quasi auf die Angabe der URL verzichtet werden könnte.
Die Angabe wird bei digitalen Dokumenten genauso vorgenommen, wie bei gedruckten Dokumenten:
Autor/verantwortende Organisation, Titel, der Ablageort bzw. die Zeitschrift, Universität, Organisation, Behörde etc., auf deren Server das Dokument liegt und das Publikationsdatum.
Dann folgen Webadresse und Abfragedatum. Das Dokument sollte nicht zitiert werden, wenn nicht bekannt ist, wer das Dokument verantwortet oder keinerlei Quellenangaben zu finden sind.
Im Text wird nur Autor und Jahr zitiert (z.\,B. [Wenz, 1998]), nicht die URL.

%%%%%%%%%%%%%%%%%%%%%%%%%%%%%%%%%%%%%%%%%%%%%%%%%%%%%%%%%%%%%%%%%%%%%%%%%%%%%%%%%%%%%%%%%%%%%%%%%%%%%%%%%%%%%%%%%%%%%%%%%%%%%%%%%%%%%%%%%%%%%%%%%%%%%%%%%%%%%%%%%%%%%%%%%%%%%%%%%%%%%%%%%%%%%%%%%%%%%%%%%%%%%%%%%%%%%%%%%%%%%%%%%%%%%%%%%%%%%%%%%%%%%%%%
\subsection{Abbildungen und Tabellen}
\label{sub:Abbildungen und Tabellen}

Abbildungen und Tabellen sollten stets eine Titelangabe sowie bei übernommenen Abbildungen eine Quellenangabe enthalten. Zu jeder Abbildung ist ein Verweis im Text zu berücksichtigen. Die Schriftgröße innerhalb von Abbildungen sollte 10\,pt. nicht unterschreiten (auf Skalierung achten!). Die Qualität der Abbildungen sollte möglichst hoch sein. Dies ist durch das Erstellen eigener Abbildungen zu erreichen.

%%%%%%%%%%%%%%%%%%%%%%%%%%%%%%%%%%%%%%%%%%%%%%%%%%%%%%%%%%%%%%%%%%%%%%%%%%%%%%%%%%%%%%%%%%%%%%%%%%%%%%%%%%%%%%%%%%%%%%%%%%%%%%%%%%%%%%%%%%%%%%%%%%%%%%%%%%%%%%%%%%%%%%%%%%%%%%%%%%%%%%%%%%%%%%%%%%%%%%%%%%%%%%%%%%%%%%%%%%%%%%%%%%%%%%%%%%%%%%%%%%%%%%%%
\subsection{Eidesstattliche Erklärung}
\label{sub:Eidesstattliche Erklärung}

Am Ende der Arbeit muss die \textbf{eidesstattliche Erklärung} stehen. 

