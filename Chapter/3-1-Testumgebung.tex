\subsection{Flange als Visualisierungstool}

Im Rahmen dieses Frameworks wird das Unity-Package
\textit{Flange} genutzt. Implementiert durch GitHub-User \textit{Preliy} und frei
verfügbar im Rahmen einer BSD 3-Clause Lizenz, bietet ein spezialisiertes
Framework für die Robotersteuerung in Unity mit modularer Architektur, die
Gelenksteuerung, Kinematik, Koordinatentransformationen und Echtzeitüberwachung
als separate Komponenten organisiert. Das System unterstützt sowohl direkte
Gelenkmanipulation auf niedriger Ebene als auch kartesische Steuerung auf
höherer Abstraktionsebene, wodurch es für verschiedene Roboteranwendungen
einsetzbar ist.\vglcite{preliyflange2024} Im Kontext dieser Entwicklungsarbeit wird Flange zur
Implementierung und Visualisierung der Denavit-Hartenberg-Parameter und damit
einhergehender Achstransformationen eingesetzt,
da diese Notation als fundamentales Werkzeug der Robotik eine systematische
Beschreibung der Geometrie serieller Robotermechanismen ermöglicht und somit die
Anwendung etablierter algorithmischer Verfahren für kinematische Berechnungen,
Jacobi-Matrizen sowie Bewegungsplanung unterstützt.\vglcite[590]{corke2007}

Flange ermöglicht eine direkte Konfiguration des Roboters über \textit{Frame}
und \textit{JointTransformation} Scripts. Ein Frame definiert dabei die
Denavit-Hartenberg-Parameter für einen Teil der kinematischen Kette, eine
JointTransformation die Paramter des Gelenks, also möglicher maximaler Ausschlag
in positive und negative Achsrotationsrichtung sowie maximale Geschwindigkeit
und Beschleunigung. Diese Funktion werden im Rahmen dieser Implentierung
genutzt, um den zu simulierenden Roboter theoretisch nachvollziehbar im Raum
bewegen zu können.

\subsection{Aufbau der Roboterzelle}

