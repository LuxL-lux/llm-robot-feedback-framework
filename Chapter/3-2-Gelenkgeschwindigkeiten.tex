\subsection{Joint Dynamics Monitor}
\label{sssec:joint_dynamics_monitor}

\paragraph{Theoretische Grundlagen} Die Überwachung kinematischer Grenzwerte
während der Werkstückhandhabung stellt eines der zentralen Ziele des Frameworks dar. Die besondere Herausforderung in HTTP-basierten Systemen
liegt in der diskreten Datenakquisition mit Polling-Intervallen von 50-1000ms,
wodurch die numerische Differentiation zu erheblichen Instabilitäten
führt.

Geschwindigkeit $\dot{\theta}_i$ und Beschleunigung $\ddot{\theta}_i$ können folgendermassen aus der diskreten Zeitreihe approximiert werden:
\begin{equation}
	\dot{\theta}_i[k] = \frac{\theta_i[k] - \theta_i[k-1]}{\Delta t_k}, \quad
	\ddot{\theta}_i[k] = \frac{\dot{\theta}_i[k] - \dot{\theta}_i[k-1]}{\Delta t_k}
	\label{eq:discrete_derivatives}
\end{equation}

In der Praxis und nach initalem Testen hat sich herausgestellt, dass die
Berechnung dessen durch die untstetige Bewegung unrealistische Werte liefert.
Das hängt damit zusammen, dass der HTTP-Endpunkt des Robotercontrollers in
festgelegten Zeitabständen abgefragt wird und so die Gelenkpositionen des
Roboters erneuert werden. Bewegt sich der Roboter über eine bestimmte Strecke,
wird die von Unity lediglich als unstetige Bewegung mit theoretisch unendlich
grosser Geschwindigkeit wahrgenommen. Daher ist hier eine Glättung der Werte
nötig, um eine stetige Bewegung zu simulieren.

\paragraph{Mehrstufiger Filteransatz}
Zur Stabilisierung implementiert das Framework einen dreistufigen Filteransatz:

\begin{enumerate}
	\item \textbf{Exponential Moving Average (EMA):}
	      $\dot{\theta}_{i,\text{EMA}}[k] = \alpha \cdot \dot{\theta}_{i,\text{raw}}[k] + (1-\alpha) \cdot \dot{\theta}_{i,\text{EMA}}[k-1]$ mit $\alpha = 0.2$

	\item \textbf{Moving Window Average:}
	      $\dot{\theta}_{i,\text{MW}}[k] = \frac{1}{N} \sum_{j=0}^{N-1} \dot{\theta}_{i,\text{EMA}}[k-j]$ mit Fenstergröße $N = 8$

	\item \textbf{Outlier Rejection:}
	      $|\dot{\theta}_{i,\text{final}}[k] - \dot{\theta}_{i,\text{final}}[k-1]| \leq \tau_v \cdot \dot{\theta}_{i,\text{max}}$ mit $\tau_v = 0.2$
\end{enumerate}
\noindent
Durch die verwendete Glättung werden schnelle Änderungen in den Gelenkwerten
erst zeitverzögert sichtbar. Dies wirkt sich unmittelbar auf die Erkennung von
Geschwindigkeitsverletzungen aus, da der Monitor die Ereignisse nicht exakt im
Moment des Überschreitens ausgibt, sondern nach einigen Abtastungen.

\paragraph{Grenzwertdefinition} Die Grenzwerte werden aus den
Flange-Framework-Konfigurationen extrahiert oder manuell spezifiziert. Ein
Sicherheitsfaktor $\lambda = 0.8$ reduziert die nominellen Maximalwerte
präventiv. Für den ABB IRB 6700 ergeben sich beispielsweise
Geschwindigkeitsgrenzen von 88°/s für die Hauptachsen und bis zu 168°/s für das
Handgelenk.

\paragraph{Implementierung der Smoothing-Pipeline}
Die zentrale Smoothing-Methode kombiniert die Filteransätze:

\begin{figure}[H]
	\inputminted[fontsize=\footnotesize]{csharp}{code-snippets/SmoothVelocities.cs}
	\caption{Mehrstufige Smoothing-Pipeline zur Geschwindigkeitsberechnung}
	\label{listing:smoothing_pipeline}
\end{figure}

\noindent
Als weiteres Feature implementiert der Monitor eine Methode, welche sich an die
Update-Rate der State-Updates (also die Rate neuer Gelenkdaten) anpasst. So kann
garantiert werden, dass ein State nicht doppelt verarbeitet und so
beispielsweise die Geschwindigkeit falsch interpretiert wird, da es sich um den
immer gleichen State des Roboters handelt.\\

\noindent
Bei der Überschreitung der Schwellwerte wird der \texttt{SafetyMonitor}
mittels eines \texttt{SafetyEvents} benachrichtigt. Als zusätzliche Metadaten enthält
diese Meldung die Eventart (Geschwindigkeits oder
-Beschleunigungsüberschreitung) und die damit zusammenhängenden Schwellwerte und
aktuellen Werte. Da es in dem Rahmen vorkommen kann, dass hier die
Geschwindigkeit über einen längeren Zeitraum überschritten wird, gibt der
Monitor ebenfalls ein SafetyEvent aus, wenn die Geschwindigkeit unter den
gegebenen Schwelllwert sinkt.
