\chapter{Meine Notizen}
\label{cap:Meine Notizen}

\section{Testsection}
\label{sec:Testsection}

Eine Fallstudie zeigt, dass die Komplexität herkömmlicher Programmiermethoden eine Barriere für die Einführung von Robotern darstellt, insbesondere für Benutzer ohne tiefgehende Programmierkenntnisse.\cite{bhat2024groundingllmsrobottask}

Die Programmierung von Industrierobotern erfordert oft tiefgreifendes Expertenwissen über Kinematik, Dynamik und proprietäre Programmiersprachen. Ein Beispiel hierfür ist die KUKA Robot Language (KRL), die spezifisch für KUKA-Roboter entwickelt wurde. Die Reverse-Engineering-Studie von Mühe et al. zeigt, dass solche domänenspezifischen Sprachen oft controller-spezifisch, wenig ausdrucksstark und schwer erweiterbar sind \cite{mühe2010reverseengineeringkukarobotlanguage}

Another challenge is the high variability in robotic settings. Robot platforms are inherently diverse with different
physical characteristics, configurations, and capabilities. Realworld environments that robots operate in are also diverse
and uncertain with a wide range of variations. Due to all
these variabilities, robotic solutions are usually tailored to
specific robot platforms with specific layouts, environments,
and objects for specific tasks. These solutions are not generalizable across various embodiments, environments, or tasks.
Hence, to build general-purpose pretrained robotic foundation
models, a key factor is to pre-train large models that are
task-agnostic, cross-embodiment, and open-ended and capture
diverse robotic data.  \cite{firoozi2023foundationmodelsroboticsapplications}

Die Notwendigkeit, Vertrauen in LLM-generierten Code zu schaffen, wird auch im Kontext von Ansätzen wie TiCoder betont, bei dem Tests zur Klärung der Nutzerintention und zur Reduktion von Fehlern eingesetzt werden  \cite{Fakhoury_2024}