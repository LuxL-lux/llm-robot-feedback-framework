\chapter{Diskussion}
\label{cap:diskussion}
Im Folgenden werden die in
Kapitel~\ref{cap:Ergebnisse} präsentierten
Ergebnisse kritisch reflektiert und in den fachlichen Kontext eingeordnet. Im
Zentrum stehen die Gesamtarchitektur des Frameworks und die vier implementierten
Safetymodule. Ergänzend wird ein Experteninterview mit Daniel Syniawa
herangezogen, in dem die Software, die Architektur sowie konkrete Testfälle in
RobotStudio gemeinsam betrachtet wurden. Ziel ist es, die Stärken und Grenzen
der Arbeit transparent zu machen und konkrete Ansatzpunkte für zukünftige
Arbeiten zu identifizieren.

\section{Diskussion des Frameworks}

Die Architektur hat sich als tragfähige Grundlage erwiesen: Durch den
adapterbasierten Zugriff auf die Roboterschnittstelle (z.\,B. ABB Robot Web
Services) und eine konsequent eventgetriebene Struktur konnten
Safetymodule unabhängig voneinander entwickelt und über ein gemeinsames
Interface in das RobotSystem integriert werden. Das Event-System
(Observer-Pattern) entkoppelt Erkennung und Verarbeitung; die JSON-basierte
Persistenz vereinfacht Nachvollziehbarkeit und Weiterverarbeitung. Positiv
wirkte sich zudem die Nutzung der Denavit–Hartenberg-Parameter aus, die eine
konsistente kinematische Modellierung verschiedener Robotermodelle erlaubt.

Gleichzeitig bleibt offen, inwiefern einzelne Simulationsparameter
oder Modellierungsentscheidungen in Unity bei hochkomplexen Geometrien eine
Limitation darstellen könnten. Die Unity-Engine bietet hier zwar eine
Grundlage physischer Modellierung, abschließende Aussagen zur Genauigkeit in
sehr dynamischen oder kleinschrittigen Prozessen können aber nicht getroffen
werden. Diese Aspekte wurden nicht systematisch untersucht und markieren Raum
für künftige Studien.

Das Experteninterview mit Daniel Syniawa ordnet die Arbeit in die
Praxis ein: Er bestätigte, dass es insgesamt nur wenige plattformübergreifende
Werkzeuge mit integrierter Physik gibt und dass die Tool-Landschaft stark
proprietär geprägt ist. Nach seiner Erfahrung ist bereits die stabile
Inbetriebnahme und Anbindung von Robotern für viele Firmen aufwendig;
physikalische Simulation mit einer Modellierung des Arbeitsraumes in RobotStudio
sowie die Auswertung der hier untersuchten Parameter ist nur in begrenztem
Umfang möglich und benötigt beträchtliches Expertenwissen und Zeit. Syniawa wies
außerdem darauf hin, dass Services proprietär Hersteller wie die RobotStudio-API
(Robot Web Services) nicht trivial in der Anwendung sind, oft schlecht
dokumentiert und auf eine insgesamt kleine Nutzerbasis trifft. Auch das lässt
sich im Rahmen der Entwicklung dieses Frameworks bestätigen.

Als hilfreich lässt sich ausserdem der Output des JSON-basierten Loggings
werten: Syniawa spricht hier vor allem von den mit einem SafetyEvent zusammen
herausgegebenen Metadaten zum aktuellen Stand des Programmzeigers und
der aktuellen
ausgeführten Routine. Dies stellt einen vielversprechenden Ansatz zum Debugging
im Roboterprogrammcode dar, welcher manuell mit deutlich mehr Aufwand verbunden
wäre. Hier gilt es weiter zu evaluieren, inwiefern sich dies quantitativ
beschreiben lässt, so Syniawa.

\subsection{Prozessfolgenüberwachung}

Das Modul erkennt Abweichungen in der vorgesehenen Reihenfolge zuverlässig,
sofern Stationen korrekt modelliert und ausgelöst werden. Nicht abgedeckt sind
Konstellationen, in denen ein Werkstück zwischen zwei Stationen
unbeabsichtigt abgelegt oder verloren wird, ohne dass eine Station detektiert
wird. Die Praxistauglichkeit hängt damit von der Qualität der
Prozessmodellierung und der Art des Prozesses ab. Im aktuellen Fokus stand hier
ein sequentieller Prozess, die Literatur beschreibt hier im Kontext
industrieller Fertigung jedoch mehrere Prozessarten und theoretische
Modellierungsansätze.{\vglcite[39\psq]{cassandras2021} Im Interview
  wurde die grundsätzliche Relevanz dieses
  Moduls bestätigt – zugleich wird deutlich, dass die industrielle Praxis oft
  komplexere Ablaufmodelle erfordert.

  \subsection{Kollisionserkennung}

  In der Simulation wurden die vorgesehenen Kollisionsfälle erkannt. Zugleich
  traten Fehlalarme auf, insbesondere beim Greifen und Loslassen von
  Werkstücken. Wie stark Modellierungsdetails oder
  Simulationsparameter das Verhalten in Grenzfällen beeinflussen,
  wurde in dieser Arbeit nicht systematisch untersucht
  und ist als potenzielle Limitation zu betrachten. Weiterführend ist die
  Genauigkeit der Modellierung der Meshes des Roboters und der Umgebung hier
  essenziell: Unity modelliert konvexe Meshes, welche die räumlichen Grenzen
  darstellen mit maximal 255 Kanten. Daher kann es passieren, dass die
  tatsächliche Topologie des Robotermodells vom Kollisionskörper abweicht.
  Insgesamt liefert das Modul einen belastbaren Proof-of-Concept, dessen
  Generalisierung im Rahmen weiterführender Evaluierungen zu prüfen ist.

  \subsection{Achsgeschwindigkeiten und -beschleunigungen}

  Grenzwertverletzungen wurden zuverlässig detektiert. In den Ergebnissen zeigte
  sich allerdings ein zeitlicher Versatz zwischen den in RobotStudio
  vorliegenden
  Referenzdaten und der Detektion in Unity3D. Dies ist plausibel auf
  Aktualisierungsrate und eingesetzten Glättungsalgorithmus
  zurückzuführen, dessen
  Parameter konfigurierbar sind. Ob dadurch ein Versatz mit den in der Praxis
  vom Roboter gefahrenen Achsgeschwindigkeiten und -beschleunigungen
  entsteht, lässt sich hier
  nicht abschließend bewerten.

  \subsection{Singularitätserkennung}

  Die gewählte, winkelbasierte Heuristik funktionierte für den betrachteten
  Roboter, ist jedoch nicht universell. Alternativ bieten sich
  Kennzahlen an, die
  näher an der Kinematik operieren, etwa das Manipulierbarkeits-Maß
  nach Yoshikawa oder
  der kleinste Singulärwert der Jacobi-Matrix als Abstandsmaß zur Singularität.
  Eine generische, Jacobi-basierte Methode zur Erkennung von
  Freiheitsgradverlusten wurde implementiert, im Rahmen der vorliegenden Tests
  jedoch nicht eingesetzt; eine Erweiterung auf andere Roboter wäre
  möglich, wurde
  aber nicht vorgenommen.

  Im Interview formulierte Syniawa einen pragmatischen Maßstab für
  die Evaluation:
  Für eine belastbare Beurteilung wäre ein direkter Vergleich mit manuellem
  Debugging in RobotStudio sinnvoll, also ein Messaufbau, der die Zeit bis zur
  Fehlerlokalisierung im RAPID-Code der Zeit gegenüberstellt, die das Framework
  über Ausführung und Logging benötigt. Zugleich hob er hervor, dass die
  automatische Bereitstellung von Programmzeiger, aktueller Pose und Kontext im
  Event-Log eine erhebliche Arbeitserleichterung darstellt und die Analyse
  prinzipiell auch weniger erfahrenen Anwendern ermöglicht.

  \section{LLM-gestützte Rückkopplung der Simulationsergebnisse}

  Die Ergebnisse zeigen, dass das Framework Fehlerzustände konsistent
  und kontextreich protokolliert: Für jedes Ereignis liegen
  Aktuelle Bewegungs- und Programmdaten, beispielsweise das aktuelle
  Modul, Routine,
  Programmzeile sowie Motordaten und Achswinkel vor. Hinzu kommen
  event-spezifische Felder im \texttt{eventDataJson}, etwa
  Kollisionspunkt und Distanz oder Gelenkwinkel und
  Manipulierbarkeitswert bei Singularitäten. Diese maschinenlesbare
  Struktur eignet sich unmittelbar als gezielter Eingabekontext für
  generative Modelle: Der Fehler wird präzise beschrieben, der
  relevante Zustandsausschnitt ist enthalten, und die Semantik stammt
  aus den domänenspezifischen Monitoren. Das bietet eine gute Basis, um Code-
  oder Pfadänderungen vorzuschlagen und anschließend identisch zu
  verifizieren.

  Operativ lässt sich darauf ein kurzer Iterationszyklus aufbauen: Aus
  einem fehlgeschlagenen Lauf durch Nutzer- oder LLM-generierten
  Roboterprogrammcode wird ein kompakter Fehlerkontext
  gebildet und an ein LLM übergeben: Das
  LLM liefert einen minimalen Änderungsvorschlag am Robotercode
  bzw.\ an der Pfaddefinition und durch die Re-Simulation kann eine erneute
  Prüfung des Codes stattfinden. Die Akzeptanzkriterien für eine erfolgreiche
  Behebung des Fehlers, beispielsweise dem minimalen Achswinkel zur Vermeidung
  einer Singularität, liegen bereits
  im Rahmen der Eventdaten und Monitorlogik im System vor, wodurch die
  Bewertung reproduzierbar bleibt.
  Damit kann das Framework als
  Schnittstelle zwischen realitätsnaher Ausführung und
  LLM-gestützter Verbesserung dienen. Praktisch sind drei Punkte entscheidend
  und aus den Ergebnissen ableitbar: erstens ein schlanker, stabiler
  Prompt-Ausschnitt aus genau den Feldern, die im Logging ohnehin
  verfügbar sind (z.\,B.\ Programmpointer, Gelenkwinkel,
  Kollisionspunkt), zweitens feste Akzeptanzregeln in der Simulation,
  drittens Versionierung und Wiederholbarkeit der Läufe. Auf Basis dieser
  Ergebnisse ließe sich in zukünftiger Forschung untersuchen, in welchem Umfang
  sich die Fehlerquote in generiertem Roboterprogrammcode tatsächlich reduziert
  und daraus resultierend auch die erforderliche Nachbearbeitungszeit
  des Programmcodes verringert wird.

  \section{Reflexion des eigenen Vorgehens}

  Die Entwicklung folgte bewusst einem iterativen Vorgehen: von einem kleinen
  Testfall hin zu einer breiteren Abdeckung, mit Zwischenstufen des
  Refactorings. Dieses Vorgehen erwies sich als geeignet, um
  Architekturentscheidungen
  (Adapter/Observer) empirisch zu validieren. Rückblickend entstanden
  stellenweise Komponenten, die für den unmittelbaren Use-Case
  komplexer waren als nötig – gleichwohl war die iterativ-explorative
  Herangehensweise im Kontext einer Framework-Entwicklung zweckmäßig
  und hat zu einer für die durchzuführenden Analysen geeigneten
  Struktur geführt.

  \section{Grenzen und Generalisierbarkeit}

  Die vorliegende Arbeit wurde innerhalb einer vorkonfigurierten
  Simulationsumgebung evaluiert. Echtzeitverhalten,
  Sensitivität gegenüber Simulationsparametern sowie Übertragbarkeit auf weitere
  Roboter wurden nicht systematisch untersucht. Darüber hinaus
  beschränkt sich die Implementierung der Adapter auf die RWS API. Das Interview
  verdeutlicht, dass proprietäre Ökosysteme, komplexe Schnittstellen
  und eine kleine Nutzerbasis zusätzliche Hürden für
  Verallgemeinerung und Transfer darstellen. Diese Punkte markieren
  bewusste Grenzen des aktuellen Stands und leiten unmittelbar zu
  Folgestudien über.
