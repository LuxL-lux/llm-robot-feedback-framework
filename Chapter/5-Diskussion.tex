\chapter{Diskussion}
\label{cap:diskussion}


\section{Einleitung} In diesem Kapitel werden die in
Kapitel~\ref{cap:Ergebnisse} präsentierten
Ergebnisse kritisch reflektiert und in den fachlichen Kontext eingeordnet. Im
Zentrum stehen die Gesamtarchitektur des Frameworks und die vier implementierten
Safetymodule. Ergänzend wird ein Experteninterview mit Daniel Syniawa
herangezogen, in dem die Software, die Architektur sowie konkrete Testcases in
RobotStudio gemeinsam betrachtet wurden. Ziel ist es, die Stärken und Grenzen
der Arbeit transparent zu machen und konkrete Ansatzpunkte für zukünftige
Arbeiten zu identifizieren.

\section{Diskussion des Frameworks}

Die Architektur hat sich als tragfähige Grundlage erwiesen: Durch den
adapterbasierten Zugriff auf die Roboterschnittstelle (z.\,B. ABB Robot Web
Services) und eine konsequent \emph{event-getriebene} Struktur konnten
Safetymodule unabhängig voneinander entwickelt und über ein gemeinsames
Interface in das \emph{RobotSystem} integriert werden. Das Event-System
(Observer-Pattern) entkoppelt Erkennung und Verarbeitung; die JSON-basierte
Persistenz vereinfacht Nachvollziehbarkeit und Weiterverarbeitung. Positiv
wirkte sich zudem die Nutzung der Denavit–Hartenberg-Parameter aus, die eine
konsistente kinematische Modellierung verschiedener Robotermodlle erlaubt.\\

\noindent Gleichzeitig bleiben offen, inwiefern einzelne Simulationsparameter
oder Modellierungsentscheidungen in Unity bei hochkomplexen Geometrien eine
Limitation darstellen \emph{könnten}. Die Unity-Engine bietet hier zwar eine
Grundlage physischer Modellerung, abschliessende Aussagen zur Genauigkeit in
sehr dynamischen oder kleinschrittigen Prozessen können aber nicht getrofen
werden. Diese Aspekte wurden nicht systematisch untersucht und markieren Raum
für künftige Studien.\\

\noindent Das Experteninterview mit Daniel Syniawa ordnet die Arbeit in die
Praxis ein: Er bestätigte, dass es insgesamt nur wenige plattformübergreifende
Werkzeuge mit integrierter Physik gibt und dass die Tool-Landschaft stark
proprietär geprägt ist. Nach seiner Erfahrung ist bereits die stabile
Inbetriebnahme und Anbindung von Robotern für viele Firmen aufwendig;
physikalische Simulation mit einer Modellerung des Arbeitsraumes in RobotStudio
sowie die Auswertung der hier untersuchten Parameter ist nur in begrenztem
Umfang möglich und benötigt beträchtliches Expertenwissen und Zeit. Syniawa wies
außerdem darauf hin, dass Services proprietär Hersteller wie die RobotStudio-API
(Robot Web Services) nicht trivial in der Anwendung sind, oft schlecht
dokumentiert und auf eine insgesamt kleine Nutzerbasis trifft. Auch das lässt
sich im Rahmen der Entwicklung dieses Frameworks bestätigen.

Als hilfreich lässt sich ausserdem der Output des JSON-basierten Loggings
werten: Syniawa spricht hier vor allem von den mit einem SafetyEvent zusammen
herausgegebenen Metadaten zum aktuellen Stand des Programmzeigers und der aktuellen
ausgeführten Routine. Dies stellt einen vielversprechenden Ansatz zum Debugging
im Roboterprogrammcode dar, welcher manuell mit deutlich mehr Aufwand verbunden
wäre. Hier gilt es weiter zu evaluieren, inwiefern sich dies quantitativ
beschreiben lässt, so Syniawa.

\section{Diskussion der Safetymodule}

\subsection{Prozessfolgenüberwachung}

Das Modul erkennt Abweichungen in der vorgesehenen Reihenfolge zuverlässig,
sofern Stationen korrekt modelliert und ausgelöst werden. Nicht abgedeckt sind
Konstellationen, in denen ein Werkstück \emph{zwischen} zwei Stationen
unbeabsichtigt abgelegt oder verloren wird, ohne dass eine Station detektiert
wird. Die Praxistauglichkeit hängt damit von der Qualität der
Prozessmodellierung und der Art des Prozesses ab. Im aktuellen Fokus stand hier
ein sequentieller Prozess, die Literatur beschreibt hier im Kontext
industrieller Fertigung jedoch mehrere Prozessarten und theoretische
Modellierungsansätze. Im Interview wurde die grundsätzliche Relevanz dieses
Moduls bestätigt; zugleich wird deutlich, dass die industrielle Praxis oft
komplexere Ablaufmodelle erfordert.

\subsection{Kollisionserkennung}

In der Simulation wurden die vorgesehenen Kollisionsfälle erkannt; zugleich
traten Fehlalarme auf, insbesondere beim Greifen und Loslassen von Werkstücken.
Wie stark Modellierungsdetails oder Simulationsparameter das Verhalten in
Grenzfällen beeinflussen, wurde in dieser Arbeit nicht systematisch untersucht
und ist als potenzielle Limitation zu betrachten. Weiterführend ist die
Genauigkeit der Modellierung der Meshes des Roboters und der Umgebung hier
essentiell: Unity modelliert konvexe Meshes, welche die räumlichen Grenzen
darstellen mit maximal 255 Kanten. Daher kann es passieren, dass die
tatsächliche Topologie des Robotermodells vom Kollisionskörper abweicht.
Insgesamt liefert das Modul einen belastbaren Proof-of-Concept, dessen
Generalisierung im Rahmen weiterführender Evaluierungen zu prüfen ist.

\subsection{Achsgeschwindigkeiten und -beschleunigungen}

Grenzwertverletzungen wurden zuverlässig detektiert. In den Ergebnissen zeigte
sich allerdings ein zeitlicher Versatz zwischen den in RobotStudio vorliegenden
Referenzdaten und der Detektion in Unity: Überschreitungen wurden etwas später
erkannt und endeten geringfügig später. Dieser Versatz ist plausibel auf
Aktualisierungsrate und eingesetzten Glättungsalgorithmus zurückzuführen, dessen
Parameter konfigurierbar sind. Ob dadurch ein Versatz mit den in der Praxis
vom Roboter gefahrenen Achsgeschwindigkeiten und -beschleunigungen entsteht, lässt sich hier
nicht abschliessend bewerten.

\subsection{Singularitätserkennung}

Die gewählte, winkelbasierte Heuristik funktionierte für den betrachteten
Roboter, ist jedoch nicht universell. Alternativ bieten sich Kennzahlen an, die
näher an der Kinematik operieren, etwa das Manipulability-Maß (Yoshikawa) oder
der kleinste Singulärwert der Jacobi-Matrix als Abstandsmaß zur Singularität.
Eine generische, Jacobian-basierte Methode zur Erkennung von
Freiheitsgradverlusten wurde implementiert, im Rahmen der vorliegenden Tests
jedoch nicht eingesetzt; eine Erweiterung auf andere Roboter wäre möglich, wurde
aber nicht vorgenommen.\\

\noindent
Im Interview formulierte Syniawa einen pragmatischen Maßstab für die Evaluation:
Für eine belastbare Beurteilung wäre ein direkter Vergleich mit manuellem
Debugging in RobotStudio sinnvoll, also ein Messaufbau, der die Zeit bis zur
Fehlerlokalisierung im RAPID-Code der Zeit gegenüberstellt, die das Framework
über Ausführung und Logging benötigt. Zugleich hob er hervor, dass die
automatische Bereitstellung von Programmzeiger, aktueller Pose und Kontext im
Event-Log eine erhebliche Arbeitserleichterung darstellt und die Analyse
prinzipiell auch weniger erfahrenen Anwendern ermöglicht.

\section{Reflexion des eigenen Vorgehens}

Die Entwicklung folgte bewusst einem iterativen Vorgehen: von einem kleinen
Testfall hin zu einer breiteren Abdeckung, mit Zwischenstufen des Refactorings.
Dieses Vorgehen erwies sich als geeignet, um Architekturentscheidungen
(Adapter/Observer) empirisch zu validieren. Rückblickend entstanden stellenweise
Komponenten, die für den unmittelbaren Use-Case komplexer waren als nötig;
gleichwohl war die iterativ-explorative Herangehensweise im Kontext einer
Framework-Entwicklung zweckmäßig und hat zur jetzigen Struktur geführt.

\section{Grenzen und Generalisierbarkeit}

Die vorliegende Arbeit wurde in der Simulation evaluiert; Echtzeitverhalten,
Sensitivität gegenüber Simulationsparametern sowie Übertragbarkeit auf weitere
Roboter wurden nicht systematisch untersucht. Darüber hinaus beschränkt sich die
Adaptervalidierung auf ABB Robot Web Services. Das Interview verdeutlicht, dass
proprietäre Ökosysteme, komplexe Schnittstellen und eine kleine Nutzerbasis
zusätzliche Hürden für Verallgemeinerung und Transfer darstellen. Diese Punkte
markieren bewusste Grenzen des aktuellen Stands und leiten unmittelbar zu
Folgestudien über.


