\subsection{Collision Detection Monitor}
\subsubsection{Theoretische Grundlagen der Kollisionserkennung}
\noindent
Die Kollisionserkennung in Unity basiert auf einem zweistufigen Verfahren, das
von der integrierten PhysX 5.1 Engine implementiert wird
\vglcite{nvidia2024physx5}. In der \textbf{Broad Phase} werden zunächst
Axis-Aligned Bounding Boxes (AABB) aller Objekte auf mögliche Überlappungen
geprüft. Dieser Schritt reduziert die Anzahl der detaillierten Prüfungen von
$O(n^2)$ auf $O(n \log n)$ durch räumliche Partitionierung
\vglcite{ericson2004real}. Nur Objektpaare mit überlappenden Bounding Boxes
werden an die \textbf{Narrow Phase} weitergereicht, wo der präzise
GJK-Algorithmus (Gilbert-Johnson-Keerthi) die tatsächliche Kollision zwischen
konvexen Geometrien ermittelt \vglcite{gilbert1988fast}.
\noindent
Unity unterscheidet dabei zwischen physikalischen Kollisionen und
Trigger-Events. Während physikalische Kollisionen Reaktionskräfte berechnen,
detektieren Trigger lediglich Überlappungen und generieren entsprechende Events
\vglcite{unity2024manual}. Die Layer-Matrix ermöglicht zusätzlich eine selektive
Kollisionserkennung, indem bestimmte Layer-Kombinationen von der Prüfung
ausgeschlossen werden können. Dies reduziert sowohl die Rechenlast als auch
unerwünschte Kollisionsmeldungen zwischen semantisch nicht relevanten Objekten.

\subsubsection{Praktisches Vorgehen}
\noindent
Für die Kollisionserkennung eines Roboterarms mit sechs Freiheitsgraden müssen
spezielle Anpassungen vorgenommen werden: Da benachbarte Glieder der
kinematischen Kette konstruktionsbedingt immer in Kontakt stehen, würden diese
ohne Filterung kontinuierlich Kollisionen melden. Unity's
\texttt{Physics.IgnoreCollision()} API ermöglicht das explizite Ausschließen
solcher Kollisionspaare.
\noindent
Das Layer-System wird zur semantischen Trennung verschiedener Objektkategorien
genutzt. Der Roboter operiert auf dem Standard-Layer, während Werkstücke auf
Layer 30 (Parts) und Stationen auf Layer 31 (ProcessFlow) (siehe Kapitel
\ref{section:prozessfolgen}) platziert werden.
Durch Konfiguration der Collision Layer Matrix wird sichergestellt, dass nur
sicherheitsrelevante Kollisionen zwischen Roboter und Hindernissen detektiert
werden, während Process Flow Trigger-Events separat behandelt werden.

\subsubsection{Konkrete Implementierung}
\noindent
Die Implementierung nutzt Trigger-Collider für berührungslose Detektion. Jedes
Roboterglied erhält einen konvexen Mesh-Collider mit einem dedizierten
Detector-Komponenten. Die Filterung benachbarter Glieder erfolgt durch hierarchische Analyse der
Roboterstruktur, dargstellent in \ref{listing:adjacentFrames}.

\begin{figure}[H]
	\inputminted[fontsize=\footnotesize]{csharp}{code-snippets/SetupAdjacentFramesIgnoring.cs}
	\caption{Algorithmus zum Ermitteln kinematisch benachbarter Mesh-Collider}
	\label{listing:adjacentFrames}
\end{figure}
\noindent
Tritt nun eine Kollision mit einem der vordefinierten Objekte auf, wird die
\texttt{OnRobotPartCollision()}-Methode des Monitors aufgerufen. Diese prüft
zunächst, ob es sich um ein gegriffenes Werkstück handelt - solche Kollisionen
werden ignoriert, da das Werkstück temporär Teil des Roboters ist. Anschließend
wird durch einen Cooldown-Mechanismus verhindert, dass dieselbe Kollision
innerhalb kurzer Zeit mehrfach gemeldet wird.

\noindent
Die Kollisionserkennung unterscheidet zwischen kritischen und unkritischen
Ereignissen basierend auf konfigurierbaren Tags. Objekte mit Tags wie
\texttt{"Machine"} oder \texttt{"Obstacles"} lösen kritische Safety Events aus,
während andere Kollisionen als Warnungen klassifiziert werden. Der Monitor
generiert dabei ein \texttt{SafetyEvent}-Objekt mit vollständigen Metadaten
über die Kollision, einschließlich der beteiligten Roboterglieder, des
Kollisionspunkts in Weltkoordinaten und der Distanz zwischen den Objekten.

\noindent
Diese Ereignisse werden an den zentralen \texttt{RobotSafetyManager}
weitergeleitet, der sie mit dem aktuellen Roboterzustand anreichert und
entsprechend der konfigurierten Logging-Strategie verarbeitet. Durch diese
Architektur bleibt der Collision Detection Monitor unabhängig vom spezifischen
Robotertyp und kann flexibel in verschiedenen Simulationsszenarien eingesetzt
werden.


