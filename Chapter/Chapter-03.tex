%%%%%%%%%%%%%%%%%%%%%%%%%%%%%%%%%%%%%%%%%%%%%%%%%%%%%%%%%%%%%%%%%%%%%%%%%%%%%%%%%%%%%%%%%%%%%%%%%%%%%%%%%%%%%%%%%%%%%%%%%%%%%%%%%%%%%%%%%%%%%%%%%%%%%%%%%%%%%%%%%%%%%%%%%%%%%%%%%%%%%%%%%%%%%%%%%%%%%%%%%%%%%%%%%%%%%%%%%%%%%%%%%%%%%%%%%%%%%%%%%%%%%%%%%
%%%&& Weitere Formalia %%%%%
%%%%%%%%%%%%%%%%%%%%%%%%%%%%%%%%%%%%%%%%%%%%%%%%%%%%%%%%%%%%%%%%%%%%%%%%%%%%%%%%%%%%%%%%%%%%%%%%%%%%%%%%%%%%%%%%%%%%%%%%%%%%%%%%%%%%%%%%%%%%%%%%%%%%%%%%%%%%%%%%%%%%%%%%%%%%%%%%%%%%%%%%%%%%%%%%%%%%%%%%%%%%%%%%%%%%%%%%%%%%%%%%%%%%%%%%%%%%%%%%%%%%%%%%

\chapter{Weitere Formalia}
\label{cap:Weitere Formalia}




%%%%%%%%%%%%%%%%%%%%%%%%%%%%%%%%%%%%%%%%%%%%%%%%%%%%%%%%%%%%%%%%%%%%%%%%%%%%%%%%%%%%%%%%%%%%%%%%%%%%%%%%%%%%%%%%%%%%%%%%%%%%%%%%%%%%%%%%%%%%%%%%%%%%%%%%%%%%%%%%%%%%%%%%%%%%%%%%%%%%%%%%%%%%%%%%%%%%%%%%%%%%%%%%%%%%%%%%%%%%%%%%%%%%%%%%%%%%%%%%%%%%%%%%

\section{Sonstiges}

\begin{itemize}
	\item Sobald eine Abkürzung eingeführt wurde (Beispiel: Mensch-Roboter-Kollaboration, kurz MRK), dann sollte diese Abkürzung durchgängig verwendet werden. In Ausnahmefällen (z.\,B. bei Überschriften) kann hiervon abgewichen werden.
	\item Alle Abkürzungen (u.\,a., z.\,B., i.\,d.\,R. etc.) oder Maßangaben (200\,kg) werden mit einem kleinen geschützten Leerzeichen getrennt. In der LaTeX-Umgebung wird dies mit \verb+\,+ getrennt.
	\item "`Hurenkinder\footnote{Als Hurenkind wird die letzte Zeile eines Absatzes bezeichnet, wenn sie zugleich die erste einer neuen Seite oder Spalte ist.}"' bzw. "`Schusterjungen\footnote{Als Schusterjunge wird eine am Seiten- oder Spaltenende stehende Zeile eines neuen Absatzes bezeichnet, der auf der Folgeseite fortgesetzt wird.}"' sollen vermieden werden.
	\item Der Gebrauch des Wortes "`man"' sollte vermieden werden.
	\item Variablen, Indizes etc. werden immer kursiv dargestellt, also z.\,B. $i$ anstatt i. Matrizen und Vektoren werden zudem kursiv und fett dargestellt, d.\,h. Vektor $\bs{x}(t)$ anstatt $x(t)$!
	\item Ein Kapitel besteht i.\,d.\,R. aus verschiedenen Abschnitten, d.\,h. es existiert ein Kapitel 2, jedoch ein Abschnitt 2.1 oder Abschnitt 3.2.1.
\end{itemize}


%%%%%%%%%%%%%%%%%%%%%%%%%%%%%%%%%%%%%%%%%%%%%%%%%%%%%%%%%%%%%%%%%%%%%%%%%%%%%%%%%%%%%%%%%%%%%%%%%%%%%%%%%%%%%%%%%%%%%%%%%%%%%%%%%%%%%%%%%%%%%%%%%%%%%%%%%%%%%%%%%%%%%%%%%%%%%%%%%%%%%%%%%%%%%%%%%%%%%%%%%%%%%%%%%%%%%%%%%%%%%%%%%%%%%%%%%%%%%%%%%%%%%%%%
\newpage
\section{Häufige Fehler}

\noindent
Besteht eine Wortgruppe aus mehreren zusammengesetzten Wörtern, dann wird ein Bindestrich gesetzt.

% Bindestrich
\renewcommand{\arraystretch}{1.2}
\begin{table}[H]
	  \centering
		\caption{Bindestrich}
		\begin{tabular}{|p{0.45\textwidth}|p{0.45\textwidth}|}
			\hline
			\textbf{Richtig} & \textbf{Falsch} \tabularnewline
			 \hline
					Mensch-Roboter-Kollaboration & Mensch-Roboter Kollaboration \\	
					Not-Halt-Funktion & Not-Halt Funktion \\
					MRK-Anwendungen & MRK Anwendungen \\
					MRK-Applikation & MRK- Applikation \\
					Primär- und Sekundäranalyse & Primär und Sekundäranalyse \\		
					DIN-Norm & DIN Norm \\			
					Ergonomie-Analyse, Ergonomieanalyse & Ergonomie Analyse \\
				\hline
		\end{tabular}	
\end{table}
\renewcommand{\arraystretch}{1}

\noindent
Bei Verweisen auf Tabellen und Abbildungen kann auf einen Artikel verzichtet werden.

% Artikel
\renewcommand{\arraystretch}{1.2}
\begin{table}[H]
	  \centering
		\caption{Artikel}
		\begin{tabular}{|p{0.45\textwidth}|p{0.45\textwidth}|}
			\hline
			\textbf{Richtig} & \textbf{Falsch} \tabularnewline
			 \hline
					In Tabelle x... & In der Tabelle x... \\
					Abbildung y zeigt... & Die Abbildung y zeigt... \\
				\hline
		\end{tabular}	
\end{table}
\renewcommand{\arraystretch}{1}

% Literaturangabe
\renewcommand{\arraystretch}{1.2}
\begin{table}[H]
	  \centering
		\caption{Literaturangabe}
		\begin{tabular}{|p{0.45\textwidth}|p{0.45\textwidth}|}
			\hline 
			\textbf{Richtig} & \textbf{Falsch} \tabularnewline
			 \hline		
					Einer der ersten Schritte der Methode nach Beumelburg [2005] ist die Berechnung der Eignungsgrade von Mensch und Roboter. &
					Einer der ersten Schritte der Methode nach [Beumelburg, 2005] ist die Berechnung der Eignungsgrade von Mensch und Roboter. \\
					Dazu wird ein Kriterienkatalog entwickelt, welcher Kriterien aus den nachfolgenden Bereichen berücksichtigt [Bauer, 2006]. &
					Dazu wird ein Kriterienkatalog entwickelt, welcher Kriterien aus den nachfolgenden Bereichen berücksichtigt Bauer [2006]. \\
	 		 \hline
		\end{tabular}	
\end{table}
\renewcommand{\arraystretch}{1}

\noindent
Hervorhebungen erfolgen in der LaTeX-Umgebung über \verb+\emph{...}+.


% Anführungszeichen und Hervorhebung
\renewcommand{\arraystretch}{1.2}
\begin{table}[H]
	  \centering
		\caption{Anführungszeichen und Hervorhebung}
		\begin{tabular}{|p{0.45\textwidth}|p{0.45\textwidth}|}
			\hline 
			\textbf{Richtig} & \textbf{Falsch} \tabularnewline
			 \hline		
					Die Merkmale "`Taktzeit"', "`Prozesssicherheit"' und "`Qualität"' werden bewertet. &
					Die Merkmale Taktzeit, Prozesssicherheit und Qualität werden bewertet. \\
					Die Merkmale \emph{Taktzeit}, \emph{Prozesssicherheit} und \emph{Qualität} werden bewertet. &
					\\
	 		 \hline
		\end{tabular}	
\end{table}
\renewcommand{\arraystretch}{1}

\noindent
Zwischen die Glieder einer Aufzählung wird ein Komma gesetzt. Wenn jedoch eine Aufzählung abschließt, folgt kein Komma hinter das letzte Glied der Aufzählung.

% Aufzählungen
\renewcommand{\arraystretch}{1.2}
\begin{table}[H]
	  \centering
		\caption{Aufzählungen}
		\begin{tabular}{|p{0.45\textwidth}|p{0.45\textwidth}|}
			\hline 
			\textbf{Richtig} & \textbf{Falsch} \tabularnewline
			 \hline		
					Ein Robotersystem besteht aus mehreren Komponenten (Gelenke, Antriebe, Endeffektor etc.). &
					Ein Robotersystem besteht aus mehreren Komponenten (Gelenke, Antriebe, Endeffektor\textbf{,} etc.). \\
	 		 \hline
		\end{tabular}	
\end{table}
\renewcommand{\arraystretch}{1}




%%%%%%%%%%%%%%%%%%%%%%%%%%%%%%%%%%%%%%%%%%%%%%%%%%%%%%%%%%%%%%%%%%%%%%%%%%%%%%%%%%%%%%%%%%%%%%%%%%%%%%%%%%%%%%%%%%%%%%%%%%%%%%%%%%%%%%%%%%%%%%%%%%%%%%%%%%%%%%%%%%%%%%%%%%%%%%%%%%%%%%%%%%%%%%%%%%%%%%%%%%%%%%%%%%%%%%%%%%%%%%%%%%%%%%%%%%%%%%%%%%%%%%%%
\newpage
\section{Beispielhafter Programmcode}

\begin{lstlisting}[caption={Ausgabe der Inversen Kinematik}]
// This is a comment
std::size_t attempts = 10;
double timeout = 0.1;
bool found_ik = kinematic_state->setFromIK(joint_model_group, end_effector_state, attempts, timeout);


if (found_ik)
{
  kinematic_state->copyJointGroupPositions(joint_model_group, joint_values);
  for (std::size_t i = 0; i < joint_names.size(); ++i)
  {
    ROS_INFO("Joint $\%$s: $\%$f", joint_names[i].c_str(), joint_values[i]);
  }
}
else
{
  ROS_INFO("Did not find IK solution");
}
\end{lstlisting}

\vspace{0.5cm}

\begin{lstlisting}[caption={Armteil}]
<?xml version="1.0"?>
<robot name="RoboterBeispiel1">
  <link name="ErsterArmteil">
    <visual>
      <geometry>
        <cylinder length="0.5" radius="0.3"/>
      </geometry>
    </visual>
  </link>
</robot>
\end{lstlisting}
\vspace{0.5cm}
