%%%%%%%%%%%%%%%%%%%%%%%%%%%%%%%%%%%%%%%%%%%%%%%%%%%%%%%%%%%%%%%%%%%%%%%%%%%%%%%%%%%%%%%%%%%%%%%%%%%%%%%%%%%%%%%%%%%%%%%%%%%%%%%%%%%%%%%%%%%%%%%%%%%%%%%%%%%%%%%%%%%%%%%%%%%%%%%%%%%%%%%%%%%%%%%%%%%%%%%%%%%%%%%%%%%%%%%%%%%%%%%%%%%%%%%%%%%%%%%%%%%%%%%%
%%%&& Grundsätzliches zur fachwissenschaftlichen Arbeit %%%%%
%%%%%%%%%%%%%%%%%%%%%%%%%%%%%%%%%%%%%%%%%%%%%%%%%%%%%%%%%%%%%%%%%%%%%%%%%%%%%%%%%%%%%%%%%%%%%%%%%%%%%%%%%%%%%%%%%%%%%%%%%%%%%%%%%%%%%%%%%%%%%%%%%%%%%%%%%%%%%%%%%%%%%%%%%%%%%%%%%%%%%%%%%%%%%%%%%%%%%%%%%%%%%%%%%%%%%%%%%%%%%%%%%%%%%%%%%%%%%%%%%%%%%%%%

\chapter{Grundsätzliches zur fachwissenschaftlichen Arbeit}
\label{cap:Grundsätzliches zur fachwissenschaftlichen Arbeit}

%%%%%%%%%%%%%%%%%%%%%%%%%%%%%%%%%%%%%%%%%%%%%%%%%%%%%%%%%%%%%%%%%%%%%%%%%%%%%%%%%%%%%%%%%%%%%%%%%%%%%%%%%%%%%%%%%%%%%%%%%%%%%%%%%%%%%%%%%%%%%%%%%%%%%%%%%%%%%%%%%%%%%%%%%%%%%%%%%%%%%%%%%%%%%%%%%%%%%%%%%%%%%%%%%%%%%%%%%%%%%%%%%%%%%%%%%%%%%%%%%%%%%%%%
\section{Einleitende Hinweise}
\label{sec:Einleitende Hinweise}
%%%%%%%%%%%%%%%%%%%%%%%%%%%%%%%%%%%%%%%%%%%%%%%%%%%%%%%%%%%%%%%%%%%%%%%%%%%%%%%%%%%%%%%%%%%%%%%%%%%%%%%%%%%%%%%%%%%%%%%%%%%%%%%%%%%%%%%%%%%%%%%%%%%%%%%%%%%%%%%%%%%%%%%%%%%%%%%%%%%%%%%%%%%%%%%%%%%%%%%%%%%%%%%%%%%%%%%%%%%%%%%%%%%%%%%%%%%%%%%%%%%%%%%%

Dieser Leitfaden ist für alle Studierenden an der Fakultät Maschinenbau der Ruhr-Universität Bochum gedacht, die im Rahmen ihres Studiums eine Semester-, Bachelor- oder Masterarbeit am Lehrstuhl für Produktionssysteme schreiben. In ihm sind, neben den \textbf{allgemeinen Richtlinien, einige Hilfestellungen und Tipps} enthalten, die das Erstellen einer solchen fachwissenschaftlichen Arbeit erleichtern und somit vor allem für diejenigen eine Hilfe darstellen, die mit dem Bearbeiten von wissenschaftlichen Arbeiten noch nicht viel Erfahrung haben. Nicht enthalten sind die detaillierten Beurteilungskriterien mit ihrer Gewichtung. Am Ende des Leitfadens ist lediglich eine grobe Übersicht zu finden. Wir empfehlen deshalb, bei Beginn der Arbeit diese mit dem jeweiligen Betreuer zu besprechen. Einen\textbf{ Anspruch auf Vollständigkeit erhebt dieser Leitfaden nicht}, sondern er soll erste Klarheiten über Formalia und Richtlinien geben.
Bei Problemen, Anmerkungen, über den Leitfaden hinausgehenden Fragen oder Fragen zu Inhalten und genaueren Formalitäten sollte der jeweilige Betreuer angesprochen werden.

In den jeweiligen Prüfungsordnungen ist festgelegt, dass die fachwissenschaftliche Arbeit zeigen soll, \emph{"`dass die Kandidatin bzw. der Kandidat in der Lage ist, innerhalb einer vorgegebenen Frist eine anspruchsvolle Fragestellung unter Anwendung der im Bachelor- oder im Masterstudium erworbenen Methoden selbstständig zu bearbeiten."'} Ferner ist der Student dazu angehalten, unabhängig von seiner Prüfungsordnung, seine fachwissenschaftliche Arbeit (gilt für Masterarbeiten) am Lehrstuhl für Produktionssysteme im Rahmen eines 20-minütigen Vortrages zu präsentieren und in einer anschließenden 10-minütigen Diskussion zu erklären. Das Ergebnis der Präsentation geht mit einem Gewicht von 10\,\% in die Abschlussnote der Arbeit ein.


%%%%%%%%%%%%%%%%%%%%%%%%%%%%%%%%%%%%%%%%%%%%%%%%%%%%%%%%%%%%%%%%%%%%%%%%%%%%%%%%%%%%%%%%%%%%%%%%%%%%%%%%%%%%%%%%%%%%%%%%%%%%%%%%%%%%%%%%%%%%%%%%%%%%%%%%%%%%%%%%%%%%%%%%%%%%%%%%%%%%%%%%%%%%%%%%%%%%%%%%%%%%%%%%%%%%%%%%%%%%%%%%%%%%%%%%%%%%%%%%%%%%%%%%
\section{Das Thema der Arbeit}
\label{sec:Das Thema der Arbeit}
%%%%%%%%%%%%%%%%%%%%%%%%%%%%%%%%%%%%%%%%%%%%%%%%%%%%%%%%%%%%%%%%%%%%%%%%%%%%%%%%%%%%%%%%%%%%%%%%%%%%%%%%%%%%%%%%%%%%%%%%%%%%%%%%%%%%%%%%%%%%%%%%%%%%%%%%%%%%%%%%%%%%%%%%%%%%%%%%%%%%%%%%%%%%%%%%%%%%%%%%%%%%%%%%%%%%%%%%%%%%%%%%%%%%%%%%%%%%%%%%%%%%%%%%

Es gibt am Lehrstuhl für Produktionssysteme eine Vielzahl an Ausschreibungen für fachwissenschaftliche Arbeiten, sowohl auf der Homepage\footnote{http://www.lps.ruhr-uni-bochum.de/} als auch an den Informationstafeln des Lehrstuhls. Selbstverständlich können Sie den Mitarbeitern auch eigene Themen vorschlagen. Nach \textbf{Zustimmung des Fachvertreters} müssen Sie auf jeden Fall stets das Thema verbindlich beim Prüfungsamt anmelden (Bachelor- und Masterarbeiten).


%%%%%%%%%%%%%%%%%%%%%%%%%%%%%%%%%%%%%%%%%%%%%%%%%%%%%%%%%%%%%%%%%%%%%%%%%%%%%%%%%%%%%%%%%%%%%%%%%%%%%%%%%%%%%%%%%%%%%%%%%%%%%%%%%%%%%%%%%%%%%%%%%%%%%%%%%%%%%%%%%%%%%%%%%%%%%%%%%%%%%%%%%%%%%%%%%%%%%%%%%%%%%%%%%%%%%%%%%%%%%%%%%%%%%%%%%%%%%%%%%%%%%%%%
\section{Anmeldung und Bearbeitungsdauer}
\label{sec:Anmeldung und Bearbeitungsdauer}
%%%%%%%%%%%%%%%%%%%%%%%%%%%%%%%%%%%%%%%%%%%%%%%%%%%%%%%%%%%%%%%%%%%%%%%%%%%%%%%%%%%%%%%%%%%%%%%%%%%%%%%%%%%%%%%%%%%%%%%%%%%%%%%%%%%%%%%%%%%%%%%%%%%%%%%%%%%%%%%%%%%%%%%%%%%%%%%%%%%%%%%%%%%%%%%%%%%%%%%%%%%%%%%%%%%%%%%%%%%%%%%%%%%%%%%%%%%%%%%%%%%%%%%%

Sobald die Voraussetzungen erfüllt sind, kann beim Prüfungsamt die Freigabe eingeholt werden. Diese Freigabe überreichen Sie Ihrem Fachvertreter. Er trägt dort das Thema sowie das Ausgabedatum der fachwissenschaftlichen Arbeit ein. Voraussetzung hierfür ist die Unterzeichnung einer Verpflichtungserklärung, die Ihnen von Ihrem Betreuer ausgehändigt wird. Mit der vollständig ausgefüllten Freigabe erhalten Sie beim Prüfungsamt das Abgabedatum der fachwissenschaftlichen Arbeit. Die fachwissenschaftliche Arbeit sollte etwa den folgenden Umfang haben (Richtwert):
\begin{itemize}
	\item \textbf{Semesterarbeit}: 40 Seiten gemäß Formatvorlage
	\item \textbf{Bachelorarbeit}: 40 bis 60 Seiten gemäß Formatvorlage
	\item \textbf{Masterarbeit}: 80 bis 100 Seiten gemäß Formatvorlage 
\end{itemize}
Deckblatt, Anhänge und Verzeichnisse werden hierbei nicht mitgezählt. Die Verteilung von Theorie- und Praxis- bzw. Eigenanteil in der fachwissenschaftlichen Arbeit sollten etwa bei 30 zu 70\,\% liegen. Die Arbeit kann in deutscher oder englischer Sprache verfasst werden. Bearbeitungsdauer, frühestmögliche Abgabe der jeweiligen Arbeit sowie Verlängerungs- und Wiederholungsregelungen entnehmen Sie bitte der relevanten Prüfungsordnung.


%%%%%%%%%%%%%%%%%%%%%%%%%%%%%%%%%%%%%%%%%%%%%%%%%%%%%%%%%%%%%%%%%%%%%%%%%%%%%%%%%%%%%%%%%%%%%%%%%%%%%%%%%%%%%%%%%%%%%%%%%%%%%%%%%%%%%%%%%%%%%%%%%%%%%%%%%%%%%%%%%%%%%%%%%%%%%%%%%%%%%%%%%%%%%%%%%%%%%%%%%%%%%%%%%%%%%%%%%%%%%%%%%%%%%%%%%%%%%%%%%%%%%%%%
\section{Betreuung der Arbeit}
\label{sec:Betreuung der Arbeit}
%%%%%%%%%%%%%%%%%%%%%%%%%%%%%%%%%%%%%%%%%%%%%%%%%%%%%%%%%%%%%%%%%%%%%%%%%%%%%%%%%%%%%%%%%%%%%%%%%%%%%%%%%%%%%%%%%%%%%%%%%%%%%%%%%%%%%%%%%%%%%%%%%%%%%%%%%%%%%%%%%%%%%%%%%%%%%%%%%%%%%%%%%%%%%%%%%%%%%%%%%%%%%%%%%%%%%%%%%%%%%%%%%%%%%%%%%%%%%%%%%%%%%%%%

Bei inhaltlichen Fragen kommen Sie bitte in die Sprechstunde Ihres Betreuers; kürzere Fragen können Sie jederzeit auch per E-Mail oder telefonisch klären.
Eine Sprechstunde ist notwendig,
\begin{itemize}
	\item bei der Wahl des Themas,
	\item zur Besprechung einer ersten Gliederung,
	\item bei Bedarf immer dann, wenn es Probleme, Unsicherheiten und Fragen gibt, die Sie nicht auf anderem Wege (Gespräch mit Anderen, Besuch von Tutorien etc.) lösen können.
\end{itemize}
Des Weiteren ist es ratsam, einen \textbf{regelmäßigen Besprechungstermin} (max. 30 Minuten; Bachelorarbeit: zweiwöchig, Masterarbeit: vierwöchig) mit dem Betreuer zu vereinbaren. In diesem Termin sollte der Student zunächst aktiv seinen aktuellen Arbeitsstand vorstellen. Anschließend kann dann gemeinsam über mögliche Probleme und das weitere Vorgehen disktuiert werden.


%%%%%%%%%%%%%%%%%%%%%%%%%%%%%%%%%%%%%%%%%%%%%%%%%%%%%%%%%%%%%%%%%%%%%%%%%%%%%%%%%%%%%%%%%%%%%%%%%%%%%%%%%%%%%%%%%%%%%%%%%%%%%%%%%%%%%%%%%%%%%%%%%%%%%%%%%%%%%%%%%%%%%%%%%%%%%%%%%%%%%%%%%%%%%%%%%%%%%%%%%%%%%%%%%%%%%%%%%%%%%%%%%%%%%%%%%%%%%%%%%%%%%%%%
\section{Erarbeitung eines groben Zeitplans}
\label{sec:Erarbeitung eines groben Zeitplans}
%%%%%%%%%%%%%%%%%%%%%%%%%%%%%%%%%%%%%%%%%%%%%%%%%%%%%%%%%%%%%%%%%%%%%%%%%%%%%%%%%%%%%%%%%%%%%%%%%%%%%%%%%%%%%%%%%%%%%%%%%%%%%%%%%%%%%%%%%%%%%%%%%%%%%%%%%%%%%%%%%%%%%%%%%%%%%%%%%%%%%%%%%%%%%%%%%%%%%%%%%%%%%%%%%%%%%%%%%%%%%%%%%%%%%%%%%%%%%%%%%%%%%%%%

Neben einer Zusammenarbeit mit dem jeweiligen Betreuer wird empfohlen, sich von Beginn an selbst einen groben Zeitplan sowie eine Gliederung der Arbeit zu erstellen.
Abbildung \ref{fig:Vorgehensweise} zeigt einige grundsätzliche Bausteine, welche für die Erstellung einer fachwissenschaftlichen Arbeit notwendig sind.

% Abbildung: Vorgehensweise zum Erstellen einer Studienarbeit
\begin{figure}[H] % H = here
	\centering
	\begin{overpic}[width=0.7\textwidth]
		{Figures/Vorgehensweise.png}
	\end{overpic}
	\caption{Vorgehensweise zum Erstellen einer Studienarbeit}
	\label{fig:Vorgehensweise}
\end{figure}

Ihre Einbettung sollte nach einem individuell gestalteten und sinnvollen Konzept erfolgen. Dabei muss auch ein Zeitpuffer für unvorhersehbare Ereignisse mit einbezogen werden. Insbesondere bei versuchslastigen Arbeiten sollten mögliche Wartezeiten durch Lieferengpässe oder Defekte an der Versuchseinrichtung mit einbezogen werden. Diese können teilweise mehrere Wochen betragen. Abbildung \ref{fig:Zeitplanung} zeigt ein Beispiel für die Umsetzung eines Zeitplanes. Dieser sollte jedoch in der Regel deutlich detaillierter ausgeführt sein.

% Abbildung: Beispiel für eine Zeitplanung
\begin{figure}[t] % t = top
	\centering
	\begin{overpic}[width=0.7\textwidth]
		{Figures/Zeitplanung.png}
	\end{overpic}
	\caption{Beispiel für eine Zeitplanung}
	\label{fig:Zeitplanung}
\end{figure}


%%%%%%%%%%%%%%%%%%%%%%%%%%%%%%%%%%%%%%%%%%%%%%%%%%%%%%%%%%%%%%%%%%%%%%%%%%%%%%%%%%%%%%%%%%%%%%%%%%%%%%%%%%%%%%%%%%%%%%%%%%%%%%%%%%%%%%%%%%%%%%%%%%%%%%%%%%%%%%%%%%%%%%%%%%%%%%%%%%%%%%%%%%%%%%%%%%%%%%%%%%%%%%%%%%%%%%%%%%%%%%%%%%%%%%%%%%%%%%%%%%%%%%%%
\section{Abgabe der Arbeit}
\label{sec:Abgabe der Arbeit}
%%%%%%%%%%%%%%%%%%%%%%%%%%%%%%%%%%%%%%%%%%%%%%%%%%%%%%%%%%%%%%%%%%%%%%%%%%%%%%%%%%%%%%%%%%%%%%%%%%%%%%%%%%%%%%%%%%%%%%%%%%%%%%%%%%%%%%%%%%%%%%%%%%%%%%%%%%%%%%%%%%%%%%%%%%%%%%%%%%%%%%%%%%%%%%%%%%%%%%%%%%%%%%%%%%%%%%%%%%%%%%%%%%%%%%%%%%%%%%%%%%%%%%%%

%%%%%%%%%%%%%%%%%%%%%%%%%%%%%%%%%%%%%%%%%%%%%%%%%%%%%%%%%%%%%%%%%%%%%%%%%%%%%%%%%%%%%%%%%%%%%%%%%%%%%%%%%%%%%%%%%%%%%%%%%%%%%%%%%%%%%%%%%%%%%%%%%%%%%%%%%%%%%%%%%%%%%%%%%%%%%%%%%%%%%%%%%%%%%%%%%%%%%%%%%%%%%%%%%%%%%%%%%%%%%%%%%%%%%%%%%%%%%%%%%%%%%%%%
\subsection{Gedruckte Exemplare}
\label{sub:Gedruckte Exemplare}

Für jede Arbeit wird ein \textbf{fester (!) Abgabetermin} vereinbart, der einzuhalten ist.
Bei verspäteter Abgabe kann die Arbeit nicht anerkannt werden.
Als Eingang der Arbeit zählt das Datum des Stempels des Prüfungsamtes.
\textbf{Drei Exemplare} der Arbeit sind am Lehrstuhl abzugeben.
Es ist jedoch im Einzelfall immer zu klären, wie viele Exemplare eingereicht werden müssen.
Die Arbeiten werden \textbf{zweiseitig} gedruckt. Zu verwenden sind weiße DIN A4 Blätter.
Ein Druck in Farbe ist notwendig, wenn es zum Verständnis der Abbildungen beiträgt.
Es empfiehlt sich, die Abbildungen so zu gestalten, dass Sie auch in Graustufen interpretiert werden können.
Die Arbeiten müssen in gebundener Form (Vorderseite: Folie, Rückseite: dunkelblauer Karton) abgegeben werden.
Dabei soll folgende Reihenfolge eingehalten werden:
\begin{enumerate}
	\item Folie des Einbundes
	\item Deckblatt
	\item (ggf. Vorwort / Danksagung)
	\item Aufgabenstellung
	\item Kurzfassung (und ggf. Abstract)
	\item Inhaltsverzeichnis
	\item Verzeichnis häufig verwendeter Symbole
	\item Kapitel 1 - x
	\item Literaturverzeichnis
	\item Abbildungsverzeichnis
	\item Tabellenverzeichnis
	\item Anhang
	\item Eidesstattliche Erklärung
	\item (ggf. Sperrvermerk)
	\item (ggf. leere Seite)
	\item Rückseite des Einbundes
\end{enumerate}
Das endgültige Deckblatt hat ein definiertes Format. Die Vorlage (PDF und Papiervorlage) wird durch den Betreuer bereitgestellt. Die Aufgabenstellung wird ebenfalls durch den Betreuer übergeben.

Vor dem offiziellen Abgabetermin beim Prüfungsamt (ca. zwei Wochen vorher) sollte die "`\emph{vorfinale}"' Version der Arbeit an den Betreuer geschickt werden, sodass noch genügend Zeit für Korrekturmaßnahmen vorhanden ist.
Des Weiteren sollten Sie vor dem endgültigen Druck nochmals die \emph{finale} Version an den Betreuern schicken und alle Modalitäten zum Drucken der Arbeit (Anzahl der Exemplare, Deckblatt, zweiseitiger/farbiger Druck etc.) besprechen.

%%%%%%%%%%%%%%%%%%%%%%%%%%%%%%%%%%%%%%%%%%%%%%%%%%%%%%%%%%%%%%%%%%%%%%%%%%%%%%%%%%%%%%%%%%%%%%%%%%%%%%%%%%%%%%%%%%%%%%%%%%%%%%%%%%%%%%%%%%%%%%%%%%%%%%%%%%%%%%%%%%%%%%%%%%%%%%%%%%%%%%%%%%%%%%%%%%%%%%%%%%%%%%%%%%%%%%%%%%%%%%%%%%%%%%%%%%%%%%%%%%%%%%%%
\subsection{Übergabe digitaler Daten}
\label{sub:Übergabe digitaler Daten}

Neben der gedruckten Version, müssen Sie zusätzlich Ihre Arbeit und Ihre Präsentation als digitale Version (LaTeX-Ordner, Powerpoint und PDF) sowie weitere Unterlagen (Abbildungen, Tabellen, Literaturdokumente) abgeben.
Die digitalen Daten können entweder auf einem USB-Stick abgegeben oder im eigenen Seafile\footnote{\url{https://seafile.noc.ruhr-uni-bochum.de}}-Ordner hochgeladen werden.
Alle Abbildungen, Tabellen usw. müssen zusätzlich in nativer Form gespeichert werden.
Wenn Sie also bspw. eine Abbildung mit MS-PowerPoint erstellen, dann müssen Sie die originale Powerpoint-Datei im entsprechenden Verzeichnis neben der Abbildung speichern.
Die in der Arbeit verwendeten Literaturquellen werden ebenfalls im PDF-Format hinterlegt.
Falls in der Arbeit weitere Daten (z.\,B. CAD-Modelle, Simulationen oder Quellcode, Messdaten) erzeugt wurden, so müssen sie ebenfalls hier abgelegt werden.


%%%%%%%%%%%%%%%%%%%%%%%%%%%%%%%%%%%%%%%%%%%%%%%%%%%%%%%%%%%%%%%%%%%%%%%%%%%%%%%%%%%%%%%%%%%%%%%%%%%%%%%%%%%%%%%%%%%%%%%%%%%%%%%%%%%%%%%%%%%%%%%%%%%%%%%%%%%%%%%%%%%%%%%%%%%%%%%%%%%%%%%%%%%%%%%%%%%%%%%%%%%%%%%%%%%%%%%%%%%%%%%%%%%%%%%%%%%%%%%%%%%%%%%%
\section{Bewertung studentischer Arbeiten}
\label{sec:Bewertung studentischer Arbeiten}
%%%%%%%%%%%%%%%%%%%%%%%%%%%%%%%%%%%%%%%%%%%%%%%%%%%%%%%%%%%%%%%%%%%%%%%%%%%%%%%%%%%%%%%%%%%%%%%%%%%%%%%%%%%%%%%%%%%%%%%%%%%%%%%%%%%%%%%%%%%%%%%%%%%%%%%%%%%%%%%%%%%%%%%%%%%%%%%%%%%%%%%%%%%%%%%%%%%%%%%%%%%%%%%%%%%%%%%%%%%%%%%%%%%%%%%%%%%%%%%%%%%%%%%%

Die nachfolgenden Kriterien werden bei der Beurteilung von studentischen Arbeiten genutzt.

\noindent
\begin{itemize}
	\item \textbf{Arbeitstechnik und Einsatz:} Selbstständigkeit; Zielstrebigkeit; Abstraktionsfähigkeit; Arbeitstempo
	\item \textbf{Ergebnisse und Ausarbeitung:} Erfüllung der Aufgabenstellung; Qualität; Inhalt
	\item \textbf{Formales:} Methodisches Vorgehen und Struktur; Ergänzung der textuellen Darstellung (Abbildungen, Tabellen); Klarheit der Darstellung und sprachliche Gestaltung; Literaturnachweise
\end{itemize}


%%%%%%%%%%%%%%%%%%%%%%%%%%%%%%%%%%%%%%%%%%%%%%%%%%%%%%%%%%%%%%%%%%%%%%%%%%%%%%%%%%%%%%%%%%%%%%%%%%%%%%%%%%%%%%%%%%%%%%%%%%%%%%%%%%%%%%%%%%%%%%%%%%%%%%%%%%%%%%%%%%%%%%%%%%%%%%%%%%%%%%%%%%%%%%%%%%%%%%%%%%%%%%%%%%%%%%%%%%%%%%%%%%%%%%%%%%%%%%%%%%%%%%%%
\section{Installation}
\label{sec:Installation}
%%%%%%%%%%%%%%%%%%%%%%%%%%%%%%%%%%%%%%%%%%%%%%%%%%%%%%%%%%%%%%%%%%%%%%%%%%%%%%%%%%%%%%%%%%%%%%%%%%%%%%%%%%%%%%%%%%%%%%%%%%%%%%%%%%%%%%%%%%%%%%%%%%%%%%%%%%%%%%%%%%%%%%%%%%%%%%%%%%%%%%%%%%%%%%%%%%%%%%%%%%%%%%%%%%%%%%%%%%%%%%%%%%%%%%%%%%%%%%%%%%%%%%%%

Der vorliegende Leitfaden ist gleichzeitig auch eine LaTeX-Vorlage, die verwendet und angepasst werden soll.
Um mit LaTeX zu arbeiten, werden zwei Dinge benötigt:
Einerseits die LaTeX-Software und andererseits eine Entwicklungsumgebung, mit der der LaTeX-Code eingegeben und die Umsetzung in ein fertig gesetztes Dokument angestoßen wird.


\subsubsection*{LaTeX-Distributionen}
Die LaTeX-Software besteht aus den TeX/LaTeX-Programmen, Schriften, Skripten und Zusatzprogrammen.
Der einfachste Weg, um die LaTeX-Software zu installieren, ist eine Distribution zu wählen.
Diese installiert alle notwendigen Programme und die gebräuchlichsten Zusätze.
Die bekannteste Distribution ist \emph{MiKTeX}\footnote{\url{https://miktex.org/download}}.

\subsubsection*{Entwicklungsumgebungen}
LaTeX-Dokumente werden im Allgemeinen mittels einer Entwicklungsumgebung erstellt.
Zwar kann man LaTeX-Dokumente auch mit Hilfe eines einfachen Texteditors und der Kommandozeile erstellen, doch bieten die auf LaTeX angepassten Programme mehr Funktionen und Komfort.
Viele LaTeX-Befehle, Sonderzeichen und Symbole sind über die grafische Benutzeroberfläche zugänglich, und teilweise lassen sich darüber auch einfache Tabellen erstellen.
Für große Projekte bieten Entwicklungsumgebungen eine Verwaltung und Strukturdarstellung.
An dieser Stelle wird die Entwicklungsumgebung \emph{TeXnicCenter}\footnote{\url{http://www.texniccenter.org/download/}} empfohlen.



