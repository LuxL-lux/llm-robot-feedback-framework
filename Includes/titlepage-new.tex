\begin{titlepage}
  \rubflama % Apply RUB Flama font for the entire title page
  \thispagestyle{empty}
  \noindent
  \begin{minipage}{0.5\textwidth}% adapt widths of minipages to your needs
    \includegraphics[width=56.9mm]{Figures/RUB-Logo-blau.png}
  \end{minipage}%
  \hfill%
  \begin{minipage}{0.5\textwidth}\raggedleft
    {\fontsize{9}{12}\selectfont \textcolor{rubgreen}{\bfseries FAKULTÄT FÜR
    MASCHINENBAU}}\par
    {\fontsize{8}{12}\selectfont \bfseries
      Institute Product and Service Engineering\\
      Lehrstuhl für Produktionssysteme\\
    PROF.\,DR.-ING.\,BERND KUHLENKÖTTER}%
  \end{minipage}
  \begin{center}
    {\bfseries \fontsize{16}{12}\selectfont Bachelor-Thesis}\par
    {Lukas Lux}
  \end{center}
  \fontsize{10}{12}\selectfont Matrikelnummer: \quad 1080 18 240012\par
  Prüfungsordnung: \quad BPO Sales Engineering and Product
  Management 2013
  \bigbreak
  \fontsize{12}{12}\selectfont{\bfseries Thema: \quad Entwicklung
    eines Frameworks zur simulationsbasierten Validierung
  LLM-generierten Robotercodes in Unity}
  \bigbreak
  \fontsize{10}{12}\selectfont Die Programmierung von Robotern ist derzeit ein
  komplexer und  zeitintensiver Prozess, der in der Regel spezielles Fachwissen
  erfordert. Um diese Hürde zu verringern, erforscht die Wissenschaft
  neue Ansätze zur Beschleunigung der Arbeitsprozesse. Generative
  KI-Systeme, insbesondere Large-Language-Modelle (LLMs), bieten
  dabei ein vielversprechendes Potenzial.
  Eine zentrale Herausforderung besteht jedoch im sicheren Einsatz
  von KI-generiertem Quellcode. Am LPS wird deshalb untersucht, wie
  Simulationstools genutzt werden können, um generierte Lösungen
  zunächst virtuell zu erproben. Die dabei gewonnenen Erkenntnisse
  sollen wiederum in die KI-Systeme zurückgeführt werden. Damit dies
  gelingt, ist eine Überführung der Simulationsergebnisse in eine für
  LLMs verständliche, textuelle bzw. natürlichsprachliche Form erforderlich.
  \bigbreak
  Ziel dieser Arbeit ist es, eine solche Überführung anhand eines
  Beispielprozesses und ausgewählter Simulationsparameter zu
  entwickeln, zu implementieren und methodisch zu evaluieren.
  \bigbreak
  Im Einzelnen sollen folgende Punkte bearbeitet werden:
  \begin{itemize}
    \item Analyse des aktuellen Stands der Technik in den Bereichen
      Roboterprogrammierung sowie Bewertung von Robotersimulationen.
    \item Konzeption einer Basis-Architektur zur Übertragung von
      Erkenntnissen aus Simulationen in eine geeignet
      weiterzuverarbeitende Form (z. B. JSON).
    \item Programmiertechnische Umsetzung der Architektur in Unity
      für ein definiertes Beispielszenario.
    \item Methodische Erprobung der entwickelten Lösung und Bewertung
      der erzielten Ergebnisse.
  \end{itemize}
  \bigbreak
  Die Arbeit leistet einen wichtigen Beitrag zum Gelingen des
  Projektes XYZ am Lehrstuhl
  für Produktionssysteme.
  \bigbreak
  Ausgabedatum: \quad 16.06.2025\par
  Betreuer: M. Sc. Daniel Syniawa

\end{titlepage}
