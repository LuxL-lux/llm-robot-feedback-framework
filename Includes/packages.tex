% === Document Layout & Geometry ===
\usepackage{geometry}
\usepackage{microtype} % Verbesserter Randausgleich bei Silbentrennung

% === Fonts (LuaLaTeX) ===
\usepackage{fontspec}

% Set Times New Roman as main font (12pt is set in document class)
\setmainfont{Times New Roman}

% Define RUB Flama font family
\newfontfamily\rubflama[
  Path = Includes/fonts/,
  Extension = .ttf,
  UprightFont = RUB-Flama-Regular,
  BoldFont = RUB-Flama-Bold,
  ItalicFont = RUB-Flama-Italic,
  FontFace = {l}{n}{RUB-Flama-Light-Regular},
  FontFace = {l}{it}{RUB-Flama-Light-Italic}
]{RUB-Flama}

% === Language & Typography ===
\usepackage[german,ngerman]{babel}
\usepackage{csquotes}

% === Mathematics ===
\usepackage{amsmath}
\usepackage{amsthm}
\usepackage{amscd}
\usepackage{amsfonts}
\usepackage{amssymb}
\usepackage{bbm}

% === Graphics & Colors ===
\usepackage{xcolor}
\usepackage{graphicx}
\usepackage{svg}
\usepackage{rotating}
\usepackage[percent]{overpic}
\usepackage{pgfplots}
\pgfplotsset{compat=1.18}

% === Tables ===
\usepackage{array}
\usepackage{longtable}
\usepackage{booktabs}
\usepackage{makecell}
\usepackage{tabularx}
\usepackage{threeparttable}
\usepackage{multirow}

% === Floats & Captions ===
\usepackage{float}
\usepackage{placeins}
\usepackage[]{caption}
\usepackage[normal]{subfigure}

% === Code Listings ===
\usepackage{listings}
\usepackage{minted}
\setminted[csharp]{style=friendly,breaklines,breakanywhere,fontfamily=tt}
\setminted[json]{style=friendly,breaklines,breakanywhere,fontfamily=tt}

% === Algorithms ===
\usepackage{algorithm}
\usepackage{algorithmic}

% === Bibliography ===
\usepackage[
  style=authoryear,       % Autor-Jahr Stil für kompakte Zitate
  backend=biber,
  autocite=footnote,
  maxcitenames=2,
  mincitenames=1,
  maxbibnames=99,
  uniquelist=false,
  uniquename=false,
  giveninits=true,        % Nur Initialen der Vornamen
  terseinits=true,        % Keine Punkte nach Initialen
  dashed=false
]{biblatex}

% === KOMA-Script & Page Layout ===
\usepackage{tocbasic}
\usepackage[automark]{scrlayer-scrpage}

% === Miscellaneous ===
\usepackage{pifont}
\usepackage{dirtree}
\usepackage[bottom,hang]{footmisc}

% === Hyperlinks (load late) ===
\usepackage{url}
\usepackage[
  pdfauthor={Vor- und Nachname},
  pdftitle={Titel der Arbeit},
  pdfsubject={Bachelor-Thesis, RUB},
  hypertexnames=false,
  pdfdisplaydoctitle
]{hyperref}

\addbibresource{literatur.bib}
% Befehle für f. und ff.
\newcommand{\psq}{\,f.}
\newcommand{\psqq}{\,ff.}

% Befehle für direkte und indirekte Zitate
\newcommand{\vglcite}[2][]{\footnote{Vgl. \citeauthor{#2}
\citeyear{#2}\ifx&#1&\else, S.~#1\fi}}
\newcommand{\directcite}[2][]{\footnote{\citeauthor{#2}
\citeyear{#2}\ifx&#1&\else, S.~#1\fi}}

% Hilfsbefehle für einzelne Zitate ohne Fußnote (für Mehrfachzitate)
\newcommand{\citesingle}[2][]{\citeauthor{#2}
\citeyear{#2}\ifx&#1&\else, S.~#1\fi}

% Befehle für mehrere Zitate in einer Fußnote
\newcommand{\vglcites}[1]{\footnote{Vgl. #1}}
\newcommand{\directcites}[1]{\footnote{#1}}

% "et al." statt "u.a."
\DefineBibliographyStrings{ngerman}{
  andothers = {et al\adddot},
}

\definecolor{rubgreen}{HTML}{8DAE10}
